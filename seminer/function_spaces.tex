\documentclass[uplatex]{jsarticle}
\usepackage{MAstandard}
\DeclareMathOperator{\Int}{Int}
\newcommand{\smooth}{C^{\infty}}
\newcommand{\setinverse}[2]{#1 ^{-1}(#2)}
\DeclareMathOperator{\sign}{sign}
\DeclareMathOperator{\Real}{Re}
\renewcommand{\qedsymbol}{$({}^\ast \scalebox{.8}{>}\raisebox{-.1ex}{\scalebox{.8}{$\triangle$}} \scalebox{.8}{<} )$}


\title{関数空間の定理}
\author{みつば(@mittlear1)}
\date{2020年10月1日}

\begin{document}
\maketitle
\begin{abstract}
  \begin{center}
		関数空間についての重要な定理2つの証明を記した.
	\end{center}
\end{abstract}

\section{Ascoli-Arzel\`{a}の定理}
本節では$K$を空でない位相空間,$(X,d)$を空でない距離空間とする.$C(K,X)$を$K$から$X$への連続写像全体の集合とする.
\subsection{Ascoli-Arzel\`{a}その1}
\begin{definition}
$\mscrA\subset C(K,X)$が同程度連続であるとは,すべての$y\in K$と$\varepsilon > 0$に対して$y$の開近傍$U$が存在し,
\[ \sup_{(f,y^\prime)\in \mscrA\times U} d(f(y),f(y^\prime)) < \varepsilon \]
が成立することをいう.
\end{definition}

\begin{definition}
$\mscrA\subset C(K,X)$が各点相対コンパクトであるとは,すべての$y\in K$で
\[ \mscrA_y = \set{f(y)}{f\in \mscrA} \]
が$X$の相対コンパクト集合であることをいう.
\end{definition}

\begin{theorem}[Ascoli-Arzel\`{a}の定理]\label{AA1}
$K$がコンパクト空間であるとする.このとき$C(K,X)$に距離$\rho$を
\[ \rho(f,g) = \max_{y\in K} d(f(y),g(y)) \]
で定めることができる.これによって$C(K,X)$を距離空間と考えたとき,$\mscrA\subset C(K,X)$について以下は同値である.
\begin{enumarabicp}
\item $\mscrA$は相対コンパクトである.
\item $\mscrA$は同程度連続かつ各点相対コンパクトである.
\end{enumarabicp}
\end{theorem}
\begin{proof}
\begin{subproof}{\IMPLIES{(1)}{(2)}}
$\mscrA$が相対コンパクトであるとする.まず$\mscrA$が同程度連続であることを示す.$\varepsilon>0$を任意にとる.$\mscrA$の全有界性より$f_1,\dots ,f_k\in \mscrA$をうまくとって
\[ \mscrA \subset \bigcup_{i=1}^k B_{\varepsilon/4} (f_i) \]
をみたすようにできる.$y\in K$を任意にとる.$f_i$たちの連続性から,$y$の開近傍$U$を,$y^\prime\in U$ならすべての$1\leq i \leq k$で$d(f_i(y),f_i(y^\prime))<\varepsilon/4$となるようにとることができる.このとき任意に$f\in \mscrA$をとると$f\in B_{\varepsilon/4}(f_i)$となる$i$が存在するので,任意の$y^\prime\in U$に対して
\begin{align*}
    d(f(y),f(y^\prime))&\leq d(f(y),f_i(y)) + d(f_i(y),f_i(y^\prime)) + d(f_i(y^\prime),f(y)) \\
    &\leq \frac{\varepsilon}{4} + \frac{\varepsilon}{4} + \frac{\varepsilon}{4} = \frac{3\varepsilon}{4}
\end{align*}
である.したがって,
\[ \sup_{(f,y^\prime)\in \mscrA\times U} d(f(y),f(y^\prime))\leq \frac{3\varepsilon}{4} < \varepsilon \]
が成立し,これは$\mscrA$が同程度連続であることを示している.

次に$\mscrA$が各点相対コンパクトであることを示す.$y\in K$を任意にとる.$\mscrA_y$内の任意の点列$\{x_n\}_n$が収束部分列を持つことを示せばよい.$\mscrA$の定義から,各$x_n$について$f_n\in \mscrA$を$f_n(y)=x_n$となるようにとれる.こうしてできる$\mscrA$内の点列$\{f_n\}_n$は収束部分列$\{f_{n_k}\}_k$を持つ.この収束先を$f\in C(K,X)$と書くと,$\{x_n\}_n$の部分列$\{x_{n_k}\}_k$は$f(y)$に収束する.
\end{subproof}

\begin{subproof}{\IMPLIES{(2)}{(1)}}
$\mscrA$が同程度連続かつ各点相対コンパクトであるとする.$\mscrA$内の任意の点列$\{f_n\}_n$について収束部分列が存在することを示せばよい.$\mscrA$の同程度連続性から,任意の$y\in K$についてその近傍$U_y$を,$y^\prime\in U_y$ならすべての$g\in \mscrA$で
\[ d(g(y),g(y^\prime)) < \frac{1}{6} \]
になるようにとれる.$K$はコンパクトだから開被覆$\{U_y\}_{y\in K}$の有限部分被覆$\{U_{y_i}\}_{1\leq i\leq l}$がとれる.各$\mscrA_{y_i}$は相対コンパクトだから,$\{f_n\}_n$の部分列$\{f_{n_k}\}_k$をすべての$1\leq i\leq l$で$\{f_{n_k}(y_i)\}_k$が収束するようにとれる.さらにその部分列$\{f_{1,n}\}_n$をとることによって,$m,n\in \setZ_{>0}$ならすべての$1\leq i\leq l$で
\[ d(f_{1,m}(y_i),f_{1,n}(y_i)) < \frac{1}{6} \]
となるようにできる.このとき任意の$y\in K$について$y\in U_{y_i}$となる$i$をとることによって
\begin{align*}
d(f_{1,m}(y),f_{1,n}(y)) &\leq d(f_{1,m}(y),f_{1,m}(y_i)) + d(f_{1,m}(y_i),f_{1,n}(y_i)) + d(f_{1,n}(y_i),f(y)) < \frac{1}{2}
\end{align*}
すなわち
\[ \rho(f_{1,m},f_{1,n}) \leq \frac{1}{2} \]
が分かる.この操作を続けることによって,$k\in \setZ_{>0}$に対して$\{f_{k-1,n}\}_n$の部分列$\{f_{k,n}\}_n$を
\[ \rho(f_{k,m},f_{k,n})\leq 2^{-k} \]
となるようにとれる.$g_n=f_{n,n}$とおくとこれは$C(K,X)$のCauchy列で,$\mscrA$の各点相対コンパクト性から各点収束極限$\map{g}{K}{X}$が存在する.$\{g_n\}_n$がCauchy列であったことから,$g$は$\{g_n\}_n$の一様収束極限であることが分かる.したがって$g\in C(K,X)$であり,$\{g_n\}_n$は$C(K,X)$の中で$g$に収束することが分かる.
\end{subproof}
\end{proof}

\subsection{Ascoli-Arzer\`{a}その2}

\begin{theorem}[Ascoli-Arzel\`{a}の定理]
$S$を$K$の可算稠密部分集合,$X$を完備距離空間とする.同程度連続な$C(K,X)$の部分集合$\mscrA$について,任意の$y\in S$で$\mscrA_{y}$が相対コンパクトであると仮定する.このとき,任意の$\mscrA$内の点列$\{f_n\}_n$は$K$上広義一様収束する部分列を持つ.
\end{theorem}
\begin{proof}
$S$の元を$y_1,y_2,\dots$と番号づけしておく.$\{f_n(y_1)\}_n$が相対コンパクトであることから,$\{f_n\}_n$の部分列$\{f_{1,n}\}_n$を$\{f_{1,n}(y_1)\}_n$が収束するようにとれる.この操作を続けることで,任意の正整数$k$について$\{f_{k,n}\}_n$の部分列$\{f_{k+1,n}\}_n$を$\{f_{k+1,n}(y_{k+1})\}_n$が収束するようにとれる.そこで$g_n=f_{n,n}$とおくと$\{g_n\}_n$は$\{f_n\}_n$の部分列であり,任意の$y\in S$で$\{g_n(z)\}_n$は収束する.

この$\{g_n\}_n$が$K$上広義一様収束することを示す.$L$を任意の$K$のコンパクト集合とする.任意に$\varepsilon>0$をとる.$\mscrA$の同程度連続性より,各$x\in L$の開近傍$U_x$を
\[ \sup_{(f,x^\prime)\in \mscrA\times U_x} d(f(x),f(x^\prime)) < \frac{\varepsilon}{6} \]
が成立するようにとれる.$L$はコンパクトだから,有限個の点$x_1,x_2\dots,x_N$をうまく選んで,$\{U_{x_i}\}_{1\leq i\leq N}$が$L$を被覆するようにできる.$S\cap U_{x_i}$は空でないから,この中から一つ元を選んで$z_i$と名づける.

このとき$N^\prime\in \setZ_{>0}$を十分大きくとると,任意の$m,n>N^\prime$と$1\leq i\leq N$に対して
\[ d(g_m(z_i),g_n(z_i)) < \frac{\varepsilon}{3} \]
となる.したがって$m,n>N^\prime$のとき,$y\in L$ならば$y\in U_{x_i}$となる$1\leq i \leq N$を選ぶことができて
\begin{align*}
    d(g_m(y),g_n(y)) \leq d(g_m(y),g_m(x_i)) + d(g_m(x_i)&,g_m(z_i))+ d(g_m(z_i),g_n(z_i)) \\
    &+ d(g_n(z_i),g_n(x_i)) + d(g_n(x_i),g_n(y)) < \varepsilon
\end{align*}
が成立する.よって$\{g_n\}_n$は$L$上の一様Cauchy列であり,$X$は完備だったからある関数に一様収束する.
\end{proof}

\section{Stone-Weierstrassの定理}
$\setK$で$\setR$または$\setC$を表すことにする.$X$をコンパクト空間とし,$C(X,\setK)$は$\sup$ノルム$\norm{\dot}$でBanach空間になっていると考える.

\begin{definition}
$\mscrA\subset C(X,\setK)$が$C(X,\setK)$の非単位的部分代数であるとは,次の条件が成り立つことをいう.
\begin{enumarabicp}
\item $f,g\in \mscrA$ならば$f+g\in \mscrA.$
\item $f\in \mscrA,\alpha\in \setK$ならば$\alpha f\in \mscrA.$
\end{enumarabicp}
さらに$1\in \mscrA$ならば$\mscrA$は$C(X,\setK)$の部分代数であるという.
\end{definition}

\begin{theorem}[Stone-Weierstrassの定理]\label{SWR}
$C(X,\setR)$の非単位的部分代数$\mscrA$が$C(X,\setR)$の中で稠密であるための必要十分条件は次が成立することである.
\begin{enumarabicp}
\item 任意の$x,y\in X$に対して,ある$f\in \mscrA$で$f(x)\neq f(y)$となるものが存在する.
\item 任意の$x\in X$に対して,$g(x)\neq 0$となる$g\in \mscrA$が存在する.
\end{enumarabicp}
特に部分代数$\mscrA$が(1)をみたせば,$\mscrA$は$C(X,\setR)$の中で稠密である.
\end{theorem}

\begin{theorem}[Stone-Weierstrassの定理]\label{SWC}
$C(X,\setC)$の非単位的部分代数$\mscrA$が$C(X,\setC)$の中で稠密であるための必要十分条件は次が成立することである.
\begin{enumarabicp}
\item 任意の$x,y\in X$に対して,ある$f\in \mscrA$で$f(x)\neq f(y)$となるものが存在する.
\item 任意の$x\in X$に対して,$g(x)\neq 0$となる$g\in \mscrA$が存在する.
\item $f\in \mscrA$ならば$\Bar{f}\in \mscrA$である.
\end{enumarabicp}
ここで,$f\in C(X,\setC)$について,$\Bar{f}$は
\[ \Bar{f}(x) = \overline{f(x)} \]
で定まる$C(X,\setC)$の元である.

特に部分代数$\mscrA$が(1)と(3)をみたせば,$\mscrA$は$C(X,\setC)$の中で稠密である.
\end{theorem}

\begin{remark}
文献によっては上の定理で$X$がHausdorff空間であることを仮定している場合があるが,実はHausdorff性の仮定は$\mscrA$の条件に内包されている.実際,上の定理の仮定(1)をみたす$\mscrA$がとれるためには$X$はHausdorff空間でなければならないことが分かる.
\end{remark}

\begin{lemma}
$\mscrA\subset C(X,\setK)$が非単位的部分代数であるとき,その閉包$\overline{\mscrA}$は非単位的部分代数である.
\end{lemma}
\begin{proof}
明らかである.
\end{proof}

\begin{proposition}[Diniの定理]
$Y$をコンパクト空間とする.$C(Y,\setR)$の点列$\{f_n\}_n$について,任意の$y\in Y$で$\{f_n(y)\}_n$は広義単調増加で上に有界であるとする.このとき,$\{f_n\}_n$はある関数$f$に一様収束する.
\end{proposition}
\begin{proof}
	$\{f_n\}_n$が各点で広義単調増加かつ上に有界であることから,この関数列には各点収束する.その収束先を$f$とおく.
	$\{f_n\}_n$が$f$に一様収束することを示す.$\varepsilon>0$を任意にとる.この$\varepsilon$に対して
		\[A_n=\set{y\in Y}{\abs{f_n(y)-f(y)} < \varepsilon} \]
	と定めると,$f$は$\{f_n\}_n$の各点収束先だから$\{A_n\}_n$は$Y$の開被覆となる.よって$Y$のコンパクト性から
	$Y$は有限個の$A_n$たちで覆える.$\{f_n\}_n$が各点で広義単調増加であることから任意の$n\in \setN$で
	$A_n\subset A_{n+1}$が成立することと合わせると,ある$N\in \setN$が存在して$A_N=Y$となる.
	$\{f_n\}_n$が各点で広義単調増加であることから,これは
		\[\forall n\geq N\ \forall y\in Y\quad \abs{f_n(y)-f(y)} < \varepsilon\]
	を意味している.
\end{proof}

\begin{lemma}\label{root}
$\sqrt{t}\in C([0,1],\setR)$に収束する$t$の多項式の列$\{h_n\}_n$が存在する.
\end{lemma}
\begin{proof}
$\{h_n\}_n$を次の漸化式で定める.
\[ h_{n+1}(t) = h_n(t) + \frac{t-h_n(t)^2}{2},\ h_0=0. \]
この関数列について
\[ h_n(t)\leq \sqrt{t},\  h_n(t)\leq h_{n+1}(t) \]
が成立することが帰納法で示せる.したがってDiniの定理によって$\{h_n\}_n$はある関数に一様収束する.さらにその収束先が$\sqrt{t}$であることが分かる.
\end{proof}

\begin{lemma}
$\mscrA$が$C(X,\setR)$の非単位的部分代数であるとき,$f\in \mscrA$に対して
\[ \abs{f}(x) = \abs{f(x)} \]
で定まる関数$\map{\abs{f}}{X}{\setR}$は$\overline{\mscrA}$の元である.また,$f_1,f_2,\dots,f_n\in \mscrA$ならば$\max\{f_1,f_2\dots,f_n\}$および$\min\{f_1,f_2,\dots f_n\}$は$\overline{\mscrA}$の元である.
\end{lemma}
\begin{proof}
補題\ref{root}の条件をみたす$\{h_n\}_n$をとる.$f\in \mscrA$について$\abs{f}\in\overline{\mscrA}$を示す.$f=0$なら$\abs{f}=0\in\mscrA$だから,$f\neq 0$の場合を考えればよい.このとき$\norm{f}\neq 0$だから
\[ f_n(x) = h_n(\norm{f}^{-2}f(x)^2) \]
はwell-definedな連続関数で,$\mscrA$が非単位的部分代数であることからすべての$n\in \setZ_{>0}$で$f_n\in \mscrA$である.さらに$\{h_n\}_n$が$\sqrt{t}$に一様収束することから$\{f_n\}_n$は$\abs{f}/\norm{f}$に一様収束する.このことと$\overline{\mscrA}$が非単位的部分代数であることから
\[ f = \norm{f}\cdot f/\norm{f}\in \overline{\mscrA} \]
であることが分かる.

後半を示す.$f_1,f_2\in \mscrA$について$\max\{f_1,f_2\},\min\{f_1,f_2\}\in \overline{\mscrA}$であることを示せば十分である.これは
\[ \max\{f_1,f_2\} = \frac{(f_1,f_2)+\abs{f_1-f_2}}{2},\ \min\{f_1,f_2\} = \frac{(f_1+f_2)-\abs{f_1-f_2}}{2} \]
であることと前半の結果から分かる.
\end{proof}

\begin{proof}[定理\ref{SWR}の証明]
まず,任意の異なる2点$x,y\in X$と任意の$a,b\in \setR$について$f(x)=a,f(y)=b$となる$f\in \overline{\mscrA}$が存在することを示す.(1),(2)より$u,v\in \mscrA$を
\[ u(x)\neq u(y),\ v(x)\neq 0 \]
をみたすようにとれる.$\lambda\in \setR$をうまくとると,$h_1=u+\lambda v$について
\[ h_1(x)\neq h_1(y),\ h_1(x)\neq 0 \]
をみたすようにできる.このとき$\alpha = h_1(x)^2-h_1(x)h_1(y)\neq 0$であり,$f_1\in \mscrA$を$f_1=\alpha^{-1}(h_1^2-h_1(y)h_1)$で定義すると
$f_1(x)=1,f_1(y)=0$をみたす.同様に$f_2\in \mscrA$を$f_2(x)=0,f_2(y)=1$となるようにとれるから,$f=af_1+bf_2$と定めればこれが求める関数である.

$f\in C(X,\setR)$を任意にとる.任意の$\varepsilon>0,\ x_0\in X$に対して,$h\in \overline{\mscrA}$を$h(x_0)=f(x_0)$かつすべての$x\in X$で$h(x)>f(x)-\varepsilon$となるようにとれることを示す.$\mscrA(x_0)$を,$g(x_0)=f(x_0)$をみたす$g\in \mscrA$全体の集合とする.各$g\in \mscrA(x_0)$について
\[ M_\varepsilon(g) = \set{x\in X}{g(x)>f(x)-\varepsilon} \]
とおく.初めに示したことから任意の$x\in X$について$g(x)=f(x)$をみたす$g\in \mscrA(x_0)$が存在し,このとき$x\in M_\varepsilon(g)$であるから,$\{M_\varepsilon(g)\}_{g\in\mscrA(x_0)}$は$X$の開被覆である.$X$のコンパクト性から$g_1,g_2,\dots,g_k$をうまく選んで$X=\cup_{i=1}^k M_\varepsilon(g_i)$となるようにできる.$h=\max\{g_1,g_2\dots,g_k\}\in\mscrA$とおけばこれが求めるものである.

$\overline{\mscrA}$の元$h$で,すべての$x\in X$に対して$h(x)>f(x)-\varepsilon$をみたすもの全体を$\overline{\mscrA}(\varepsilon)$と書く.$h\in \overline{\mscrA}(\varepsilon)$に対して
\[ N_\varepsilon(h) = \set{x\in X}{h(x)<f(x)+\varepsilon} \]
とおく.前段落で示したことから,任意に与えられた$x\in X$に対して$\overline{\mscrA}(\varepsilon)$の元$h$を$h(x)=f(x)$が成り立つようにとれる.このことは$\{N_\varepsilon(h)\}_{h\in \overline{\mscrA}(\varepsilon)}$が$X$の開被覆であることを示しており,再び$X$のコンパクト性から$h_1,h_2,\dots,h_l\in \overline{\mscrA}(\varepsilon)$を$X=\cup_{i=1}^l N_\varepsilon(h_i)$が成り立つようにとれる.そこで$f_0=\min\{h_1,h_2,\dots,h_l\}\in \overline{\mscrA}(\varepsilon)$とおくと,とり方から$\norm{f-f_0}\leq \varepsilon$であることが分かる.したがって$f$は$\overline{\mscrA}$の元の一様収束極限として書けるが,$\overline{\mscrA}$は$C(X,\setR)$の閉集合であるから$f\in \overline{\mscrA}$であることが分かる.
\end{proof}

\begin{proof}[定理\ref{SWC}の証明]
$\mscrA$が定理の仮定をみたすとき,
\[ \Real \mscrA  = \set{\Real f}{f\in \mscrA}, \Image \mscrA = \set{\Image f}{f\in \mscrA} \]
について$\Real \mscrA = \Image \mscrA \subset  \mscrA$が成立し,しかも$\Real f\subset C(X,\setR)$は定理\ref{SWR}の仮定をみたす.したがって,任意の$f\in C(X,\setC)$と$\varepsilon>0$について$\Real \mscrA=\Image \mscrA$の元$u,v$をうまくとって
\[ \norm{\Real f-u}<\varepsilon/2,\ \norm{\Image f-v}<\varepsilon/2 \]
が成り立つようにできる.このとき
\[ \norm{(f-(u+iv)} < \varepsilon \]
かつ$u+iv\in \mscrA$である.したがって$\mscrA$は$C(X,\setC)$の中で稠密である.
\end{proof}

\begin{thebibliography}{10}
	\item 内田伏一,『集合と位相』,裳華房,1986.
	\item Rudin, W., \textit{Real and Complex Analysis}, McGraw-Hill Co., New York, 1973.
\end{thebibliography}
\end{document}

2019/10/30 誤植訂正