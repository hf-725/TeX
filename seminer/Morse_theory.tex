\documentclass[uplatex]{jsarticle}
\usepackage{MAstandard}
\DeclareMathOperator{\Int}{Int}

\title{Morse理論}
\author{hf\_725}
\date{2019年7月25日}


\begin{document}
\maketitle

\section{準備}
\begin{definition}
$X$をHausdorff空間とし,$\{\map{\varphi_{\lambda}}{D_{\lambda}}{X}\}_{\lambda\in \Lambda}$を$n_{\lambda}$次元円板$D_{\lambda}$から$X$への連続写像の族とする.
組$(X,\{{\varphi_{\lambda}}\}_{\lambda\in \Lambda})$がCW複体であるとは次の条件を満たすことをいう.
\begin{enumarabicp}
\item すべての$\lambda$に対して$\varphi_{\lambda}$の$\Int D_{\lambda}$への制限は単射であり,$X$は集合として$e_{\lambda} = \varphi_{\lambda}(\Int D_{\lambda})$たちの直和になる.
\item 各$\varphi_{\lambda}(\partial D)$は有限個の$e_{\mu}$たちとしか交わらず,しかもそのような$\mu$に対して必ず$n_{\mu}<n_{\lambda}$が成立する.
\item $X^n = \bigcup_{n_{\lambda}\leq n} e_{\lambda}$とおく.$F\subset X$が$X$の閉集合であることと,すべての$n\in \setZ_{\geq 0}$で$F\cap X^n$が$X^n$の閉集合であることは同値である.
\end{enumarabicp}
\end{definition}
\begin{definition}\label{Hessian}
$M$を$n$次元多様体,$\map{f}{M}{\setR}$を$C^{\infty}$級関数とする.$f$の臨界点$p\in M$に対してこの点でのHesse形式$\map{f_{\ast \ast}}{T_p M\times T_p M}{\setR}$を次のように定義する.$v,w\in T_p M$を任意にとり,$\tilde{v},\tilde{w}$を$p$の近傍で定義されたベクトル場で$\tilde{v}_p = v,\tilde{w}_p = w$をみたすものとするとき
\[ f_{\ast \ast}(v,w) = \tilde{v}_p(\tilde{w}(f)). \]
\end{definition}

\begin{proposition}
上の定義は$\tilde{v},\tilde{w}$のとり方によらない.また,$f_{\ast \ast}$は$T_p M$上の対称双線型形式であり,局所座標をとって行列表示すると$f$のHesse行列となる.
\end{proposition}

\begin{remark}
Hesse形式は$f$の臨界点以外ではwell-definedにならない.
\end{remark}

\begin{definition}
一般に対称双線型形式$\map{b}{V\times V}{\setR}$を考える.$V$の部分空間$W$であって$b$の$W$への制限が不定値になるようなもののうち極大なものの次元を$b$の指数(index)という.また,$V$の部分空間
\[ \set{v\in V}{\forall w\in V b(v,w) = 0} \]
を$b$のnull spaceといい,その次元を退化次数(nullity)という.

また,定義\ref{Hessian}の状況で,臨界点$p\in M$でのHesse形式$f_{\ast \ast}$の指数を単に$f$の$p$での指数という.
\end{definition}

\begin{definition}
$M$上の$C^{\infty}$級関数$f$の臨界点$p$でHesse形式が非退化であるとき,$p$は$f$の非退化臨界点であるという.すべての臨界点が非退化であるとき,$f$はMorse関数であるという.
\end{definition}

\begin{theorem}[Morseの補題]
$\map{f}{M}{R}$を$C^{\infty}$級関数とし,$p\in M$を指数$\lambda$の非退化臨界点とする.このとき,$p$を中心とするチャート$(U;x^1,x^2,\cdots,x^n)$をうまくとると,$U$上
\[ f = f(p)-(x^1)^2-\cdots -(x^{\lambda})^2+(x^{\lambda})^2+\cdots (x^n)^2 \]
と表される.
\end{theorem}

$M$上の関数$f$について$M^a$で$f$の値が$a$以下の点全体の集合を表すことにする.

\begin{theorem}
$f$を$M$上の$C^{\infty}$級関数とする.$a<b$について$f^{-1}([a,b])$はコンパクトでしかもこの中に臨界点はないとする.このとき,$M^a$と$M^b$は微分同相である.また,$M^a$は$M^b$の変位レトラクトである.
\end{theorem}

\begin{theorem}
$f$を$M$上の$C^{\infty}$級関数とする.$c$を$f$の臨界値とし,$f^{-1}(c)$に含まれる臨界点は有限個ですべて非退化であるとする.さらに,$\varepsilon$を十分小さくとれば$f^{-1}([c-\varepsilon,c+\varepsilon])$はコンパクトで$[c-\varepsilon,c+\varepsilon]$の中の臨界値は$c$だけであるようにできると仮定する.さらに各臨界点の指数を$\lambda_1,\dots,\lambda_k$とするとき,十分小さい$\varepsilon$に対して,$M^{c+\varepsilon}$は$M^{c-\varepsilon}$に$\lambda_1$次元円板,$\dots$,$\lambda_k$次元円板をつけた空間とホモトピー同値である.
\end{theorem}

\begin{theorem}\label{CW decomposition}
$f$をMorse関数で,任意の$a\in \setR$で$M^a$がコンパクトになるようなものとする.このとき,$M$は次のようなCW複体$K$とホモトピー同値である.すなわち,$K$の$\lambda$胞体の数は指数$\lambda$の臨界点の個数に等しい.
\end{theorem}

\begin{remark}
上の定理で指数$\lambda$の臨界点は無数にあることもあるが,その場合でも高々可算である.この場合は可算無限個の胞体があると解釈する.
\end{remark}

\begin{theorem}[Morse関数の存在]
定理\ref{CW decomposition}の仮定をみたすようなMorse関数は存在する.
\end{theorem}
\begin{proof}[略証]
Whitneyの埋め込み定理から$M$はEuclid空間に閉集合として埋め込める.$p\in \setR$に対して関数$\map{l_p}{M}{\setR}$を
\[ l_p(q) = \norm{q-p}^2 \]
で定義する.これがほとんどすべての$p$についてMorse関数になることが言える.$M$は閉に埋め込んであったから,この関数が定理\ref{CW decomposition}の仮定をみたすことが分かる.
\end{proof}

\section{無限次元Morse理論}
本章では$(M,g)$を連結な$n$次元Riemann多様体とする.
\subsection{曲線全体の空間}

\begin{definition}
$p,q \in M$とする.区分的$C^{\infty}$級写像$\map{\omega}{[0,1]}{M}$で$\omega(0) = p,\omega(1) = q$をみたすもの全体を$\Omega(M;p,q)$と書く.$\Omega(M)$や$\Omega$と書くこともある.
\end{definition}

\begin{definition}
$0\leq a \leq b\leq 1$とする.$\map{L_a^b}{\Omega}{\setR}$を弧長汎関数すなわち
\[ L_a^b(\omega) = \int_a^b \normlr{\frac{d\omega}{dt}}dt \]
で定義し,特に$L=L_0^1$とおく.また,$\map{E_a^b}{\Omega}{R}$を
\[ E_a^b(\omega) = \int_a^b \normlr{\frac{d\omega}{dt}}^2 dt \]
で定義する.特に$E = E_0^1$とおき,これをエネルギー汎関数という.
\end{definition}

\begin{remark}
文献によっては上の定義の$1/2$倍を$E$と定めている.$E$は$\omega$を粒子の軌跡と思ったときの作用積分のことなので,その意味ではこちらの定義の方が自然である.
\end{remark}

\begin{definition}
$g$から定まる$M$上の距離を$\rho$で表す.すなわち
\[ \rho(p,q) = \inf_{\omega \in \Omega(M;p,q)} L(\omega) \]
とおく.ただし,$L(\omega)$は曲線の長さを表す.このとき,$\Omega(M;p,q)$上の距離$d$を
\[ d(\omega_1, \omega_2) = \max_{t\in [0,1]} \rho(\omega_1, \omega_2)+\parenlr{\int_0^1 \parenlr{\normlr{\frac{d\omega_1}{dt}}-\normlr{\frac{d\omega_2}{dt}}}^2 dt }^{\frac{1}{2}} \]
で定める.
\end{definition}

\begin{proposition}
上の$d$は確かに$\Omega$上の距離関数である.また,この距離から定まる位相に関して$E_a^b$は連続である.
\end{proposition}
\begin{proof}
$d$が距離であることは容易に示せる.後半を示す.三角不等式から
\[\abslr{\parenlr{\int_a^b \parenlr{\normlr{\frac{d\omega_1}{dt}}}^2 dt }^{\frac{1}{2}} - \parenlr{\int_a^b \parenlr{\normlr{\frac{d\omega_2}{dt}}}^2 dt }^{\frac{1}{2}}}\leq \parenlr{\int_a^b \parenlr{\normlr{\frac{d\omega_1}{dt}}-\normlr{\frac{d\omega_2}{dt}}}^2 dt }^{\frac{1}{2}} \]
なので$\sqrt{E_a^b}$は連続である.よって$E_a^b$も連続である.
\end{proof}

\begin{remark}
$d$から定まる$\Omega$上の位相はそこまで不自然なものではない.実際,この空間は$p$と$q$を結ぶ連続曲線全体に$\sup$ノルムで位相を入れた空間とホモトピー同値になる(詳細は略).
\end{remark}

こうして定まった$\Omega(M;p,q)$という空間のホモトピー型を記述するのが目標である.

\subsection{変分公式}

$\Omega(M;p,q)$の位相を調べるときの基本的なアイデアは,この空間を「無限次元の多様体」と思い,$E$を$\Omega$上の「Morse関数」と考えて理論を展開することである.このアイデアをそのまま使うわけではないが,この意識のもとに種々の概念を定義していく.

\begin{definition}
$\map{\pi}{TM}{M}$を自然な射影,$\omega \in \Omega$とする.
\begin{enumarabicp}
\item $\omega$に沿ったベクトル場とは,区分的$C^{\infty}$級写像$\map{W}{[0,1]}{TM}$であって$\pi \circ W = \omega$をみたすもののことをいう.
\item $\omega$に沿ったベクトル場$W$が$W(0)=0,W(1)=0$をみたすとき,$W$は$\omega$に沿った変分ベクトル場であるという.$\omega$に沿った変分ベクトル場全体を$T_\omega \Omega$で表す.$T_\omega \Omega$は自然にベクトル空間の構造を持つ.
\item $O \subset \setR^N$を$0$を含む開集合とする.$\map{\Bar{\alpha}}{O}{\Omega}$が$\omega$の$N$パラメータ変分であるとは,次の条件をみたすことをいう.
\begin{enumromanp}
\item $\Bar{\alpha}(0) = \omega.$
\item $\map{\alpha}{O\times [0,1]}{M}$を$\alpha(u,t) = \Bar{\alpha}(u)(t)$で定めるとき,$[0,1]$の分割
\[ 0 = t_0 \le t_1 \le \dots \le t_k =1 \]
が存在して,$\alpha$は各$O\times [t_i,t_{i+1}]$上$C^{\infty}$級である.
\end{enumromanp}
特に1パラメータ変分のことを単に変分という.
\end{enumarabicp}
\end{definition}

\begin{proposition}
$\omega$の変分$\Bar{\alpha}$に対して$\map{W}{[0,1]}{\Omega}$を
\[ W_t = \frac{d\Bar{\alpha}}{du}(0)_t = \frac{\partial \Bar{\alpha}}{\partial u}(0,t) \]
で定めると$W \in T_\omega \Omega$である.また,任意の$W\in T_\omega \Omega$に対し,$\omega$のある変分$\Bar{\alpha}$で$W = \frac{d\Bar{\alpha}}{du}(0)$をみたすものが存在する.
\end{proposition}
\begin{proof}
前半は条件をチェックすればよい.後半については$[0,1]$のコンパクト性に注意するとある$\varepsilon \geq 0$について
\[ \alpha(u,t) = \exp_{\omega(t)}(uW(t)) \]
が$(-\varepsilon,\varepsilon)\times [0,1]$上でwell-definedになる.この$\alpha$から定まる変分$\Bar{\alpha}$が条件をみたす.
\end{proof}

\begin{theorem}[第一変分公式]
$\Bar{\alpha}$を$\omega \in \Omega$の変分とする.
\[ V = \frac{d\omega}{dt},\ W = \frac{d\Bar{\alpha}}{dt},\ \Delta_{t_i} V = V(t_i+0) - V(t_i-0) \]
とおくと,
\begin{align*}
    \restrlr{\frac{1}{2}\frac{d}{du}E(\Bar{\alpha}(u))}{u=0}
    = -\sum_{i=1}^{k-1} g\parenlr{W_{t_i},\Delta_{t_i} V} - \int_0^1 g\parenlr{W,\frac{DV}{dt}}dt
\end{align*}
である.
\end{theorem}

\begin{definition}
$M$上の曲線$\gamma$が測地線であるとは
\[ \frac{D}{dt}\frac{d\gamma}{dt}=0 \]
をみたすことをいう.
\end{definition}

\begin{proposition}
$\gamma \in \Omega$について,$\gamma$が測地線であることと$\gamma$が$E$の臨界点であることは同値である.ここで,$\gamma$が$E$の臨界点であるとは任意の$\gamma$の変分$\Bar{\alpha}$について
\[ \restrlr{\frac{d}{du}E(\Bar{\alpha}(u))}{u=0} = 0 \]
をみたすことである.
\end{proposition}

\begin{proposition}
$\gamma \in \Omega$が最短測地線ならば,$E$は$\gamma$で最小値をとる.
\end{proposition}

次に,$E$の臨界点におけるHesse形式を定義したい.有限次元多様体のときを思い出せば,次のような定義をするのが妥当であると考えられる.

\begin{definition}
$\gamma \in \Omega$を測地線とする.$\gamma$における$E$のHesse形式$E_{\ast \ast}$を次のように定義する.$W_1,W_2\in T_\gamma \Omega$に対して2パラメータ変分$\Bar{\alpha}$を
\[ W_1 = \frac{\partial \Bar{\alpha}}{\partial u_1}(0,0),\ W_2 = \frac{\partial \Bar{\alpha}}{\partial u_2}(0,0) \]
をみたすようにとり,
\[ E_{\ast \ast}(W_1,W_2) = \restrlr{\frac{\partial^2}{\partial u_1 u_2}E(\Bar{\alpha}(u))}{(u_1,u_2)=(0,0)} \]
とおく.
\end{definition}

この定義が$\Bar{\alpha}$のとり方によらないことは,次の定理から分かる.

\begin{theorem}[第二変分公式]
$\gamma \in \Omega$を測地線,$\Bar{\alpha}$を2パラメータ変分とする.
\[ W_1 = \frac{\partial \Bar{\alpha}}{\partial u_1}(0,0),\ W_2 = \frac{\partial \Bar{\alpha}}{\partial u_2}(0,0) \]
とおくと
\begin{align*}
&\restrlr{\frac{\partial^2}{\partial u_1 \partial u_2}E(\Bar{\alpha}(u_1,u_2))}{(u_1,u_2)=(0,0)} \\ 
= &-\sum_{i=1}^{k-1} g\parenlr{W_2(t_i),\Delta_{t_i} \frac{DW_1}{dt}}
-\int_0^1 g\parenlr{W_2,\frac{D^2W_1}{dt^2}+R(W_1,V)V}dt
\end{align*}
となる.
\end{theorem}

\begin{proposition}
$E_{\ast \ast}$の定義は変分$\Bar{\alpha}$のとり方によらない.また,$E_{\ast \ast}$は対称双線型形式である.
\end{proposition}
\begin{proof}
$E_{\ast \ast}$がwell-definedであることとその双線型性は第二変分公式から分かる.また,
\[ \frac{\partial^2 E}{\partial u_1 \partial u_2} = \frac{\partial^2 E}{\partial u_2 \partial u_1} \]
であるから$E_{\ast \ast}$は対称である.
\end{proof}

\subsection{Jacobi場}

\begin{definition}
$\gamma$を測地線とする.$\gamma$に沿ったベクトル場$J$がJacobi場であるとは,Jacobi方程式
\[ \frac{D^2J}{dt^2}+R\parenlr{J,\frac{d\gamma}{dt}}\frac{d\gamma}{dt} = 0 \]
をみたすことをいう.
\end{definition}

\begin{remark}
Jacobi方程式は二階の線型常微分方程式だから,この方程式の初期値問題の解の定義域は$\gamma$の定義域全体にとれる.また,測地線を一つ固定するときその測地線に沿ったJacobi場全体のなすベクトル空間の次元は$2n$であり,特に有限次元である.
\end{remark}

次の命題は,$E$が$\gamma$で「退化」していることを$\gamma$に沿った変分Jacobi場の存在で特徴づけられることを示している.

\begin{proposition}
$J$を測地線$\gamma$に沿った変分ベクトル場とする.このとき$J$がJacobi場であることと$J$が$E_{\ast \ast}$のnull spaceの元であることは同値である.
\end{proposition}

Jacobi場を考えることは,局所的には測地線の族による変分を考えることと同じである.これを示そう.

\begin{proposition}
$\Bar{\alpha}$を測地線$\gamma \in \Omega$の変分とする.すべての$u$で$\Bar{\alpha}(u)$が測地線ならば,
\[ W = \frac{d\Bar{\alpha}}{du}(0) \]
はJacobi場である.
\end{proposition}
\begin{proof}
まず
\begin{align*}
    0 &= \frac{D}{\partial u} \frac{D}{\partial t} \frac{\partial \alpha}{\partial t} 
      = R\parenlr{\frac{\partial \alpha}{\partial u},\frac{\partial \alpha}{\partial t}}\frac{\partial \alpha}{\partial t}+\frac{D}{\partial t}\frac{D}{\partial u}\frac{\partial \alpha}{\partial t}  
      = R\parenlr{\frac{\partial \alpha}{\partial u},\frac{\partial \alpha}{\partial t}}\frac{\partial \alpha}{\partial t}+\frac{D}{\partial t}\frac{D}{\partial t}\frac{\partial \alpha}{\partial u}
\end{align*}
である.この式に$u=0$を代入して
\[ \frac{D^2W}{dt}+R\parenlr{W,\frac{d\gamma}{dt}}\frac{d\gamma}{dt} =0 \]
を得る.
\end{proof}

\begin{proposition}\label{Jacobi char}
$\map{\pi}{TM}{M}$を自然な射影とする.$M$の開集合$U$と$TM$の開集合$\mscrU$を$\pi(\mscrU) = U$かつ$\map{\exp}{\mscrU}{U\times U}$が微分同相になるようにとる(以下このことを$U$上の2点を結ぶ最短測地線は端点になめらかに依存する,と表現する).さらに$\map{\gamma}{[0,1]}{U}$を測地線とすると,$\gamma$に沿ったJacobi場全体のなすベクトル空間$\mscrJ$と$T_p M\oplus T_q M$は自然に同型である.
\end{proposition}

\subsection{接空間の有限次元近似}

本節では,$E_{\ast\ast}$が負定値となる$T_\gamma \Omega$の部分空間が有限次元空間であることを示す.つまり,$T_\gamma \Omega$の元のうちほとんどは無視してもかまわないのである.この事実は,$\Omega$自体を有限次元多様体で近似できることの証明の第一歩になる.

$\gamma\in \Omega$を測地線とする.各点$\gamma(t)$についてその開近傍$U$を,$U$の2
点を結ぶ最短測地線が端点になめらかに依存するようにとれる.Lebesgueの被覆補題より$[0,1]$の分割
\[ 0 = t_0 \leq t_1 \leq \dots \leq t_k = 1 \]
をすべての$0 \leq i \leq k-1$について$\gamma([t_i,t_{i+1}])$がこのような$U$に含まれるようにできる.$W\in T\gamma \Omega$であって,すべての$0\leq i \leq k-1$で$\restr{W}{[t_i,t_{i+1}]}$が$\restr{\gamma}{[t_i,t_{i+1}]}$に沿ったJacobi場であるようなもの全体のなすベクトル空間を$T_\gamma \Omega(t_0,t_1,\dots ,t_k)$で表す.また,すべての$0\leq i \leq k$で$W(t_i) = 0$をみたす$W\in T_\gamma \Omega$全体のなすベクトル空間を$T^\prime$と書く.

\begin{proposition}
$T_\gamma \Omega$は$T_\gamma \Omega(t_0,t_1,\dots ,t_k)$と$T^\prime$の直和に書ける.また,この二つの部分空間は$E$について直交していて,さらに$T^\prime$上$E$は正定値である.
\end{proposition}
\begin{proof}
$W \in T_\gamma \Omega$に対してJacobi場$J\in T_\gamma \Omega(t_0,t_1,\dots ,t_k)$を
\[ J(t_i) = W(t_i) \]
がすべての$0\leq i\leq k$で成立するようにとる.すると$W-J \in T^\prime$だから,
\[ W = J+(W-J) \in T_\gamma \Omega(t_0,t_1,\dots ,t_k) + T^\prime \]
となる.また,$W\in T_\gamma \Omega(t_0,t_1,\dots ,t_k) \cap T^\prime$なら,各$[t_i,t_{i+1}]$上$W$は端点で消えるJacobi場であるから恒等的に0である.結局$W = 0$となるので,
\[ T_\gamma \Omega = T_\gamma \Omega(t_0,t_1,\dots ,t_k) \oplus T^\prime \]
であることが分かった.

$T_\gamma \Omega(t_0,t_1,\dots ,t_k)$と$T^\prime$が直交していることを示す.$W_1\in T_\gamma \Omega(t_0,t_1,\dots ,t_k),W_2\in T^\prime$とする.$[0,1]$の分割$0=\tau_0\leq \tau_1 \leq \dots \leq \tau_l$を$W_2$がなめらかでないような$t$をすべて含む$0 = t_0 \leq t_1 \leq \dots \leq t_k = 1$の細分とする.このときすべての$0\leq j\leq l$で
\[ \Delta_{\tau_j} \frac{DW_1}{dt} = 0 \]
または
\[ W_2(\tau_j) = 0 \]
が成立する.$W_1$が区分的にJacobi場であることに注意すると,第二変分公式から
\begin{align*}
    E_{\ast \ast}(W_1,W_2) 
   =-\sum_{i=1}^{k-1} g\parenlr{W_2(t_i),\Delta_{t_i} \frac{DW_1}{dt}}
      -\int_0^1 g\parenlr{W_2,\frac{D^2W_1}{dt^2}+R(W_1,V)V}dt = 0
\end{align*}
となる.

最後に$T^\prime$上$E$が正定値であることを示そう.まずは半正定値性を示す.$W\in T^\prime$とする.$W$に対応する$\gamma$の変分$\Bar{\alpha}$を,すべての$0\leq i\leq k$で$\alpha(u.t_i)$が$u$によらず$\gamma(t_i)$となるようにとることができる.各$[t_i,t_{i+1}]$上では$\gamma$が最短線であることに注意すると
\begin{align*}
    E(\Bar{\alpha}(u)) &= \sum_{i=0}^{k-1} E_{t_i}^{t_{i+1}}(\Bar{\alpha}(u))
        \geq \sum_{i=0}^{k-1} L_{t_i}^{t_{i+1}}(\Bar{\alpha}(u)) \\
    &\geq \sum_{i=0}^{k-1} L_{t_i}^{t_{i+1}}(\gamma) = \sum_{i=0}^{k-1} E_{t_i}^{t_{i+1}}(\gamma)
        = E(\gamma)
\end{align*}
であることが分かる.これの二回微分を計算すれば
\[ E_{\ast \ast}(W,W) \geq 0 \]
を得る.

あとは$E_{\ast\ast}(W,W)=0$なら$W=0$であることを示せばよい.これには任意の$W^\prime\in T_\gamma \Omega$に対して$E_{\ast\ast}(W,W^\prime)=0$であることを示せば$W$が$\gamma$に沿ったJacobi場であることが分かり,先に示した直和分解から$W=0$となうことが分かる.再び先の直和分解と$E$の双線型性から,$W^\prime\in T_\gamma \Omega(t_0,t_1,\dots ,t_k)$のときと$W^\prime\in T^\prime$のときに場合分けして示せばよい.前者の場合はすでに示してあるので,後者の場合を考える.任意の$c\in \setR$について$W+cW^\prime \in T^\prime$であるから,すでに示した半正定値性から
\[ E_{\ast\ast}(W+cW^\prime,W+cW^\prime) \geq 0 \]
であり,これを展開すると
\[ 2cE_{\ast\ast}(W,W^\prime)+c^2E_{\ast\ast}(W^\prime,W^\prime) \geq 0 \]
が分かる.$c>0$のもとでこの式を$c$で割り,$c \to 0$とすることで
\[ E_{\ast\ast}(W,W^\prime)\geq 0 \]
を得る.$c<0$においても同様のことをすると
\[ E_{\ast\ast}(W,W^\prime)\leq 0 \]
となるから,$E_{\ast\ast}(W,W^\prime) = 0$が分かる.
\end{proof}

\end{document}