\documentclass[uplatex]{jsarticle}
%DIF LATEXDIFF DIFFERENCE FILE
%DIF DEL diff/./good_covering.tex   Sun Nov 29 15:21:23 2020
%DIF ADD ./good_covering.tex        Sun Nov 29 15:20:50 2020
\usepackage{MAstandard, appendix, cleveref}
\DeclareMathOperator{\Int}{Int}
\newcommand{\smooth}{C^{\infty}}
\newcommand{\setinverse}[2]{#1 ^{-1}(#2)}
\DeclareMathOperator{\sign}{sign}
\DeclareMathOperator{\Real}{Re}


\title{good coveringについて}
\author{みつば(@mittlear1)}
\date{2020年9月24日}
%DIF PREAMBLE EXTENSION ADDED BY LATEXDIFF
%DIF UNDERLINE PREAMBLE %DIF PREAMBLE
\RequirePackage[normalem]{ulem} %DIF PREAMBLE
\RequirePackage{color}\definecolor{RED}{rgb}{1,0,0}\definecolor{BLUE}{rgb}{0,0,1} %DIF PREAMBLE
\providecommand{\DIFadd}[1]{{\protect\color{blue}\uwave{#1}}} %DIF PREAMBLE
\providecommand{\DIFdel}[1]{{\protect\color{red}\sout{#1}}}                      %DIF PREAMBLE
%DIF SAFE PREAMBLE %DIF PREAMBLE
\providecommand{\DIFaddbegin}{} %DIF PREAMBLE
\providecommand{\DIFaddend}{} %DIF PREAMBLE
\providecommand{\DIFdelbegin}{} %DIF PREAMBLE
\providecommand{\DIFdelend}{} %DIF PREAMBLE
\providecommand{\DIFmodbegin}{} %DIF PREAMBLE
\providecommand{\DIFmodend}{} %DIF PREAMBLE
%DIF FLOATSAFE PREAMBLE %DIF PREAMBLE
\providecommand{\DIFaddFL}[1]{\DIFadd{#1}} %DIF PREAMBLE
\providecommand{\DIFdelFL}[1]{\DIFdel{#1}} %DIF PREAMBLE
\providecommand{\DIFaddbeginFL}{} %DIF PREAMBLE
\providecommand{\DIFaddendFL}{} %DIF PREAMBLE
\providecommand{\DIFdelbeginFL}{} %DIF PREAMBLE
\providecommand{\DIFdelendFL}{} %DIF PREAMBLE
%DIF LISTINGS PREAMBLE %DIF PREAMBLE
\RequirePackage{listings} %DIF PREAMBLE
\RequirePackage{color} %DIF PREAMBLE
\lstdefinelanguage{DIFcode}{ %DIF PREAMBLE
%DIF DIFCODE_UNDERLINE %DIF PREAMBLE
  moredelim=[il][\color{red}\sout]{\%DIF\ <\ }, %DIF PREAMBLE
  moredelim=[il][\color{blue}\uwave]{\%DIF\ >\ } %DIF PREAMBLE
} %DIF PREAMBLE
\lstdefinestyle{DIFverbatimstyle}{ %DIF PREAMBLE
	language=DIFcode, %DIF PREAMBLE
	basicstyle=\ttfamily, %DIF PREAMBLE
	columns=fullflexible, %DIF PREAMBLE
	keepspaces=true %DIF PREAMBLE
} %DIF PREAMBLE
\lstnewenvironment{DIFverbatim}{\lstset{style=DIFverbatimstyle}}{} %DIF PREAMBLE
\lstnewenvironment{DIFverbatim*}{\lstset{style=DIFverbatimstyle,showspaces=true}}{} %DIF PREAMBLE
%DIF END PREAMBLE EXTENSION ADDED BY LATEXDIFF

\begin{document}
\maketitle
\begin{abstract}
    何かと必要になる(気がする)多様体のgood coveringの存在を示す.その周辺の話題なども多少書く.
\end{abstract}

\section{good coveringの存在}
以下,多様体は第二可算で$C^\infty$\DIFaddbegin \DIFadd{級}\DIFaddend なものを考える.


\begin{definition}
  $M$を多様体とする.$M$の開被覆$\mscrU$がgood coveringであるとは,
  任意の有限個の$\mscrU$の元$U_1,U_2,\dots,U_k$に対して,その共通部分が空でなければ
  $\setR^n$と微分同相であることをいう.
\end{definition}

本稿の目標は次の定理である.

\begin{theorem}\label{main}
  任意の多様体$M$についてgood coveringは存在する.
\end{theorem}

以下,$\dim M = n$とする.また,$M$は連結であるとしてよい.
証明のため,$M$上のRiemann計量$g$を一つとって固定する.

\begin{definition}
  $M$の開集合$U$が測地的に凸であるとは,任意の$p, q \in M$に対して$p$と$q$を結ぶ最短測地線がただ一つ
  存在し,それが$U$に含まれることをいう.
\end{definition}

\begin{proposition}\label{existence}
  測地的に凸な開集合全体の集合は$M$の位相の開基をなす.
\end{proposition}

\begin{proof}
  $p\in M$を任意にとる.
  $0\in T_pM$の近傍$V$を小さくとって,$\map{\exp=\exp_p}{V}{M}$が像への微分同相になり,かつ
  任意の$V$の2点についてそれらを結ぶ最短測地線が$M$内でただ一つ存在するようにする.
  さらに$\setR^n$と$T_p M$との間の線型同型写像を一つとることで$p$まわりのチャート$(\exp(V), \varphi)$
  ができる.\DIFaddbegin \DIFadd{以下では,}\DIFaddend $\exp(V)$に含まれ,$p$を含む測地的に凸な開集合を構成する.
  $x^i(p)\Gamma_{ij}^k(p) = 0$であり,$i, j, k$の動く範囲は有限だから,$V$に含まれる
  0の近傍$V^\prime$が存在して,任意の$i, j, k$に対して$\exp(V^\prime)$上
    \[\abs{x^k\Gamma_{ij}^k} < 1 / n^2 \]
  となる.
  $\gamma$を$\exp(V^\prime)$内の測地線とする.$(\exp(V),\varphi)$を使って$\gamma$を
  局所表示したときの第$i$成分を$\gamma^i$と書くことにすると,
  相加・相乗平均の不等式から
  \begin{align*}
    \sum_{i, j} \frac{d\gamma^i}{dt} \frac{d\gamma^j}{dt} 
      &\leq \sum_{i, j} \frac{1}{2} \bracelr{\parenlr{\frac{d\gamma^i}{dt}}^2 + 
      \parenlr{\frac{d\gamma^j}{dt}}^2} = n\sum_k \parenlr{\frac{d\gamma^k}{dt}}^2
  \end{align*}
  となる.$\map{r}{\exp(V^\prime)}{\setR}$を
    \[r(q) = (d(p,q))^2 \]
  で定める.ただし,$d$はRiemann計量から定まる$M$上の距離を表す.このとき
  \[(r\circ \gamma)(t) = \sum_k (\gamma^k(t))^2\]
  なので,
  \begin{align*}
    \frac{1}{2}\frac{d^2}{dt^2}(r\circ \gamma)(t)
      &= \sum_k \parenlr{\frac{d\gamma^k}{dt}}^2 + \sum_k \gamma^k \frac{d^2\gamma^k}{dt^2} 
      = \parenlr{\sum_k\frac{d\gamma^k}{dt}}^2 - \sum_{i, j, k} \gamma^k \Gamma_{ij}^k\frac{d\gamma^i}{dt}\frac{d\gamma^j}{dt} \\
    &> \sum_k \parenlr{\frac{d\gamma^k}{dt}}^2 
      - \sum_{i,j,k} \frac{1}{n^2}\frac{d\gamma^i}{dt}\frac{d\gamma^j}{dt} \\
    &\geq \sum_k \parenlr{\frac{d\gamma^k}{dt}}^2 - n^2\sum_k\frac{1}{n^2}\parenlr{\frac{d\gamma^k}{dt}}^2  = 0
  \end{align*}
  となる.したがって$r\circ \gamma$は極大値を持たないことが分かった.

  $a>0$に対して
    \[B_a = \set{v\in T_p M}{\norm{v} < a} \]
  と書くことにする.$\varepsilon>0$を$B_{3\varepsilon} \subset V^\prime$となるようにとり,
  $U = \exp(B_\varepsilon)$とおく.$U$上の2点$q, q^\prime$を任意にとり,これらを結ぶ最短測地線を
  $\eta$とするとき,任意の$t$で$\eta(t)\in \exp(B_{3\varepsilon})\subset \exp(V^\prime)$である.
  したがって,上で調べたことから
  $\eta(t)$と$p$との間の距離は$q$または$q^\prime$で最大となる.以上より$\eta(t)$は常に
  $U$に含まれることが分かったので,$U$は測地的に凸である.

  初めの$V$はいくらでも小さくとれるので,命題の主張が正しいことが分かる.
\end{proof}

\begin{proposition}
  \begin{enumarabicp}
    \item 測地的に凸な開集合の有限個の共通部分は空でなければ測地的に凸である.
    \item 任意の$p\in M$に対してある開近傍$W$を十分小さくとることで,
    $W$に含まれる測地的に凸な開集合$U$は$\setR^n$の星型領域と微分同相であるようにできる.
  \end{enumarabicp}
\end{proposition}

\begin{proof}
  (1)は明らかである.(2)を示す.$0\in T_p M$の$TM$での近傍$\mscrV$と$p$の開近傍$W$を,
  $\map{\exp}{\mscrV}{W\times W}$が微分同相写像になるようにとる.この$W$が条件をみたすことを示す.
  $W$に含まれる空でない測地的に凸な開集合$U$を任意にとる.
  $q\in U$を一つ選んでおく.$\map{\exp_q}{T_q M}{M}$を考える.$0\in T_q M$の開近傍$V$を
  \[V = \set{v\in T_q M}{\exp_q v \in U\text{かつ}\exp_q tv\text{は$q$と$\exp_q v$を結ぶ最短測地線}} \]
  で定める.$\restr{\exp_q}{V}$が$V$から$U$への微分同相写像であることを示す.
  まず,$q$と$U$の点を結ぶ最短測地線はただ一つしかないので単射であり,全射性も明らかである.
  また,$W$のとり方から$\restr{\exp_q}{V}$は各点で微分が正則である.したがってこれは微分同相である.
  $V$は星型領域なので命題の主張が分かる.
\end{proof}

あとは星型領域が$\setR^n$と微分同相であることを示せばよい.

\begin{lemma}\label{function}
$U$を$\setR^n$の開集合とすると,$C^\infty$級関数$\map{h}{\setR^n}{\setR}$が存在して次をみたす.
\begin{enumarabicp}
  \item $h\geq 0,$
  \item $U=h^{-1}((0,\infty)),$
  \item $\setR^n\setminus U = h^{-1}(0).$
\end{enumarabicp}
\end{lemma}

\begin{proof}
  $U$がLidel\"{o}fなので,開球の族$\{B_{x_n}(r_n)\}_{n\in \setN}$が存在して
    \[U = \bigcup_n B_{x_n}(r_n)\]
  が成立する.$C^\infty$級関数$\map{\varphi_n}{\setR^n}{\setR}$を
  \begin{enumarabicp}
    \item $0\leq \varphi_n \leq 1,$
    \item $\varphi^{-1}((0,1])=B_{x_n}(r_n),$
    \item $\varphi^{-1}(0)=\setR^n\setminus B_{x_n}(r_n)$
  \end{enumarabicp}
  となるようにとる.各$\varphi_n$はコンパクト台をもつ関数なので,$C_n>0$を,$\varphi_n$の
  $n$階までのすべての導関数の上界となるようにとれる.そこで$h$を
    \[h(x)=\sum_{n=0}^\infty \frac{1}{2^n} C_n \varphi_n(x)\]
  とおくと,これは確かにすべての$x\in \setR^n$で収束し,$\setR^n$上の関数を定める.
  また,項別微分定理を繰り返し適用することにより,$h$が$C^\infty$級関数であることも分かる.
  これが補題の条件をみたすことは明らかである.
\end{proof}

\begin{proposition}
  $U$を$\setR^n$の中の星型領域とする.このとき$U$は$\setR^n$と微分同相である.
\end{proposition}

\begin{proof}
  $0\in U$であり,しかも0と$U$の任意の点が線分で結べるとしてよい.\cref{function}の$h$をとり,
  $\map{f}{U}{\setR^n}$を
    \[f(x)=\parenlr{1+\parenlr{\int_0^1 \frac{1}{h(vx)}dv}^2\norm{x}^2}x
    =\parenlr{1+\parenlr{\int_0^{\norm{x}}\frac{1}{h(wx/\norm{x})}dw }^2 }x \]
  で定める.1つ目の表記より$f$は$C^\infty$級である.次に$f$が全単射であることを示す.
  そのためには,$x\in \setR^n$で$\norm{x}=1$であるものを任意にとって
    \[A(x)=\sup\set{t\in \setR_{\geq0}}{tx\in U}\]
  とおいたとき,$f$が$[0, A(x))x$と$[0,\infty)x$の間の全単射になることを示せばよい.
  $f$の2つ目の表記よりこれは確かに単射である.次に全射性を示す.これには
    \[\lim_{t\to A(x)} \norm{f(tx)} = \infty \]
  であることを示せばよい.まず$A(x)=\infty$のとき,
    \[\norm{f(tx)} \geq \norm{tx} \xrightarrow{t\to \infty} \infty \]
  である.また$A(x)< \infty$のとき,平均値の定理より
    \[h(A(x)x)-h(tx)=\restrlr{\frac{dh(tx)}{dt}}{t=t_0} (A(x)-t) \]
  となる$t_0\in (t, A(x))$が存在する.$[0,A(x)]$のコンパクト性から微分の絶対値は
  ある定数$C$で上からおさえられる.$h(A(x)x)=0$であることに注意すると
  \begin{align*}
    \int_0^1 \frac{1}{h(vtx)}dv&\geq \int_0^1 \frac{1}{C(A(x)-tv)}dv
    =\frac{1}{ct}\log\frac{A(x)}{A(x)-t}\xrightarrow{t\to A(x)} \infty
  \end{align*}
  なので
    \[\lim\DIFdelbegin \DIFdel{_{t\to A(X)} }\DIFdelend \DIFaddbegin \DIFadd{_{t\to A(x)} }\DIFaddend \norm{f(tx)}=\infty \]
  となる.以上より$f$が全単射であることが分かった.

  最後に$f$の微分がどの点でも正則であることを示す.もしそうでないとすると,$x\in U$と
  $v\in \setR^n\setminus\{0\}$で
    \[df_x(v) = 0\]
  となるものが存在することになる.
  $\lambda(x) = \norm{f(x)}/\norm{x}$とおくと,これは
    \[d\lambda_x(v)x+\lambda(x)v = 0 \]
  と書ける.よってある$\mu \in \setR\setminus\{0\}$を用いて$v = \mu x$と書ける.
  $v\neq 0$より$x\neq 0$である.したがって
  \begin{align*}
    d\lambda_x(\mu x)x+\lambda(x)\mu x=0 \iff d\lambda_x(x)+\lambda(x)=0
  \end{align*}
  となる.しかし,$\lambda$の定義から$\lambda(x)>0$で,
    \[d\lambda_x(x)=\restrlr{\frac{d}{dt}}{t=1}\lambda(tx)\geq 0\]
  なので矛盾する.
\end{proof}

十分小さい測地的に凸な開集合で作った開被覆は必ずgood coveringになるので,次の系が分かる.

\begin{corollary}
  任意の$M$の開被覆に対して,それの細分になっているようなgood coveringが存在する.
\end{corollary}

\section{bootsrap lemma}

good coveringの存在定理は,次のbootstap lemmaと組み合わせることで威力を発揮する.

\begin{theorem}[bootsrap lemma]
  $M$を位相多様体とする.$\mfrakB$を$M$の開基であって
  \begin{enumromanp}
    \item $\emptyset\in \mfrakB$,
    \item $U, V\in \mfrakB$ならば$U\cap V \in \mfrakB$
  \end{enumromanp}
  をみたすものとする.$M$の開集合からなる集合$\mfrakA$が次の性質をみたすとする.
  \begin{enumarabicp}
    \item $\mfrakB\subset \mfrakA$.
    \item $\mfrakA$の任意の元$U, V$について$U\cap V\in \mfrakA$が成立するなら$U\cup V\in \mfrakA$.
    \item 互いに交わらない可算個の$\mfrakA$の元$U_1, U_2, \dots$について
    $\cup_{n=1}^\infty U_n\in \mfrakA$.
  \end{enumarabicp}
  このとき,$M\in \mfrakA$である.
\end{theorem}

\begin{proof}
  まず,$\mfrakB$の$k$個の元$V_1, \dots, V_k$に対して$\bigcup_{i=1}^k V_i\in \mfrakA$であることを$k$
  についての帰納法で示す.$k=1$のときは明らかである.$k-1$まで正しいことが示されているとき,
  帰納法の仮定と(ii)から
  $V_1\cup\dots\cup V_{k-1},\ V_k$および$(V_1\cap V_k)\cup\dots \cup (V_{k-1}\cap V_k)$
  は$\mfrakA$の元である.したがって$(2)$より$\bigcup_{i=1}^k V_i\in\mfrakA$である.

  $M$は局所コンパクトHausdorff空間だから,コンパクト集合の増大列$\{K_n\}_{n\in \setN}$
  であって
  \begin{enumerate}
    \item すべての$n\in\setN$で$K_n\subset \Int K_{n+1}$,
    \item $\bigcup_{n\in\setN} K_n = M$
  \end{enumerate}
  となるものがとれる.$U_{n}=\Int K_{n+1}\setminus K_{n-2},\ L_n=K_n\setminus \Int K_{n-1}$とおく.
  ただし,$K_{-2}=K_{-1}=\emptyset$と約束する.$\mfrakB$は開基であることと$L_n$のコンパクト性から,
  $\set{V\in\mfrakB}{V\subset U_{n+1}}$の有限個の元$V_{n,1},\dots, V_{n,k_n}$で$L_n$を被覆できる.
  このとき$W_n=\bigcup_{i=1}^{k_n} V_i$は$\mfrakA$の元であり,$W_n\subset U_{n+1}$をみたす.
  そこで
    \[W^\prime_0=\bigcup_{i=0}^\infty W_{3i},\  W^\prime_1=\bigcup_{i=0}^\infty W_{3i+1},\ 
      W^\prime_2=\bigcup_{i=0}^\infty W_{3i+2}\]
  とおくと,(3)より$W^\prime_0,\ W^\prime_1,\ W^\prime_2\in\mfrakA$である.したがって
    \[M=W^\prime_0\cup W^\prime_1\cup W^\prime_2\in\mfrakA\]
  となる.
\end{proof}

good coverの存在とbootstrap lemmaをどのように使うかを簡単に説明する.
一般の多様体$M$について何か示したい命題$P$が与えられているとする(Poincar\'{e}双対などの
(コ)ホモロジーについての定理を念頭において考えるとよい).
まず,bootstrap lemmaにおける$\mfrakA$として$P$を真にする開集合全体をとり,
$\mfrakB$としてgood coverをとる.この時点でbootstrap lemmaの$\mfrakB$についての仮定は
($\mfrakB$に空集合を含めることにすれば)みたされているので,あとは後半の$\mfrakA$についての
性質を確かめればよい.

(1)は結局$\setR^n$の場合に$P$を示すということなので一番簡単な
場合であるはずである.(2)と(3)についてはもはやgood coverは関係のない,命題$P$のもつ形式的な
性質の証明である.(2)はその形から分かる通り,Mayer-Vietoris完全列を意識した条件である.
(3)も(コ)ホモロジーが連結成分ごとに分解することを反映した条件である.

多くの場合,一番骨が折れるところは(2)のMayer-Vietoris完全列に関する図式の可換性などのチェック
であるが,これも単なるdiagram chasingで終わることもしばしばである.

\section{応用}
2節の最後で述べた手法で証明できる命題をいくつか列挙する.

\begin{theorem}[de RhamコホモロジーのPoincar\'{e}双対]
  向きづけられた$n$次元多様体$M$と非負整数$k$について,
    \[([\omega],\DIFaddbegin [\DIFaddend \eta\DIFaddbegin ]\DIFaddend )=\int_M \omega\wedge \eta\]
  で定まるペアリング$\map{(\ ,\ )}{H^{n-k}(M)\times H^k_c(M)}{\setR}$は同型である.
  ただし,$H^\ast_c(M)$はコンパクト台のde Rhamコホモロジーを表す.
\end{theorem}

\begin{theorem}[de RhamコホモロジーのK\"{u}nneth公式]
  $M,\ N$を多様体とし,さらに$N$のde Rhamコホモロジー$H^\ast(N)$が有限生成であるとする.
  このとき
    \[H^\ast(M\times N)=H^\ast(M)\otimes H^\ast(N)\]
  \DIFdelbegin \DIFdel{である}\DIFdelend \DIFaddbegin \DIFadd{が成立する}\DIFaddend .
\end{theorem}

これらの定理の証明は,有限個の元からなるgood coveringをもつ場合には\DIFdelbegin \DIFdel{\mbox{%DIFAUXCMD
\cite{Bott-Tu}}\hspace{0pt}%DIFAUXCMD
}\DIFdelend \DIFaddbegin \DIFadd{\mbox{%DIFAUXCMD
\cite{Bott--Tu}}\hspace{0pt}%DIFAUXCMD
}\DIFaddend で証明されている.
実際にはその仮定は上で書いたように弱められるのであるが,2節の最後に書いた手法を用いれば
\DIFdelbegin \DIFdel{\mbox{%DIFAUXCMD
\cite{Bott-Tu}}\hspace{0pt}%DIFAUXCMD
}\DIFdelend \DIFaddbegin \DIFadd{\mbox{%DIFAUXCMD
\cite{Bott--Tu}}\hspace{0pt}%DIFAUXCMD
}\DIFaddend の証明で実質的な部分の証明はほぼ終了している.

\begin{theorem}
  $H_{dR}^\ast$をde Rhamコホモロジーを与える関手,$H_{sing}^\ast$を特異コホモロジーを与える関手
  とする.このとき,$H_{dR}^\ast$と$H_{sing}^\ast$は自然同値である.
\end{theorem}

この定理は\DIFdelbegin \DIFdel{\mbox{%DIFAUXCMD
\cite{Lee}}\hspace{0pt}%DIFAUXCMD
}\DIFdelend \DIFaddbegin \DIFadd{\mbox{%DIFAUXCMD
\cite{LeeSmooth}}\hspace{0pt}%DIFAUXCMD
}\DIFaddend に証明がある.

\DIFdelbegin %DIFDELCMD < \begin{thebibliography}{10}
%DIFDELCMD <   \bibitem{Bott-Tu} %%%
\DIFdel{Bott, R. and Tu, L., 
    }\textit{\DIFdel{Differential Forms in Algebraic Topology}}%DIFAUXCMD
\DIFdel{, Springer, 1982.
  }%DIFDELCMD < \bibitem{Lee} %%%
\DIFdel{Lee, J. M., 
    }\textit{\DIFdel{Introduction to Topological Manifolds}}%DIFAUXCMD
\DIFdel{, Springer, 2000.
  }\DIFdelend \DIFaddbegin \bibliographystyle{MA}
\DIFaddend 

\DIFdelbegin %DIFDELCMD < \end{thebibliography}
%DIFDELCMD < %%%
\DIFdelend \DIFaddbegin \bibliography{data}
\DIFaddend 

\end{document}
good cover全体が開被覆全体のなす有向集合上共終であること

2020/9/21 1節の執筆
2020/9/22 1節の執筆およびbootstrap lemmaの主張
2020/9/24 bootstrap lemmaの証明,3節