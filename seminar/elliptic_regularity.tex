
%持つ,分かる,
%対して,すべて

\documentclass[uplatex,dvipdfmx]{jsarticle}
\usepackage{MAstandard,appendix}

\DeclareMathOperator{\Int}{int}
\DeclareMathOperator{\sign}{sign}
\newcommand{\smooth}{$C^{\infty}$級}
\newcommand{\sobolev}[1]{L^{2}_{#1}}
\newcommand{\sobolevlocal}[3]{\sobolev{#1}(\setR^{#2},\setC^{#3})}
\newcommand{\smoothfct}{C^{\infty}}
\newcommand{\cptsmoothfct}[2]{C^{\infty}_{c}(\setR^{#1},\setC^{#2})}
\newcommand{\schwartz}[2]{\mcalS(\setR^{#1},\setC^{#2})}
\newcommand{\dd}{d} %本当はバー付きにしたい

\crefname{theorem}{定理}{定理}
\crefname{proposition}{命題}{命題}
\crefname{lemma}{補題}{補題}
\crefname{definition}{定義}{定義}
\crefname{example}{例}{例}
\crefname{remark}{注}{注}

\title{楕円型正則性}
\author{みつば(@mittlear1)}
\date{\today}

\begin{document}

\maketitle
\tableofcontents

\newpage
\section*{記法}
$\setN$:0以上の整数全体.

$\smoothfct(M,N)$:$M$から$N$への\smooth 写像全体.

$C^{\infty}_c(M)$:コンパクト台を持つ$M$上の\smooth 関数全体.

$\mcalS(\setR^n,C^m)$:$\setR^n$上の$\setC^m$値のSchwartzの急減少関数全体.



\newpage
\section{微分作用素}

\subsection{微分作用素の定義}

\subsection{主表象}

\subsection{形式的随伴作用素}

\newpage
\section{Sobolev空間}

\subsection{Euclid空間上のSobolev空間}

$(\cdot,\cdot)$を$\setC^m$上の標準的なHermite計量とする.

\begin{definition}
  $f, g\in \schwartz{n}{m}$および$s\in \setR$に対して
  \[ \abracket{f,g}_s=\int_\setR^n (\hat{f}(\xi),\hat{g}(\xi))(1+\abs{\xi})^{2s}\dd \xi \]
\end{definition}
で内積$\abracket{\cdot,\cdot}_s$を定め,ここからできるノルム$\norm{\cdot}$による
$\schwartz{n}{m}$の完備化を$\sobolevlocal{s}{n}{m}$と書き,$s$次の\textbf{Sobolev空間}という.

\begin{lemma}
  $f$に対して$\hat{f}(\xi)(1+\abs{\xi})^s$を対応させることでできる$\schwartz{n}{m}$から
  $L^2(\setR^n,\setC^m)$への線型写像は$\sobolevlocal{s}{n}{m}$から$L^2(\setR^n,\setC^m)$
  へのHilbert空間としての同型写像に一意的に延長される.
\end{lemma}

\begin{proposition}
  $\cptsmoothfct{n}{m}$は$\sobolevlocal{s}{n}{m}$の中で稠密である.
\end{proposition}

\begin{proposition}\label{Sobolev space injective}
  $s>t\in \setR$とする.$\schwartz{n}{m}$からそれ自身への恒等写像は有界線型写像
  \[ \map{\iota_{st}}{\sobolevlocal{s}{n}{m}}{\sobolevlocal{t}{n}{m}} \]
  を誘導し,$\iota_{st}$は単射である.
\end{proposition}

\begin{definition}
  位相線形空間$\sobolevlocal{\infty}{n}{m}$を
  \[ \sobolevlocal{\infty}{n}{m} = \varprojlim_{s\in\setR} \sobolevlocal{s}{n}{m} \]
  で定める.\cref{Sobolev space injective}より,自然な写像
  $\map{\iota_{s}}{\sobolevlocal{\infty}{n}{m}}{\sobolevlocal{s}{n}{m}}$は単射である.
\end{definition}

\begin{definition}
  $s\in \setN$のとき,$f\in\schwartz{n}{m}$に対して$\norm{f}_{W^{s,2}}$を
  \[ \norm{f}^2_{W^{s,2}} = \sum_{\abs{\alpha}\leq s} \norm{\partial^\alpha f}_{L^2}^2 \]
  で定める.これは$\schwartz{n}{m}$上のノルムである.
\end{definition}

\begin{proposition}
  $s\in \setN$のとき,$\schwartz{n}{m}$上のノルムとして$\norm{\cdot}_s$と$\norm{\cdot}_{W^{s,2}}$
  は同値である.
\end{proposition}

\begin{proposition}
  $\phi\in \schwartz{n}{}$に対して$\map{M_\phi}{\schwartz{n}{m}}{\schwartz{n}{m}}$
  を
  \[ M_\phi f = \phi f \]
  で定めると,これは$\sobolevlocal{s}{n}{m}$からそれ自身への有界線型写像に延長される.
\end{proposition}

\begin{proposition}
  $\map{(\cdot,\cdot)_{L^2}}{\schwartz{n}{m}\times\schwartz{n}{m}}{\setC}$を
  \[(f,g)_{L^2} = \int_{\setR^n} (f(x),g(x)) \dd x \]
  で定めると,$(\cdot,\cdot)_{L^2}$は任意の$s\in \setR$で連続なsesqui-linear form
  \[ \map{(\cdot,\cdot)^s_{L^2}}{\sobolevlocal{s}{n}{m}\times\sobolevlocal{s}{n}{m}}{\setC} \]
  を定める.また,任意の$f\in \sobolevlocal{s}{n}{m}$について
  \[ \norm{f}_s = \sup_{g\in \schwartz{n}{m}\setminus 0} \frac{\abs{(f,g)_{L^2}^s}}{\norm{g}_{-s}} \]
  が成立する.
\end{proposition}

\begin{definition}
  \begin{enumarabicp}
    \item $k\in \setN$に対して$C^k_0(\setR^n,\setC^m)$を,$C^k$級関数$\map{f}{\setR^n}{\setC^m}$であって
    \[ \abs{\alpha}\leq k \Rightarrow \lim_{\abs{x}\to \infty} \abs{\partial^\alpha f} = 0 \]
    をみたすもの全体のなすベクトル空間とする.また,$C^k_0(\setR^n,\setC^m)$上のノルム
    $\norm{\cdot}_{C^k_0}$を
    \[ \norm{f}_{C^k_0} = \sum_{\abs{\alpha}\leq k} \sup_{x\in \setR^n} \abs{\partial^\alpha f} \]
    で定める.
    \item $C^\infty_0(\setR^n,\setC^m)$を
    \[C^\infty_0(\setR^n,\setC^m) = \bigcap_{k\in \setN} C^k_0(\setR^n,\setC^m) \]
    で定め,すべての$k\in \setN$で$C^\infty_0(\setR^n,\setC^m)\to C^k_0(\setR^n,\setC^m)$が連続となる
    最弱の位相を入れる.
  \end{enumarabicp}
\end{definition}

\begin{proposition}
  $C^k_0(\setR^n,\setC^m)$は$\norm{\cdot}_{C^k_0}$によってBanach空間になる.
\end{proposition}

\begin{theorem}[Sobolevの埋め込み定理]\label{Sobolev embedding}
  $k\in \setN,\ s>k+n/2$のとき,自然な包含$\schwartz{n}{m}\to C^k_0(\setR^n,\setC^m)$は
  単射有界線型写像
  \[ \map{\eta_{sk}}{\sobolevlocal{s}{n}{m}}{C^k_0(\setR^n,\setC^m)}\]
  へと一意的に延長される.
\end{theorem}

\begin{corollary}
  $\sobolevlocal{\infty}{n}{m}$から$C^\infty_0(\setR^n,\setC^m)$への単射連続線型写像$\eta$であって,
  $\eta_{sk}$と整合的なものがただ一つ存在する.
\end{corollary}

\subsection{コンパクト多様体上のSobolev空間}

\newpage
\section{擬微分作用素}

\subsection{擬微分作用素の定義}

\subsection{表象の漸近展開}

\subsection{楕円性}

\newpage
\section{楕円型正則性}

\subsection{局所正則性}

\subsection{大域正則性}

\newpage
\section{Hodge分解}

\bibliographystyle{MA}
\nocite{*}
\bibliography{data}

\end{document}

te