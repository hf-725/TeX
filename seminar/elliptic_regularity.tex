
%持つ,分かる,みなす
%対して,すべて
%本節,

\documentclass[uplatex,dvipdfmx]{jsarticle}
\usepackage{MAstandard,appendix}

\DeclareMathOperator{\Int}{int}
\DeclareMathOperator{\sign}{sign}
\newcommand{\smooth}{$C^{\infty}$級}
\newcommand{\sobolev}[1]{L^{2}_{#1}}
\newcommand{\sobolevlocal}[3]{\sobolev{#1}(\setR^{#2},\setC^{#3})}
\newcommand{\smoothfct}[2]{C^{\infty}(\setR^{#1},\setC^{#2})}
\newcommand{\cptsmoothfct}[2]{C^{\infty}_{c}(\setR^{#1},\setC^{#2})}
\newcommand{\schwartz}[2]{\mcalS(\setR^{#1},\setC^{#2})}
\newcommand{\dd}{d} %本当はバー付きにしたい

\crefname{theorem}{定理}{定理}
\crefname{proposition}{命題}{命題}
\crefname{lemma}{補題}{補題}
\crefname{definition}{定義}{定義}
\crefname{example}{例}{例}
\crefname{remark}{注}{注}

\title{楕円型正則性}
\author{みつば(@mittlear1)}
\date{\today}

\begin{document}

\maketitle
\tableofcontents

\newpage
\section*{記法}
$\setN$:0以上の整数全体.

$C^\infty(M,N)$:$M$から$N$への\smooth 写像全体.

$C^{\infty}_c(M)$:コンパクト台を持つ$M$上の\smooth 関数全体.

$\mcalS(\setR^n,C^m)$:$\setR^n$上の$\setC^m$値のSchwartzの急減少関数全体.



\newpage
\section{微分作用素}

\subsection{微分作用素の定義}

\subsection{主表象}

\subsection{形式的随伴作用素}

\newpage
\section{Sobolev空間}

\subsection{Euclid空間上のSobolev空間}

$(\cdot,\cdot)$を$\setC^m$上の標準的なHermite計量とする.

\begin{definition}
  $f, g\in \schwartz{n}{m}$および$s\in \setR$に対して
  \[ \abracket{f,g}_s=\int_{\setR^n} (\hat{f}(\xi),\hat{g}(\xi))(1+\abs{\xi})^{2s}\dd \xi \]
で内積$\abracket{\cdot,\cdot}_s$を定め,ここからできるノルム$\norm{\cdot}_s$による
$\schwartz{n}{m}$の完備化を$\sobolevlocal{s}{n}{m}$と書き,$s$次の\textbf{Sobolev空間}という.
\end{definition}

各$s\in \setR$において$\schwartz{n}{m}$から$\sobolevlocal{s}{n}{m}$への自然な単射がある.
これによって$\schwartz{n}{m}$を$\sobolevlocal{s}{n}{m}$の部分集合とみなす.

\begin{lemma}
  $f$に対して$\hat{f}(\xi)(1+\abs{\xi})^s$を対応させることでできる$\schwartz{n}{m}$から
  $L^2(\setR^n,\setC^m)$への線型写像は$\sobolevlocal{s}{n}{m}$から$L^2(\setR^n,\setC^m)$
  へのHilbert空間としての同型写像に一意的に延長される.
\end{lemma}

\begin{proposition}
  $\cptsmoothfct{n}{m}$は$\sobolevlocal{s}{n}{m}$の中で稠密である.
\end{proposition}

\begin{proposition}\label{Sobolev space injective}
  $s>t\in \setR$とする.$\schwartz{n}{m}$からそれ自身への恒等写像は有界線型写像
  \[ \map{\iota_{st}}{\sobolevlocal{s}{n}{m}}{\sobolevlocal{t}{n}{m}} \]
  を誘導し,$\iota_{st}$は単射である.
\end{proposition}

\begin{definition}
  位相線形空間$\sobolevlocal{\infty}{n}{m}$を
  \[ \sobolevlocal{\infty}{n}{m} = \varprojlim_{s\in\setR} \sobolevlocal{s}{n}{m} \]
  で定める.\cref{Sobolev space injective}より,自然な写像
  $\map{\iota_{s}}{\sobolevlocal{\infty}{n}{m}}{\sobolevlocal{s}{n}{m}}$は単射である.
\end{definition}

\begin{definition}
  $s\in \setN$のとき,$f\in\schwartz{n}{m}$に対して$\norm{f}_{W^{s,2}}$を
  \[ \norm{f}^2_{W^{s,2}} = \sum_{\abs{\alpha}\leq s} \norm{\partial^\alpha f}_{L^2}^2 \]
  で定める.これは$\schwartz{n}{m}$上のノルムである.
\end{definition}

\begin{proposition}
  $s\in \setN$のとき,$\schwartz{n}{m}$上のノルムとして$\norm{\cdot}_s$と$\norm{\cdot}_{W^{s,2}}$
  は同値である.
\end{proposition}

\begin{proposition}
  $\phi\in \schwartz{n}{}$に対して$\map{M_\phi}{\schwartz{n}{m}}{\schwartz{n}{m}}$
  を
  \[ M_\phi f = \phi f \]
  で定めると,これは$\sobolevlocal{s}{n}{m}$からそれ自身への有界線型写像に延長される.
\end{proposition}

\begin{proposition}
  $\map{(\cdot,\cdot)_{L^2}}{\schwartz{n}{m}\times\schwartz{n}{m}}{\setC}$を
  \[(f,g)_{L^2} = \int_{\setR^n} (f(x),g(x)) \dd x \]
  で定めると,$(\cdot,\cdot)_{L^2}$は任意の$s\in \setR$で連続なsesqui-linear form
  \[ \map{(\cdot,\cdot)^s_{L^2}}{\sobolevlocal{s}{n}{m}\times\sobolevlocal{s}{n}{m}}{\setC} \]
  を定める.また,任意の$f\in \sobolevlocal{s}{n}{m}$について
  \[ \norm{f}_s = \sup_{g\in \schwartz{n}{m}\setminus 0} \frac{\abs{(f,g)_{L^2}^s}}{\norm{g}_{-s}} \]
  が成立する.
\end{proposition}

\begin{definition}
  \begin{enumarabicp}
    \item $k\in \setN$に対して$C^k_0(\setR^n,\setC^m)$を,$C^k$級関数$\map{f}{\setR^n}{\setC^m}$であって
    \[ \abs{\alpha}\leq k \Rightarrow \lim_{\abs{x}\to \infty} \abs{\partial^\alpha f} = 0 \]
    をみたすもの全体のなすベクトル空間とする.また,$C^k_0(\setR^n,\setC^m)$上のノルム
    $\norm{\cdot}_{C^k_0}$を
    \[ \norm{f}_{C^k_0} = \sum_{\abs{\alpha}\leq k} \sup_{x\in \setR^n} \abs{\partial^\alpha f} \]
    で定める.
    \item $C^\infty_0(\setR^n,\setC^m)$を
    \[C^\infty_0(\setR^n,\setC^m) = \bigcap_{k\in \setN} C^k_0(\setR^n,\setC^m) \]
    で定め,すべての$k\in \setN$で$C^\infty_0(\setR^n,\setC^m)\to C^k_0(\setR^n,\setC^m)$が連続となる
    最弱の位相を入れる.
  \end{enumarabicp}
\end{definition}

\begin{proposition}
  $C^k_0(\setR^n,\setC^m)$は$\norm{\cdot}_{C^k_0}$によってBanach空間になる.
\end{proposition}

\begin{theorem}[Sobolevの埋め込み定理]\label{Sobolev embedding}
  $k\in \setN,\ s>k+n/2$のとき,自然な包含$\schwartz{n}{m}\to C^k_0(\setR^n,\setC^m)$は
  単射有界線型写像
  \[ \map{\eta_{sk}}{\sobolevlocal{s}{n}{m}}{C^k_0(\setR^n,\setC^m)}\]
  へと一意的に延長される.
\end{theorem}

\begin{corollary}
  $\sobolevlocal{\infty}{n}{m}$から$C^\infty_0(\setR^n,\setC^m)$への単射連続線型写像$\eta$であって,
  $\eta_{sk}\ (s>k+n/2)$たちと整合的なものがただ一つ存在する.
\end{corollary}

\begin{definition}
  $\setR^n$の開集合$U$に対して$\sobolev{s}(U,\setC^m)$を$C^\infty_c(U,\setC^m)$の
  $\sobolevlocal{s}{n}{m}$での閉包で定義する.
\end{definition}

\begin{theorem}[Rellichの補題]\label{Rellich lemma}
  $U$を$\setR^n$の相対コンパクト開集合とする.$s>t$のとき,包含
  $\map{\iota_{st}}{\sobolev{s}(U,\setC^m)}{\sobolev{t}(U,\setC^m)}$はコンパクト作用素である.
\end{theorem}

\subsection{コンパクト多様体上のSobolev空間}

本節では,$M$を$n$次元コンパクト多様体,$E$を$M$上の階数$m$の複素ベクトル束とする.
2.1節で考察したことをコンパクト多様体上に拡張するのが本節の目的である.

\begin{definition}
  $(U,\kappa,\tau)$が$E$のtotal trivializationであるとは,$(U,\kappa)$が$M$のチャート
  であり,$(U,\tau)$が$E$の局所自明化であることをいう.
\end{definition}

$E$上のSobolev空間を定義するため,まずは$\Gamma(E)$上の内積を定義する.$E$のtotal trivialization
からなる有限族$\{(U_i,\kappa_i,\tau_i)\}_{1\leq i \leq N}$を$\{(U_i)\}$が$M$の開被覆となるようにとり,
${U_i}$に従属する1の分割$\{\phi_i\}$をとる.さらに$\map{\sigma_i}{\Gamma(E)}{C^\infty{\kappa_i(U_i),\setC^m}}$
を,$f\in \Gamma(E),\ x\in \kappa_i(U_i)$に対して
\[ (x,\sigma_i(f)(x)) = \tau_i(f(\kappa_i^{-1}(x))) \in U_i\times \setC^m \]
をみたすように定める.以上のデータを用いて,$s\in \setZ$に対して$\map{\abracket{\cdot,\cdot}_s^E}{\Gamma(E)\times\Gamma(E)}{\setC}$を
\[ \abracket{f,g}^E_s=\sum_{1\leq i\leq N} \abracket{\sigma_i(\phi_i f),\sigma_i(\phi_i g)}_s \]
で定め,この内積が定めるノルムを$\norm{\cdot}^E_s$と書く.

\begin{proposition}
  $s\in\setZ$のとき,$\norm{\cdot}_s^E$の同値類は${(U_i,\kappa_i,\tau_i)}$および${\phi_i}$のとり方によらない.
\end{proposition}

\begin{definition}
  $s\in \setZ$について,$\Gamma(E)$の$\norm{\cdot}^E_s$(と同値なノルム)による完備化を$\sobolev{s}(E)$
  と書き,$E$上の$s$次のSobolev空間という.
\end{definition}

2.1節のときと同様に,各$s\in\setZ$に対して$\Gamma(E)$は$\sobolev{s}(E)$の部分集合とみなすことにする.

以下,本節では$\{U_i,\kappa_i,\tau_i\}$と$\{\phi_i\}$を1つ固定し,各$s\in\setZ$に対して$\abracket{\cdot,\cdot}_s$
をこれらのデータからできる内積とする.

\newpage
\section{擬微分作用素}

\subsection{擬微分作用素の定義}

\begin{definition}
  $d\in\setR$とする.$\map{p(x,\xi)}{\setR^n\times \setR^n}{\Hom(C^{m_1},\setC^{m_1})}$が次数$d$の
  \textbf{表象(symbol)}であるとは,次の条件をすべてみたすことをいう.
  \begin{enumarabicp}
    \item $p$は\smooth である.
    \item 任意の$\alpha,\beta\in\setN^n$に対して定数$C_{\alpha\beta}$が存在して
    \[ \forall (x,\xi) \in \setR^n\times\setR^n\quad \norm{D^\alpha_x D^\beta_\xi p(s,\xi)} \leq C_{\alpha \beta}(1+\abs{\xi})^{d-\abs{\xi}} \]
    をみたす.ここで$\norm{\cdot}$は$\Hom(\setC^{m_1},\setC^{m_2})$の作用素ノルムである.
  \end{enumarabicp}
  次数$d$の表象全体のなすベクトル空間を$S^d=S^d(m_1,m_2)$と書き,
  \[ S^\infty = \bigcup_{d\in\setR} S^d,\ \quad S^{-\infty} = \bigcap_{d\in\setR} S^d \]
  とおく.
\end{definition}

$p\in S^d$とする.このとき,$f\in \cptsmoothfct{n}{m_1}$に対して$\Psi_p f\in \smoothfct{n}{m_2}$
が
\[ \Psi_p f(x) = \int_\setR^n e^{ix\xi} p(x,\xi)\hat{f}(\xi)\dd xi \]
で定まる.

\begin{definition}
  $\map{\Psi_p}{\cptsmoothfct{n}{m_1}}{\smoothfct{n}{m_1}}$を$p$の定める$d$次の\textbf{擬微分作用素(pseudo-diffirential operator)}
  という.$d$次の擬微分作用素全体のなすベクトル空間を$\Psi^d$と書き,
  \[ \Psi^\infty = \bigcup_{d\in \setR} \Psi^d,\quad \Psi^{-\infty} = \bigcap_{d\in \setR} \Psi^d \]
  とおく.
\end{definition}

\begin{lemma}\label{symbol injectivity}
  $p$に$\Psi_p$を対応させる線型写像$S^d\to \Psi^d$は全単射である.
\end{lemma}

\begin{definition}
  $\map{\sigma}{\Psi^d}{S^d}$を\cref{symbol injectivity}の全単射の逆写像とし,全表象写像という.
\end{definition}

\begin{proposition}
  $p\in S^d$を$x$方向の台がコンパクトな表象とする.このとき,任意の$s\in\setR$で$\Psi_p$は$\sobolevlocal{s}{n}{m_1}$
  から$\sobolevlocal{s-d}{n}{m_2}$への有界線型写像に一意的に延長できる.
\end{proposition}

証明のため,今後繰り返し使うことになる基本的な不等式を導入する.

\begin{lemma}[Peetreの不等式]\label{Pettre inequality}
  任意の$x,y\in \setR^n$,\ $s\in\setR$に対して
  \[ (1+\abs{x+y})^s\leq(1+\abs{x})^{\abs{s}}(1+\abs{y})^s \]
  が成立する.
\end{lemma}

\subsection{表象の漸近展開}

\subsection{楕円性}

\newpage
\section{楕円型正則性}

\subsection{局所正則性}

\subsection{大域正則性}

\newpage
\section{Hodge分解}

\newpage
\bibliographystyle{MA}
\nocite{*}
\bibliography{data}

\end{document}

te