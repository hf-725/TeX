\documentclass[uplatex]{jsarticle}
\usepackage{MAstandard}
\DeclareMathOperator{\Int}{Int}
\newcommand{\smooth}{C^{\infty}}
\newcommand{\setinverse}[2]{#1 ^{-1}(#2)}
\DeclareMathOperator{\sign}{sign}

\title{関数空間の定理}
\author{hf\_725}
\date{}

\begin{document}
\maketitle
\begin{abstract}

\end{abstract}

\section{Ascoli-Arzel\`{a}の定理}

本節では$K$を空でない位相空間,$(X,d)$を空でない完備距離空間とする.$C(K,X)$を$K$から$X$への連続写像全体の集合とする.

\subsection{}

\begin{definition}
    $\mscrA\subset C(K,X)$が同程度連続であるとは,すべての$y\in K$と$\varepsilon > 0$に対して$y$の開近傍$U$が存在し,
    \[ \sup_{(f,y^\prime)\in \mscrA\times U} d(f(y),f(y^\prime)) < \varepsilon \]
    が成立することをいう.
\end{definition}

\begin{definition}
    $\mscrA\subset C(K,X)$が各点有界であるとは,すべての$y\in K$で
    \[ \mscrA_y = \set{f(y)}{f\in \mscrA} \]
    が$X$の有界集合であることをいう.また,ある$M > 0$が存在して
    \[ \sup_{f,g\in \mscrA} \sup_{y\in K} d(f(y),g(y)) < M \]
    が成立するとき,$\mscrA$は一様有界であるという.
\end{definition}

\begin{theorem}[Ascoli-Arzel\`{a}の定理]\label{AA1}
    $K$がコンパクト空間であるとする.このとき$C(K,X)$に距離$\rho$を
    \[ \rho(f,g) = \max_{y\in K} d(f(y),g(y)) \]
    で定めることができる.これによって$C(K,X)$を距離空間と考えたとき,$\mscrA\subset C(K,X)$について以下は同値である.
    \begin{enumarabicp}
        \item $\mscrA$は相対コンパクトである.
        \item $\mscrA$は同程度連続かつ各点有界である.
        \item $\mscrA$は同程度連続かつ一様有界である.
    \end{enumarabicp}
\end{theorem}

\begin{lemma}\label{bdd}
    定理\ref{AA1}の仮定の下で,$\mscrA$が同程度連続であるとする.このとき,$\mscrA$が各点有界であることと一様有界であることは同値である.
\end{lemma}
\begin{proof}
    $\mscrA$が一様有界なら各点有界であることは明らかである.逆を示すため,$\mscrA$が各点有界であるとする.$\mscrA$が同程度連続であることから,各$y\in K$についてその近傍$U_y$を十分小さくとるとすべての$f\in \mscrA$と$y^\prime\in U_y$について$d(f(y),f(y^\prime)) < 1$が成立するようにできる.$\{U_y\}_{y\in K}$は$K$の開被覆なので,$K$のコンパクト性から$y_1,\dots , y_n\in K$が存在して$\{U_{y_i}\}_{1\leq i\leq n}$が$K$の開被覆になる.$\mscrA$の各点有界性から$M > 0$が存在して,すべての$1\leq i \leq n$と$f,g\in \mscrA$に対して
    \[ d(f(y_i),g(y_i)) < M \]
    が成立する.このとき
    \[ \sup_{f,g\in \mscrA} \sup_{y\in k} d(f(y),g(y)) \leq M+2 \]
    が成立することを示す.任意に$y_0\in K$をとったとき,ある$1\leq i\leq n$が存在して$y\in U_{y_i}$が成立する.したがってすべての$f,g\in \mscrA$について
    \begin{align*}
        d(f(y_0),g(y_0)) & \leq d(f(y_0),f(y_i)) + d(f(y_i),g(y_i)) + d(g(y_i),g(y_0)) \\
                         & < 1 + M + 1 = M+2
    \end{align*}
    がとなる.以上より$\mscrA$は一様有界である.
\end{proof}

\begin{proof}[定理\ref{AA1}の証明]
    (2)と(3)が同値であることは補題\ref{bdd}で証明されている.また,$C(K,X)$が完備距離空間であることから$\mscrA$が相対コンパクトであることと全有界であることは同値である.

    \begin{subproof}{\IMPLIES{(1)}{(2)}}
        $\mscrA$が相対コンパクトであるとする.まず$\mscrA$が各点有界であることを示す.$\mscrA$の全有界性から$\mscrA$の有限個の元$f_1,\dots f_n$が存在して
        \[ \mscrA \subset \bigcup_{i=1}^n B_1(f_i) \]
        が成立する.ただし$f\in C(K,X)$と$r > 0$について
        \[ B_r(f) = \set{g\in C(K,X)}{\rho(f,g) < r} \]
        である.
        \[ M = \max \set{\rho(f_i,f_j)}{1\leq i < j \leq n} \]
        とおく.このとき任意の$f,g\in \mscrA$について$f\in B_1(f_i),g\in B_1(f_j)$をみたす$i,j$がとれて,
        \[ \rho(f,g) \leq \rho(f,f_i) + \rho(f_i.f_j) + \rho(f_j,g) \leq M+2 \]
        が成立する.これは$\mscrA$が各点有界であることを示している.

        次に$\mscrA$が同程度連続であることを示す.$\varepsilon>0$を任意にとる.$\mscrA$の全有界性より$f_1,\dots ,f_k\in \mscrA$をうまくとって
        \[ \mscrA \subset \bigcup_{i=1}^k B_{\frac{\varepsilon}{4}} (f_i) \]
        をみたすようにできる.$y\in K$を任意にとる.$y$の開近傍$U$を,$y^\prime\in U$ならすべての$1\leq i \leq k$で$d(f_i(y),f_i(y^\prime))<\varepsilon/4$となるようにとる.このとき任意に$f\in \mscrA$をとると$f\in B_{\varepsilon/4}(f_i)$となる$i$が存在するので,任意の$y^\prime\in U$に対して
        \begin{align*}
            d(f(y),f(y^\prime)) & \leq d(f(y),f_i(y)) + d(f_i(y),f_i(y^\prime)) + d(f_i(y^\prime),f(y))                               \\
                                & \leq \frac{\varepsilon}{4} + \frac{\varepsilon}{4} + \frac{\varepsilon}{4} = \frac{3\varepsilon}{4}
        \end{align*}
        である.したがって,
        \[ \sup_{(f,y^\prime)\in \mscrA\times U} d(f(y),f(y^\prime))\leq \frac{3\varepsilon}{4} < \varepsilon \]
        が成立し,これは$\mscrA$が全有界であることを示している.
    \end{subproof}

    \begin{subproof}{\IMPLIEDBY{(3)}{(1)}}

    \end{subproof}
\end{proof}

\subsection{}

\begin{theorem}[Ascoli-Arzel\`{a}の定理]
    $K$が可分であるとし,$S$を$K$の可算稠密部分集合とする.$\mscrA$を同程度連続な$C(K,X)$の部分集合で任意の$y\in S$で$\mscrA_{y}$が相対コンパクトであるとする.このとき,任意の$\mscrA$内の点列$\{f_n\}_n$に
    ついて,その部分列$\{g_n\}_n$を$K$上広義一様収束するようにとれる.
\end{theorem}
\begin{proof}
    $S$の元を$y_1,y_2,\dots$と番号づけしておく.$\{f_n(y_1)\}_n$が相対コンパクトであることから,$\{f_n\}_n$の部分列$\{f_{1,n}\}_n$を$\{f_{1,n}(y_1)\}_n$が収束するようにとれる.この操作を続けることで,任意の正整数$k$について$\{f_{k,n}\}_n$の部分列$\{f_{k+1,n}\}_n$を$\{f_{k+1,n}(y_{k+1})\}_n$が収束するようにとれる.そこで$g_n=f_{n,n}$とおくと,$\{g_n\}_n$は$\{f_n\}_n$の部分列であり任意の$k\in \setZ_{>0}$で$\{g_n(y_k)\}_n$は収束する.

    この$\{g_n\}_n$が$K$上広義一様収束することを示す.$L$を任意の$K$のコンパクト集合とする.任意に$\varepsilon>0$をとる.$\mscrA$の同程度連続性より,各$k\in \setZ_{>0}$で$y_k$の開近傍$U_k$を任意の$y\in U_k$に対して
    \[ \sup_{f\in \mscrA} d(f(y_k),f(y)) < \frac{\varepsilon}{3} \]
    が成立するようにとれる.$L$はコンパクトだから,ある$N_1\in \setZ_{>0}$が存在して
    \[ L \subset \bigcup_{k=1}^{N_1} U_k \]
    が成立する.このとき$N_2\in \setZ_{>0}$を十分大きくとると任意の$m,n>N_2$と$1\leq k\leq N_1$に対して
    \[ d(g_m(y_k),g_n(y_k)) < \frac{\varepsilon}{3} \]
    となる.したがって$m,n\in N_2$のとき,$y\in L$ならば$y\in U_k$となる$1\leq k \leq N_1$を選ぶことができて
    \begin{align*}
        d(g_m(y),g_n(y)) \leq d(g_m(y),g_m(y_k)) + d(g_m(y_k),g_n(y_k)) + d(g_n(y_k),g_n(y)) < \varepsilon
    \end{align*}
    が成立する.よって$\{g_n\}_n$は$L$上一様収束する.
\end{proof}

\section{Stone-Weierstrassの定理}
$K$で$\setR$または$\setC$を表すことにする.$X$をコンパクト空間とし,$C(X,K)$は$\sup$ノルムでBanach空間になっているとする.

\begin{definition}
    $\mscrA\subset C(X,K)$が$C(X,K)$の非単位的部分代数であるとは,次の条件が成り立つことをいう.
    \begin{enumarabicp}
        \item $f,g\in \mscrA$ならば$f+g\in \mscrA.$
        \item $f\in \mscrA,\alpha\in K$ならば$\alpha f\in \mscrA.$
    \end{enumarabicp}
    さらに$1\in \mscrA$であるならば$\mscrA$は$C(X,K)$の部分代数であるという.
\end{definition}

\begin{theorem}[Stone-Weierstrassの定理]\label{SWR}
    $C(X,\setR)$の非単位的部分代数$\mscrA$が$C(X,\setR)$の中で稠密であるための必要十分条件は次が成立することである.
    \begin{enumarabicp}
        \item 任意の$x,y\in X$に対して,ある$f\in \mscrA$で$f(x)\neq f(y)$となるものが存在する.
        \item 任意の$x\in X$に対して,$g(x)\neq 0$となる$g\in \mscrA$が存在する.
    \end{enumarabicp}
\end{theorem}

\begin{theorem}[Stone-Weierstrassの定理]\label{SWC}
    $C(X,\setC)$の非単位的部分代数$\mscrA$が$C(X,\setC)$の中で稠密であるための必要十分条件は次が成立することである.
    \begin{enumarabicp}
        \item 任意の$x,y\in X$に対して,ある$f\in \mscrA$で$f(x)\neq f(y)$となるものが存在する.
        \item 任意の$x\in X$に対して,$g(x)\neq 0$となる$g\in \mscrA$が存在する.
        \item $f\in \mscrA$ならば$\Bar{f}\in \mscrA$である.
    \end{enumarabicp}
    ここで,$f\in C(X,\setC)$について,$\Bar{f}$は
    \[ \Bar{f}(x) = \overline{f(x)} \]
    で定まる$C(X,\setC)$の元である.
\end{theorem}

\begin{remark}
    文献によっては上の定理で$X$がHausdorff空間であることを仮定している場合があるが,実はHausdorff性の仮定は$\mscrA$の条件に内包されている.実際,上の定理の仮定(1)をみたす$\mscrA$がとれるためには$X$はHausdorff空間でなければならないことが分かる.
\end{remark}

\begin{lemma}
    $\mscrA\subset C(X,\setC)$が非単位的部分代数のとき,$\Bar{\mscrA}$は非単位的部分代数である.
\end{lemma}
\begin{proof}

\end{proof}

\begin{proposition}[Diniの定理]
    $Y$をコンパクト空間とする.$C(Y,\setR)$の点列$\{f_n\}_n$について,任意の$y\in Y$で$\{f_n(y)\}_n$は広義単調増加で上に有界であるとする.このとき,$\{f_n\}_n$はある関数に一様収束する.
\end{proposition}
\begin{proof}

\end{proof}

\begin{lemma}
    $\sqrt{t}\in C([0,1],\setR)$に収束する$t$の多項式の列$\{h_n\}_n$が存在する.
\end{lemma}
\begin{proof}
    $\{h_n\}_n$を次の漸化式で定める.
    \[ h_{n+1}(t) = h_n(t) + \frac{t-h_n(t)^2}{2},\ h_0=0. \]
    この関数列について
    \[ h_n(t)\leq \sqrt{t},\  h_n(t)\leq h_{n+1}(t) \]
    が成立することが帰納法で示せる.したがってDiniの定理によって$\{h_n\}_n$はある関数に一様収束する.さらにその収束先が$\sqrt{t}$であることが分かる.
\end{proof}

\begin{proof}[定理\ref{SWR}の証明]

\end{proof}

\begin{proof}[定理\ref{SWC}の証明]

\end{proof}
\end{document}