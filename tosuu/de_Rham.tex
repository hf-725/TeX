\documentclass[uplatex]{jsarticle}
\usepackage{tosuustd2019}
\usepackage{type1cm}

\begin{document}
\title{de Rhamコホモロジーと幾何学}
\author{みつば(@mittlear1)}
\date{2018年9月10日}
\maketitle

\section{de Rhamコホモロジーの定義}
この章では種々の基本概念を紹介し,de Rhamコホモロジーを定義する.$\mathbb{R}^n$の座標を$(x_1,x_2,\cdots,x_n)$と書き,$U$を$\mathbb{R}^n$の開集合とする.
\begin{definition}[微分形式]
$C^{\infty}(U)$を$U$上の$C^{\infty}$級関数のなす環とする.
\begin{enumerate}[label=(\arabic*)]
\item $U$上の0次微分形式とは$C^{\infty}$級関数$f\colon U\rightarrow\mathbb{R}$のことである.0次微分形式全体の集合を$\Omega^0(U)$と書く.これは$C^{\infty}(U)$加群の構造を持つ.
\item $1\le k\le n$とする.$U$上の$k$次微分形式を,次の性質を満たす$C^{\infty}(U)$加群$\Omega^k(U)$の元のこととする.
  \begin{enumerate}
  \item $\Omega^k(U)$は
  \[ \{dx_{i_1} \wedge dx_{i_2} \wedge \cdots \wedge dx_{i_k}\mid 1\le i_1, \cdots, i_k\le n\} \]
  によって形式的に生成される.
  \item $\mathfrak{S}_k$を$k$次対称群とする.任意の$\sigma\in \mathfrak{S}_k$に対し関係式
  \[ dx_{i_{\sigma(1)}}\wedge\cdots\wedge dx_{i_{\sigma(k)}}= \operatorname{sgn}(\sigma)dx_{i_1}\wedge\cdots\wedge dx_{i_k} \]
  がある.
  \end{enumerate}
\item $0\le k\le n$でないとき,$\Omega^k(U):=0$ (一元集合)と定める.
\end{enumerate}
\end{definition}
堅苦しい定義になってしまったが,要するに微分形式とは
\begin{enumerate}
\item $f$を$C^{\infty}$級関数として$fdx_{i_1}\wedge\cdots\wedge dx_{i_k}$の形をしたものを足し合わせたものであり,
\item $dx_i$の部分を入れ替えると符号が変わる
\end{enumerate}
ようなもののことである.具体的に計算方法を見ていこう.
\begin{example}
$n=3$のとき
\begin{enumerate}[label=(\arabic*)]
\item 1次微分形式: $fdx_1+gdx_2+hdx_3.$
\item 2次微分形式: $fdx_1\wedge dx_2+gdx_2\wedge dx_3+hdx_3\wedge dx_1.$
\item 3次微分形式: $fdx_1\wedge dx_2\wedge dx_3.$
\end{enumerate}
また,
\begin{gather*}
 dx_i\wedge dx_i=0 \left(i=1,2,3\right), \\
 dx_2\wedge dx_1\wedge dx_3=-dx_1\wedge dx_2\wedge dx_3
\end{gather*}
である.
\end{example}
$k\in\mathbb{Z}$について$\Omega^k(U)$は自然に$\mathbb{R}$上のベクトル空間の構造を持つ.また,$1\le k \le n$のとき$\Omega^k(U)$は$C^{\infty}(U)$上
\[ \{dx_{i_1}\wedge\cdots\wedge dx_{i_k}\mid 1\le i_1<\cdots<i_k\le n \} \]
を基底に持つ.
\begin{definition}[外微分]
外微分$d$を次のように定義する.
\begin{enumerate}[label=(\arabic*)]
\item $f\in\Omega ^0(U)$のとき
\[ df:=\sum_{i=1}^n \frac{\partial f}{\partial x_i}dx_i . \]
\item $1\le k\le n$,$\omega=fdx_{i_1}\wedge\cdots\wedge dx_{i_k}\in\Omega^k(U)$のとき
\[ d\omega:=df\wedge dx_{i_1}\wedge\cdots\wedge dx_{i_k}=\sum^n_{i=1}\frac{\partial f}{\partial x_i}dx_i\wedge dx_{i_1}\wedge\cdots\wedge dx_{i_k} \]
と定め,これを線型に拡張する.
\item $\omega\in\Omega^n(U)$のとき
\[ d\omega :=0. \]
\end{enumerate}
\end{definition}
$k\ge 0$に対し,外微分は$\mathbb{R}$上のベクトル空間の間の線形写像
\[ d\colon \Omega^k(U)\rightarrow\Omega^{k+1}(U) \]
とみなすことができる.さらに次の性質が成り立つ.
\begin{proposition}\label{prop1}
すべての$k\geq 0$に対して
\[ dd=0\colon \Omega^k(U)\rightarrow\Omega^{k+2}(U) \]
が成立する.
\end{proposition}
\begin{proof}
ここでは$n=3$,$k=1$のときにのみ証明する(一般の場合でも本質は変わらない).$\omega:=fdx_i\in\Omega^1(U)$ ($i=1,2,3$)のとき
\begin{align*}
dd\omega &= d\left(\frac{\partial f}{\partial x_1}dx_1\wedge dx_i+\frac{\partial f}{\partial x_2}dx_2\wedge dx_i+\frac{\partial f}{\partial x_3}dx_3\wedge dx_i\right)   \\
&= \left(\frac{\partial^2 f}{\partial x_2 \partial x_1}dx_2\wedge dx_1\wedge dx_i+\frac{\partial^2 f}{\partial x_3\partial x_1}dx_3\wedge dx_1\wedge dx_i\right)  \\
&\quad + \left(\frac{\partial^2 f}{\partial x_1\partial x_2}dx_1\wedge dx_2\wedge dx_i+\frac{\partial^2 f}{\partial x_3\partial x_2}dx_3\wedge dx_2\wedge dx_i\right)  \\
&\qquad +\left(\frac{\partial^2 f}{\partial x_1\partial x_3}dx_1\wedge dx_3\wedge dx_i+\frac{\partial^2 f}{\partial x_2 \partial x_3}dx_2\wedge dx_3\wedge dx_i\right)   =0 
\end{align*}
となる.
\end{proof}
\begin{example}
$n=3$とする.

$f\in\Omega^0(U)$のとき
\[ df=\frac{\partial f}{\partial x_1}dx_1+\frac{\partial f}{\partial x_2}dx_2+\frac{\partial f}{\partial x_3}dx_3, \]
$\omega = fdx_1+gdx_2+hdx_3\in\Omega^1(U)$のとき
\[ d\omega = \left(\frac{\partial h}{\partial x_2}-\frac{\partial g}{\partial x_3}\right)dx_2\wedge dx_3+\left(\frac{\partial f}{\partial x_3}-\frac{\partial h}{\partial x_1}\right)dx_3\wedge dx_1+\left(\frac{\partial g}{\partial x_1}-\frac{\partial f}{\partial x_2}\right)dx_1\wedge dx_2, \]
$\eta = fdx_2\wedge dx_3+gdx_3\wedge dx_1+hdx_1\wedge dx_2$のとき
\[ d\eta = \left(\frac{\partial f}{\partial x_1}+\frac{\partial g}{\partial x_2}+\frac{\partial h}{\partial x_3}\right)dx_1\wedge dx_2\wedge dx_3. \]
すなわち外微分は
\[ \Omega^0(U)\stackrel{\operatorname{grad}}{\to}\Omega^1(U)\stackrel{\operatorname{rot}}{\to}\Omega^2(U)\stackrel{\operatorname{div}}{\to}\Omega^3(U) \]
であり,命題\ref{prop1}は
\[ \operatorname{rot}(\operatorname{grad}f)=0, \operatorname{div}(\operatorname{rot}\mathbf{F})=0 \]
を意味する.
\end{example}

\begin{definition}
\begin{enumerate}[label=(\arabic*)]
\item $\mathbb{R}$上のベクトル空間とその間の線形写像が作る図式
\[ \cdots \stackrel{d^{-k-1}}{\to}M^{-k}\stackrel{d^{-k}}{\to}M^{-k+1}\stackrel{d^{-k+1}}{\to}\cdots\stackrel{d^{-2}}{\to}M^{-1}\stackrel{d^{-1}}{\to}M^0\stackrel{d^0}{\to}M^1\stackrel{d^1}{\to}\cdots \stackrel{d^{k-1}}{\to}M^k\stackrel{d^k}{\to}M^{k+1}\stackrel{d^{k+1}}{\to} \cdots \]
がコチェイン複体であるとは,すべての$k\in\mathbb{Z}$に対して
\[ d^{k+1}\circ d^k=0 \]
が成立することである.この図式を$(M^*,d^*)$または単に$M^*$と書く.
\item 命題\ref{prop1}により
\[ \stackrel{d}{\to}\Omega^{-k}(U)\stackrel{d}{\to}\cdots\stackrel{d}{\to}\Omega^0(U)\stackrel{d}{\to}\Omega^1(U)\stackrel{d}{\to}\cdots\stackrel{d}{\to}\Omega^k(U)\stackrel{d}{\to}\cdots \]
はコチェイン複体となる.これをde Rham複体といい,$(\Omega^*(U),d)$または単に$\Omega^*(U)$と書く.
\end{enumerate}
\end{definition}
\begin{definition}
$(M^*,d^*)$をコチェイン複体とする.すべての$k\in \mathbb{Z}$に対して$\Ker d^k\supset\Image d^{k-1}$であるから商ベクトル空間
\[ H^k(M^*):=\Ker d^k/\Image d^{k-1} \]
が定まる.これをコチェイン複体$(M^*,d^*)$の$k$次コホモロジーという.さらに
\[ H^*(M^*):=\{H^k(M^*)\}_{k\in\mathbb{Z}} \]
と書く.
\end{definition}
\begin{definition}
de Rham複体$(\Omega^*(U),d)$の$k$次コホモロジー
\[ H^k(U):=H^k(\Omega^*(U)) \]
を$k$次de Rhamコホモロジーという.
\end{definition}
\begin{example}
$n=3$とする.
\begin{gather*}
 H^1(U)=\Ker (\operatorname{rot})/\Image (\operatorname{grad}), \\
 H^2(U)=\Ker (\operatorname{div})/\Image (\operatorname{rot})
\end{gather*}
である.よって,$H^1(U)$,$H^2(U)$は「スカラーポテンシャル・ベクトルポテンシャルを持たないベクトル場がどのくらいあるか」を測る解析的な量と思える.
\end{example}
\begin{example}
$U=\mathbb{R}^1$のときのde Rhamコホモロジーを計算する.$H^0(U)$について,
\[ df=0 \Leftrightarrow \frac{df}{dx}=0 \Leftrightarrow f=\mathrm{const} .\]
したがって$H^0(U)\cong \mathbb{R}$である.

次に$H^1(U)$を計算する.$fdx\in \Ker d=\Omega^1(\mathbb{R}^1)$に対し$F\colon \mathbb{R}^1\to\mathbb{R}$を
\[ F(x):=\int^x_0 f(t)dt \]
で定めると,微分積分学の基本定理から
\[ dF=\frac{dF}{dx}dx=fdx. \]
よって$H^1(\mathbb{R}^1)=0$となる.

実は,より一般に
\begin{eqnarray*}
  H^k(\mathbb{R}^n)=
  \begin{cases}
    \mathbb{R} & (k=0) \\
    0\ & (k\neq 0)
  \end{cases}
\end{eqnarray*}
が成立する.
\end{example}
\begin{remark}
上の例からの類推で,$U$が連結であるとき0次de Rhamコホモロジーは$\mathbb{R}$に同型であることが分かる.
\end{remark}
\section{de Rhamコホモロジーの幾何学的意味}
この章では,de Rhamコホモロジーの持つ幾何学的意味を解説する.結論から言うと,$U$が単純な図形であれば,de Rhamコホモロジーは「空間内の穴の数を数えている」と考えられる.これを詳しく見ていく\footnote{この章では,厳密性よりもイメージをつかむことを重視した.曖昧なところが多いのは許してほしい.}.

空間の穴の個数を数える別の道具として特異ホモロジー群があるが,大雑把に言うと,特異ホモロジーは空間内の穴を包む「非自明な$k$次元球」がいくつあるか,を表現している.これに対してde Rhamコホモロジーは少し複雑で,空間内に球が置かれているとき,それが穴を含む「非自明な$k$次元球」か,穴を含まない「自明な$k$次元球」かを選り分ける機械のようなものである.この解釈を正当化してくれる定理としてStokesの定理を紹介する.簡単のため,考える空間は2次元空間であるとする.

\begin{definition}[微分形式の積分]
$U$を$\mathbb{R}^2$の開集合とする.
\begin{enumerate}[label=(\arabic*)]
\item $D\subset U$を向きづけられた有界閉領域,$\omega:=fdx_1\wedge dx_2\in \Omega^2(U)$とする.$\omega$の$D$上での積分を
\[ \int_D \omega :=\int_D fdx_1\wedge dx_2 \]
で定める.
\item $C\subset U$を向きづけられた閉曲線とし,$\eta=fdx_1+gdx_2\in \Omega^1(U)$とする.$\eta$の$C$上での積分を
\[ \int_C \eta := \int_C (fdx_1+gdx_2) \]
で定める.
\end{enumerate}
\end{definition}
\begin{theorem}[Stokesの定理]
$U\subset \mathbb{R}^2$を開集合,$D\subset U$を向きづけられた有界閉領域とする.このとき,任意の$\eta\in \Omega^1(U)$に対して
\[ \int_D d\eta = \int_{\partial D} \eta \]
が成立する.
\end{theorem}
これを基に$U=\mathbb{R}^2\setminus \{0\}$の1次de Rhamコホモロジーについて考える.$U$上の1次微分形式
\[ \omega:=\frac{-ydx+xdy}{x^2+y^2} \]
をとる.計算により,$d\omega=0$となるので
\[ [\omega]\in H^1(U) \]
が定まる.

$U=\mathbb{R}^2\setminus \{0\}$は$0\in U$に「穴」を持ち,したがってその穴を取り囲む「非自明なループ」が存在する.$\omega$はちょうど,この「非自明なループ」を検出する機械の役割を果たしている.2つのループ
\begin{align*}
  C_1&:=\{(x,y)\in U\mid x^2+y^2=1\}, \\
  C_2&:=\{(x,y)\in U\mid (x-2)^2+y^2=1\} 
\end{align*}
を考える.$C_1$は「非自明なループ」,$C_2$は「自明なループ」である.直接計算により
\[ \int_{C_1} \omega=2\pi, \quad \int_{C_2} \omega=0 \]
となることが分かる.さらに,$D_1$,$D_2$として別の「非自明なループ」,「自明なループ」をとると,これらを境界に持つ閉領域$E_1$,$E_2$\footnote{もちろんこのような都合のよい領域の存在は非自明であるが,だいたいこのようなイメージでよい.}に対してストークスの定理を適用することにより
\[ \int_{\partial E_1} \omega = \int_{E_1} d\omega=0,\quad  \int_{\partial E_2} \omega = \int_{E_2} d\omega=0, \]
すなわち
\[ \int_{D_1} \omega = 2\pi,\quad  \int_{D_2} \omega = 0 \]
となる.この結果は,
\begin{quotation}
$\omega$は「非自明なループ」だけをもれなく検出する
\end{quotation}
と標語化することができる.

実は,次の命題が成り立つ.
\begin{proposition}
$U$,$\omega$を上の通りとする.$U$上の1次微分形式$\eta$が$d\eta=0$を満たすとき,$c\in \mathbb{R}$と$f\in \Omega^0(U)$が存在して
\[ \eta=c\omega+df \]
と書ける.
\end{proposition}
\begin{proof}[略証]
\[ c:=\frac{1}{2\pi}\int_{C_1} \eta \]
ととる.また,$f$は次のように定める.$p\in U$を固定する.$p$と原点を結んだ線と$C_1$との交点を$p^{\prime}$とする.$l_p\colon [0,1]\to U$を,$(1,0)$を出発して$C_1$上を反時計回りに$p^{\prime}$まで進み,$p^{\prime}$から$p$までまっすぐ進むような曲線とする.そして
\[ f(p):=\int_{l_p} (\eta-c\omega) \]
とおく.これが条件を満たしていることを示せばよい.
\end{proof}
\begin{corollary}\label{cor1}
$U=\mathbb{R}^2\setminus \{0\}$とすると,$H^1(U)=\mathbb{R}$.
\end{corollary}
次に,$\mathbb{R}^2$から穴を$m$個取り除いた空間$V$を考える.このとき,「非自明なループ」は各々の穴に対応して存在し,本質的には全部で$m$個あると考えられる.そしてそのループそれぞれに,それを検出する1次微分形式があるはずである.したがって,
\[ H^1(V)=\mathbb{R}^m \]
が成立すると予想される.これが正しいことは次章で見る.
\section{Mayer-Vietoris完全列}
前章のような,やや複雑な空間のde Rhamコホモロジーはどのようにして計算すればよいのだろうか.この問いに答える道具の一つがMayer-Vietoris完全列である.以下,断らない限り$U$を$\mathbb{R}^n$の開集合とする.
\begin{definition}
\begin{enumerate}[label=(\arabic*)]
\item $\mathbb{R}$上のベクトル空間の列
\[ L\stackrel{f}{\to}M\stackrel{g}{\to}N \]
が完全であるとは,
\[ \Image f=\Ker g \]
が成立することである.
\item $\mathbb{R}$上のベクトル空間の列
\[ \cdots\to M^{-k}\to M^{-k+1}\to\cdots\to M^0\to M^1\to\cdots\to M^{k}\to M^{k+1}\to\cdots \]
が完全であるとは,すべての$l\in \mathbb{Z}$で
\[ M^{l-1}\to M^l\to M^{l+1} \]
が完全であることである.
\end{enumerate}
\end{definition}
\begin{theorem}[Mayer-Vietoris完全列]\label{thm1}
$\{V,W\}$を$U$の開被覆とする.このとき完全列
\begin{align*}
\cdots&\to H^k(U)\to H^k(V)\oplus H^k(W)\to H^k(V\cap W) \\
&\to H^{k+1}(U)\to H^{k+1}(V)\oplus H^{k+1}(W)\to H^{k+1}(V\cap W)\to\cdots
\end{align*}
が存在する.
\end{theorem}
証明のためには,まずコホモロジーの間の準同型の構成から始めなければならない.
\begin{definition}
$U$を$\mathbb{R}^n$の開集合,$V$を$\mathbb{R}^m$の開集合とする.写像$p_i\colon V\to \mathbb{R}$ ($i=1,2,\cdots m$)を第$i$成分への射影とする.また,$C^{\infty}$級写像$f\colon U\to V$に対して$f_i:=p_i\circ f\colon U\to\mathbb{R}$とする.$f^*\colon \Omega^k(V)\to\Omega^k(U)$を,$\omega=gdy_{i_1}\wedge dy_{i_2}\wedge\cdots\wedge dy_{i_k}\in\Omega^k(V)$に対し
\[ f^*\omega:=\left(g\circ f\right)df_{i_1}\wedge df_{i_2}\wedge\cdots\wedge df_{i_k} \]
で定め,これを線型に拡張する.$\eta\in\Omega^k(V)$に対し,$f^*\eta$を$\eta$の$f$による引き戻しという.
\end{definition}
引き戻しの簡単な性質を述べる.
\begin{proposition}
$U$を$\mathbb{R}^n$の開集合,$V$を$\mathbb{R}^m$の開集合,$W$を$\mathbb{R}^l$の開集合とする.
\begin{enumerate}[label=(\arabic*)]
\item $\mathrm{id}_U\colon U\to U$について,すべての$k\in\mathbb{Z}$に対し
\[ \mathrm{id}^*_U=\mathrm{id}_{\Omega^k(U)}\colon \Omega^k(U)\to \Omega^k(U). \]
\item $f\colon U\to V$,$g\colon V\to W$を$C^{\infty}$級写像とすると,
\[ \left(g\circ f\right)^*=f^*g^*\colon \Omega^k(W)\to \Omega^k(U). \]
\end{enumerate}
\end{proposition}
\begin{proof}
どちらも定義から明らか.
\end{proof}
\begin{definition}
$(M^*,d^*_M)$,$(N^*,d^*_N)$をコチェイン複体とする.
$\varphi^*=\{\varphi^k\colon M^k\to N^k\}_{k\in\mathbb{Z}}$がコチェイン写像であるとは,すべての$k\in\mathbb{Z}$で
\[ d^k_N\circ \varphi^k=\varphi^{k+1}\circ d^k_M \]
が成立していることをいう.このとき$\varphi^*\colon(M^*,d^*_M)\to(N^*,d^*_N)$と書く.
\end{definition}
\begin{proposition}
$f^*:=\{f^k\colon \Omega^k(V)\to \Omega^k(U) \}_{k\in\mathbb{Z}}$はコチェイン写像である.
\end{proposition}
\begin{proof}
\begin{align*}
&\phantom{=}\mathrel{}f^*d\left(gdy_{i_1}\wedge dy_{i_2}\wedge\cdots\wedge dy_{i_k}\right) \\
&=f^*\left(\sum^m_{i=1} \frac{\partial g}{\partial y_i} dy_i\wedge dy_{i_1}\wedge\cdots\wedge dy_{i_k}\right) \\
&=\sum^m_{i=1}\left(\frac{\partial g}{\partial y_i}\circ f\right)df_i\wedge df_{i_1}\wedge\cdots\wedge df_{i_k} \\
&=\sum^m_{i=1}\left(\frac{\partial g}{\partial y_i}\circ f\right)\left(\sum^n_{j=1}\frac{\partial f_i}{\partial x_j}dx_j\right)\wedge df_{i_1}\wedge\cdots\wedge df_{i_k} \\
&=\sum^n_{j=1}\left(\sum^m_{i=1}\left(\frac{\partial g}{\partial y_i}\circ f\right)\frac{\partial f_i}{\partial x_j}dx_j\right)\wedge df_{i_1}\wedge\cdots\wedge df_{i_k} \\
&=\sum^n_{j=1}\frac{\partial(g\circ f)}{\partial x_j}dx_j\wedge\cdots\wedge df_{i_k} \\
&=d\left(\left(g\circ f\right)df_{i_1}\wedge\cdots\wedge df_{i_k}\right) \\
&=df^*\left(gdy_{i_1}\wedge\cdots\wedge dx_{i_k}\right). \qedhere
\end{align*}
\end{proof}
コチェイン写像の重要性は,それがコホモロジーの間の準同型を誘導することにある.
\begin{proposition}
$(L^*,d^*_L)$,$(M^*,d^*_M)$をコチェイン複体とし,$f^*\colon(L^*,d^*_L)\to(M^*,d^*_M)$をコチェイン写像とする.$f^*$はすべての$k\in\mathbb{Z}$で
\[ H^k(f^*)\colon H^k(L^*)\to H^k(M^*);[l]\mapsto [f^k(l)] \]
を満たす線型写像を誘導する.さらに,$(N^*,d^*_N)$をコチェイン複体,$g^*\colon (M^*,d^*_M)\to (N^*,d^*_N)$をコチェイン写像とすると,すべての$k\in\mathbb{Z}$で
\begin{gather*}
  H^k(\mathrm{id}^*_L)=\mathrm{id}_{H^k(L^*)}, \\
  H^k(g^*\circ f^*)=H^k(g^*)\circ H^k(f^*)\colon H^k(L^*)\to H^k(N^*) 
\end{gather*}
が成立する.
\end{proposition}
証明は省略する.$H^k(f^*)$のことを$f^*$と略記することがある.
\begin{definition}
複体の図式
\[ (L^*,d^*_L)\stackrel{f^*}{\to}(M^*,d^*_M)\stackrel{g^*}{\to}(N^*,d^*_N) \]
が完全であるとは,すべての$k\in\mathbb{Z}$で
\[ L^k\stackrel{f^k}{\to}M^k\stackrel{g^k}{\to}N^k \]
が完全であることである.
\end{definition}
\begin{proposition}
複体の完全列
\[ 0\to (L^*,d^*_L)\stackrel{f^*}{\to}(M^*,d^*_M)\stackrel{g^*}{\to}(N^*,d^*_N)\to 0 \]
は完全列
\begin{align*}
\cdots&\to H^k(L^*)\stackrel{f^*}{\to}H^k(M^*)\stackrel{g^*}{\to}H^k(N^*) \\
&\to H^{k+1}(L^*)\stackrel{f^*}{\to}H^{k+1}(M^*)\stackrel{g^*}{\to}H^{k+1}(N^*)\to\cdots
\end{align*}
を誘導する.これをコホモロジー長完全列という.
\end{proposition}
この命題は(コ)ホモロジー論において非常に重要である.証明はたとえば\cite{枡田}にある.

この命題により,次の補題を示せばよいことが分かる.
\begin{lemma}\label{lem1}
記号を定理\ref{thm1}の通りとし,
\[ i_V\colon V\to U,  i_W\colon W\to U,  \iota_V\colon V\cap W\to V, \iota_W\colon V\cap W\to W \]
を包含とする.このとき
\[ 0\to\Omega^*(U)\stackrel{(i^*_V,i^*_W)}{\to}\Omega^*(V)\oplus\Omega^*(W)\stackrel{-\iota^*_V+\iota^*_W}{\to}\Omega^*(V\cap W)\to 0 \]
は完全である.
ただし,$\omega\in\Omega^k(U)$,$(\eta,\tau)\in\Omega^k(V)\oplus\Omega^k(W)$に対して
\begin{gather*}
 (i^*_V,i^*_W)(\omega)=(i^*_V\omega,i^*_W\omega), \\
 (-\iota^*_V+\iota^*_W)(\eta,\tau)=-\iota^*_V \eta+\iota^*_W \tau
\end{gather*}
である.
\end{lemma}
準備として,次の事実を認める.証明は\cite{松本}を参照.
\begin{proposition}
記号を定理\ref{thm1}の通りとする.このとき,$C^{\infty}$級関数$\rho_V,\rho_W\colon U\to \mathbb{R}$が存在して次が成り立つ.
\begin{enumerate}[label=(\roman*)]
\item $0 \leq \rho_V\leq 1, 0\leq \rho_W\leq 1.$
\item $\operatorname{supp}(\rho_V)\subset V, \operatorname{supp}(\rho_W)\subset W$.ただし,$f\colon U\to\mathbb{R}$に対し$\operatorname{supp}(f)$は$\{p\in U\mid f(p)\neq 0\}$の閉包である.
\item すべての$p\in U$で$\rho_V(p)+\rho_W(p)=1.$
\end{enumerate}
$\{\rho_V,\rho_W\}$のことを$\{V,W\}$に従属する1の分割という.
\end{proposition}
\begin{proof}[補題\ref{lem1}の証明]
$-\iota^*_V+\iota^*_W$の全射性だけ示す.$\{\rho_V,\rho_W\}$を$\{V,W\}$に従属する1の分割とする.$\alpha\in\Omega^k(V\cap W)$を任意にとると,
\[ -\rho_W\alpha\in\Omega^k(V),\quad \rho_V\alpha\in\Omega^k(W) \]
と見なせる.さらにこのとき,
\[ (-\iota^*_V+\iota^*_W)(-\rho_W\alpha,\rho_V\alpha)=\alpha \]
となる.
\end{proof}
これで定理\ref{thm1}の証明が完了した.これを用いて2章の予想を証明する.
\begin{theorem}
$\mathbb{R}^2$から$m$個の点を取り除いた空間を$U_m$とする.このとき
\begin{eqnarray*}
H^k(U_m)=
  \begin{cases}
    \mathbb{R} & (k=0)  \\
    \mathbb{R}^m &(k=1) \\
    0 & (k\neq 0,1)
  \end{cases}
\end{eqnarray*}
\end{theorem}
\begin{proof}
$m$についての帰納法で示す.$m=1$のときは系\ref{cor1}で示されている.$l\leq m$で正しいとき$l=m+1$で正しいことを示す.$U_m$の開被覆$\{V,W\}$を,$V$が穴を1個,$W$が穴を$m$個含み,さらに$V\cap W$が$\mathbb{R}^2$と微分同相になるようにとる.すると,定理\ref{thm1}から完全列
\[ H^0(V)\oplus H^0(W)\to H^0(V\cap W)\to H^1(U_{m+1})\to H^1(V)\oplus H^1(W) \]
がある.
\[ \Ker (H^0(V\cap W)\to H^1(U_{m+1}))=\Image(H^0(V)\oplus H^0(W)\to H^0(V\cap W))=\mathbb{R} \]
であるから,
\[ \Ker(H^1(U_{m+1})\to H^1(V)\oplus H^1(W))= \Image(H^0(V\cap W)\to H^1(U_{m+1}))=0 \]
となり,$H^1(U_{m+1})\to H^1(V)\oplus H^1(W)$の単射性が分かる.再び定理\ref{thm1}から
\[ H^1(U_{m+1})\to H^1(V)\oplus H^1(W)\to H^1(V\cap W)=0 \]
は完全なので$H^1(U_{m+1})\to H^1(V)\oplus H^1(W)$は全射でもある.以上より
\[ H^1(U_{m+1})=\mathbb{R}^{m+1} \]
となる.他の$k$についても同様にできる.
\end{proof}
\begin{thebibliography}{9}
\bibitem{Bott-Tu}
  Bott, R., and Tu, L., 
  \textit{Differential Forms in Algebraic Topology},
  Springer, 1982.
\bibitem{松本}
  松本幸夫,『多様体の基礎』,
  東京大学出版会,1998.
\bibitem{枡田}
  枡田幹也,『代数的トポロジー』,
  朝倉出版,2002.
\bibitem{小林}
  小林昭七,『曲線と曲面の微分幾何』,
  裳華房,1995.
\end{thebibliography}
\end{document}
































