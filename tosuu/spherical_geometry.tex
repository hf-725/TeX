\documentclass[uplatex]{jsarticle}
\usepackage{tosuustd2019}
\usepackage{type1cm}

\DeclareMathOperator{\capp}{cap}

\begin{document}

\title{球面幾何の世界}
\author{みつば(@mittlear1)}
\date{2017年9月4日}
\maketitle
\begin{abstract}
球面幾何学は非ユークリッド幾何学の代表例で、平行線公準を満たさないモデルとして知られている.主な研究対象は球面上の領域の面積であり,球面上の三角形の面積公式や三角形の「等積変形」の原理などユークリッド幾何学とは似ても似つかない結果が多く存在する.ここでは,これらの話題を中心に球面幾何学の世界を紹介する.
\end{abstract}



\section{球面上の多角形}
以下,考える球は全て半径1の単位球であるとする.この節では球面幾何の基礎的概念を定義し,球面上の三角形の面積公式を導出する.

\begin{definition} 
\begin{enumerate}[label=(\arabic*)]
\item 球の中心を通る平面での切り口の円を大円という.
\item 球面上の点$A$と球の中心$O$を通る$\mathbb{R}^3$内の直線と球の$A$でない方の交点を点$A$の対心点といい,$A^*$で
表す.
\item 球面上の点$A$,$B$を結ぶ直線とは,点$A$,$B$を通る大円のこと.点$A$から点$B$までの球面距離とは,大円
上の点$A$から点$B$までの線分の長さのうち短い方のこと.
\item 球面距離が$\pi$の点$A$,$B$を通る二つの線分によって囲まれる領域を月形という.
\item 球$S$とそれに外接する円筒$\Gamma$を用意する.円筒の中心を通る直線を$l$とする.$S$から極$S\cap l$を除いたもの
を$\check{S}$とする.点$X\in\check{S}$に対して,直線$l$から点$X$に向かい$l$に垂直な半直線と$\Gamma$との交点を$\varphi(X)$とする
とき,$\varphi:\check{S}→\Gamma$を円柱投影という.
\end{enumerate}
\end{definition}

円柱投影について,次の定理が成り立つ.

\begin{theorem}[アルキメデス・ランバートの円柱投影定理] 円柱投影は面積を保存する.つまり,領域$W\subset\check{S}$,$\varphi(W)$ともに面積が定義できるとき$W$と$\varphi(W)$の面積は等しい.
\end{theorem}
\begin{proof}
$A(X)$で領域$X$の面積を表す.球面上の図形$W$のパラメータ表示を
\[ f(\theta,\psi)=\begin{pmatrix} \cos \theta \cos \psi\\\cos \theta \sin \psi\\\sin \theta \end{pmatrix}  ((\theta,\psi)\in U)\]
とすると,曲面積の定義から
\[ A(W)=\iint_U |f_{\theta}\times f_{\psi}|d\theta d\psi \]
となる.同様に
\[ g=\varphi \circ f \]
と定義すると
\[ A(\varphi(W))=\iint_U |g_{\theta}\times g_{\psi}|d\theta d\psi \]
であるから,
\[ |f_{\theta}\times f_{\psi}|=|g_{\theta}\times g_{\psi}| \]
がすべての$(\theta,\psi)$で成立することを示せばよい.まず,
\[ f_{\theta}=\begin{pmatrix}-\sin \theta \cos \psi \\-\sin \theta \sin \psi \\\cos \theta \end{pmatrix},f_{\psi}=\begin{pmatrix}-\cos \theta \sin \psi \\\cos \theta \cos \psi \\0 \end{pmatrix} .\]
したがって
\[ f_{\theta} \times f_{\psi}=\begin{pmatrix}-\cos ^2 \theta \cos \psi\\-\cos ^2 \theta \sin \psi\\-\sin \theta \cos \theta \end{pmatrix}=-\cos \theta f(\theta,\psi) \]
である.また,
\[ g(\theta,\psi)=\begin{pmatrix}  \cos \psi\\ \sin \psi\\\sin \theta \end{pmatrix} \]
であるから
\[ g_\theta=\begin{pmatrix}0 \\ 0 \\ \cos \theta \end{pmatrix},g_\psi=\begin{pmatrix}-\sin\psi\\ \cos \psi \\0 \end{pmatrix}. \]
したがって
\[ g_{\theta}\times g_{\psi}=\begin{pmatrix} -\cos \theta \cos \psi\\ -\cos \theta \sin \psi \\ 0 \end{pmatrix} \]
以上より
\[ |f_{\theta}\times f_{\psi}|=|g_{\theta}\times g_{\psi}| \]
となることが確かめられる.
\end{proof}

\begin{corollary}
半径1の球の表面積は$4\pi$である.
\end{corollary}

次に,球面上の角を定義する.

\begin{definition}
球面上の異なる3点$A,B,C$において,$\angle ABC$を「直線$AB$の点$B$における$\mathbb{R}^3$内での接線と直線$CB$の点$B$における$\mathbb{R}^3$内での接線のなす角」と定義する.また,球面上の凸$n$角形とは,隣り合う3点が同一直線上にない$n$個の角を持つ凸領域のこと.
\end{definition}

球面上の多角形の面積を$[ABC]$などと表すことにする.次に示すのが球面上の三角形の面積公式である.

\begin{theorem}[ジラールの公式]球面上の三角形の面積は
\[ [ABC]=\angle A+\angle B+\angle C-\pi \]
で与えられる.
\end{theorem}
\begin{center}
%WinTpicVersion4.32a
{\unitlength 0.1in%
\begin{picture}(16.7000,16.7000)(18.0000,-74.7000)%
% CIRCLE 2 0 3 0 Black White  
% 4 2635 6635 3463 6529 3463 6529 3463 6529
% 
\special{pn 8}%
\special{ar 2635 6635 835 835 0.0000000 6.2831853}%
% SPLINE 2 0 3 0 Black White  
% 42 2308 7386 2292 7377 2277 7363 2265 7345 2255 7323 2248 7295 2241 7264 2239 7229 2237 7190 2238 7147 2241 7101 2246 7053 2255 7000 2264 6947 2276 6891 2289 6832 2305 6774 2323 6713 2342 6652 2363 6591 2385 6530 2409 6470 2434 6410 2460 6351 2486 6294 2513 6240 2542 6188 2570 6138 2598 6090 2627 6047 2655 6007 2684 5970 2711 5937 2738 5909 2763 5885 2789 5865 2813 5850 2835 5839 2856 5834 2875 5833 2893 5837 2893 5837
% 
\special{pn 8}%
\special{pa 2308 7386}%
\special{pa 2282 7368}%
\special{pa 2263 7342}%
\special{pa 2252 7312}%
\special{pa 2245 7281}%
\special{pa 2240 7249}%
\special{pa 2238 7217}%
\special{pa 2237 7185}%
\special{pa 2238 7153}%
\special{pa 2240 7121}%
\special{pa 2242 7090}%
\special{pa 2245 7058}%
\special{pa 2250 7026}%
\special{pa 2256 6995}%
\special{pa 2261 6963}%
\special{pa 2267 6932}%
\special{pa 2274 6900}%
\special{pa 2288 6838}%
\special{pa 2304 6776}%
\special{pa 2313 6745}%
\special{pa 2322 6715}%
\special{pa 2332 6684}%
\special{pa 2341 6654}%
\special{pa 2352 6623}%
\special{pa 2362 6593}%
\special{pa 2384 6533}%
\special{pa 2408 6473}%
\special{pa 2420 6444}%
\special{pa 2432 6414}%
\special{pa 2458 6356}%
\special{pa 2471 6326}%
\special{pa 2484 6297}%
\special{pa 2498 6269}%
\special{pa 2513 6240}%
\special{pa 2528 6212}%
\special{pa 2560 6156}%
\special{pa 2575 6128}%
\special{pa 2591 6101}%
\special{pa 2627 6047}%
\special{pa 2645 6021}%
\special{pa 2664 5995}%
\special{pa 2704 5945}%
\special{pa 2725 5922}%
\special{pa 2748 5899}%
\special{pa 2772 5878}%
\special{pa 2798 5859}%
\special{pa 2826 5843}%
\special{pa 2856 5834}%
\special{pa 2888 5836}%
\special{pa 2893 5837}%
\special{fp}%
% SPLINE 2 2 3 0 Black White  
% 42 3413 6340 3419 6357 3420 6376 3416 6397 3409 6421 3396 6446 3380 6474 3359 6502 3334 6532 3304 6562 3270 6594 3234 6627 3193 6660 3150 6694 3103 6728 3054 6761 3003 6795 2950 6827 2894 6859 2838 6891 2781 6921 2722 6950 2664 6977 2605 7003 2548 7028 2490 7050 2435 7070 2380 7087 2328 7103 2277 7117 2229 7127 2183 7136 2141 7142 2102 7144 2067 7145 2035 7143 2007 7138 1983 7131 1965 7121 1949 7108 1939 7094 1939 7094
% 
\special{pn 8}%
\special{pn 8}%
\special{pa 3413 6340}%
\special{pa 3415 6348}%
\special{fp}%
\special{pa 3418 6385}%
\special{pa 3416 6393}%
\special{fp}%
\special{pa 3405 6429}%
\special{pa 3402 6436}%
\special{fp}%
\special{pa 3383 6469}%
\special{pa 3378 6475}%
\special{fp}%
\special{pa 3356 6506}%
\special{pa 3351 6512}%
\special{fp}%
\special{pa 3326 6540}%
\special{pa 3320 6546}%
\special{fp}%
\special{pa 3293 6572}%
\special{pa 3287 6578}%
\special{fp}%
\special{pa 3260 6604}%
\special{pa 3254 6609}%
\special{fp}%
\special{pa 3225 6634}%
\special{pa 3219 6639}%
\special{fp}%
\special{pa 3190 6663}%
\special{pa 3183 6668}%
\special{fp}%
\special{pa 3154 6691}%
\special{pa 3147 6696}%
\special{fp}%
\special{pa 3117 6718}%
\special{pa 3111 6723}%
\special{fp}%
\special{pa 3079 6744}%
\special{pa 3073 6749}%
\special{fp}%
\special{pa 3041 6770}%
\special{pa 3034 6774}%
\special{fp}%
\special{pa 3003 6795}%
\special{pa 2996 6799}%
\special{fp}%
\special{pa 2964 6819}%
\special{pa 2957 6823}%
\special{fp}%
\special{pa 2924 6842}%
\special{pa 2917 6846}%
\special{fp}%
\special{pa 2884 6864}%
\special{pa 2877 6869}%
\special{fp}%
\special{pa 2845 6888}%
\special{pa 2838 6892}%
\special{fp}%
\special{pa 2804 6909}%
\special{pa 2797 6913}%
\special{fp}%
\special{pa 2764 6930}%
\special{pa 2756 6933}%
\special{fp}%
\special{pa 2722 6950}%
\special{pa 2715 6953}%
\special{fp}%
\special{pa 2681 6969}%
\special{pa 2674 6973}%
\special{fp}%
\special{pa 2639 6988}%
\special{pa 2632 6991}%
\special{fp}%
\special{pa 2597 7007}%
\special{pa 2590 7010}%
\special{fp}%
\special{pa 2555 7025}%
\special{pa 2548 7028}%
\special{fp}%
\special{pa 2513 7042}%
\special{pa 2505 7044}%
\special{fp}%
\special{pa 2470 7057}%
\special{pa 2462 7060}%
\special{fp}%
\special{pa 2426 7073}%
\special{pa 2419 7075}%
\special{fp}%
\special{pa 2383 7086}%
\special{pa 2375 7089}%
\special{fp}%
\special{pa 2339 7100}%
\special{pa 2331 7102}%
\special{fp}%
\special{pa 2295 7112}%
\special{pa 2287 7114}%
\special{fp}%
\special{pa 2250 7123}%
\special{pa 2242 7125}%
\special{fp}%
\special{pa 2205 7131}%
\special{pa 2197 7133}%
\special{fp}%
\special{pa 2160 7139}%
\special{pa 2152 7141}%
\special{fp}%
\special{pa 2115 7143}%
\special{pa 2106 7144}%
\special{fp}%
\special{pa 2069 7145}%
\special{pa 2061 7145}%
\special{fp}%
\special{pa 2023 7141}%
\special{pa 2015 7139}%
\special{fp}%
\special{pa 1980 7128}%
\special{pa 1972 7125}%
\special{fp}%
\special{pa 1944 7100}%
\special{pa 1939 7094}%
\special{fp}%
% SPLINE 2 2 3 0 Black White  
% 42 2850 5834 2866 5843 2880 5856 2893 5874 2903 5896 2911 5924 2917 5955 2921 5990 2923 6029 2923 6072 2920 6118 2915 6166 2908 6218 2899 6272 2888 6329 2875 6388 2860 6446 2843 6506 2825 6568 2805 6630 2783 6691 2760 6751 2737 6811 2712 6870 2685 6927 2659 6982 2632 7035 2604 7086 2576 7133 2548 7178 2520 7217 2493 7255 2466 7288 2439 7317 2414 7341 2389 7361 2366 7377 2344 7387 2323 7393 2303 7394 2285 7391 2285 7391
% 
\special{pn 8}%
\special{pn 8}%
\special{pa 2850 5834}%
\special{pa 2857 5839}%
\special{fp}%
\special{pa 2884 5863}%
\special{pa 2889 5870}%
\special{fp}%
\special{pa 2905 5904}%
\special{pa 2908 5912}%
\special{fp}%
\special{pa 2916 5948}%
\special{pa 2917 5957}%
\special{fp}%
\special{pa 2921 5994}%
\special{pa 2922 6002}%
\special{fp}%
\special{pa 2923 6040}%
\special{pa 2923 6048}%
\special{fp}%
\special{pa 2922 6086}%
\special{pa 2922 6094}%
\special{fp}%
\special{pa 2919 6131}%
\special{pa 2918 6139}%
\special{fp}%
\special{pa 2913 6177}%
\special{pa 2912 6185}%
\special{fp}%
\special{pa 2907 6222}%
\special{pa 2906 6230}%
\special{fp}%
\special{pa 2900 6267}%
\special{pa 2898 6275}%
\special{fp}%
\special{pa 2891 6312}%
\special{pa 2890 6320}%
\special{fp}%
\special{pa 2882 6357}%
\special{pa 2880 6365}%
\special{fp}%
\special{pa 2871 6402}%
\special{pa 2870 6410}%
\special{fp}%
\special{pa 2859 6446}%
\special{pa 2857 6454}%
\special{fp}%
\special{pa 2848 6490}%
\special{pa 2845 6498}%
\special{fp}%
\special{pa 2835 6534}%
\special{pa 2832 6542}%
\special{fp}%
\special{pa 2822 6578}%
\special{pa 2819 6586}%
\special{fp}%
\special{pa 2808 6622}%
\special{pa 2805 6630}%
\special{fp}%
\special{pa 2792 6665}%
\special{pa 2790 6673}%
\special{fp}%
\special{pa 2777 6708}%
\special{pa 2774 6716}%
\special{fp}%
\special{pa 2760 6751}%
\special{pa 2757 6759}%
\special{fp}%
\special{pa 2744 6794}%
\special{pa 2741 6801}%
\special{fp}%
\special{pa 2727 6836}%
\special{pa 2723 6844}%
\special{fp}%
\special{pa 2708 6878}%
\special{pa 2705 6886}%
\special{fp}%
\special{pa 2689 6920}%
\special{pa 2685 6927}%
\special{fp}%
\special{pa 2669 6961}%
\special{pa 2665 6969}%
\special{fp}%
\special{pa 2649 7002}%
\special{pa 2645 7009}%
\special{fp}%
\special{pa 2627 7043}%
\special{pa 2624 7050}%
\special{fp}%
\special{pa 2605 7083}%
\special{pa 2601 7090}%
\special{fp}%
\special{pa 2582 7123}%
\special{pa 2578 7130}%
\special{fp}%
\special{pa 2558 7162}%
\special{pa 2554 7169}%
\special{fp}%
\special{pa 2533 7200}%
\special{pa 2528 7206}%
\special{fp}%
\special{pa 2506 7237}%
\special{pa 2501 7243}%
\special{fp}%
\special{pa 2478 7273}%
\special{pa 2473 7280}%
\special{fp}%
\special{pa 2448 7308}%
\special{pa 2442 7313}%
\special{fp}%
\special{pa 2415 7340}%
\special{pa 2409 7345}%
\special{fp}%
\special{pa 2379 7368}%
\special{pa 2372 7372}%
\special{fp}%
\special{pa 2338 7388}%
\special{pa 2330 7390}%
\special{fp}%
\special{pa 2293 7392}%
\special{pa 2285 7391}%
\special{fp}%
% SPLINE 2 0 3 0 Black White  
% 42 1953 7108 1946 7090 1945 7071 1948 7049 1956 7026 1968 7000 1984 6973 2005 6945 2030 6915 2059 6884 2093 6851 2128 6818 2168 6784 2212 6751 2258 6716 2307 6682 2358 6648 2410 6615 2465 6582 2522 6550 2579 6519 2637 6490 2694 6461 2753 6435 2811 6410 2867 6388 2923 6367 2977 6348 3030 6332 3080 6318 3128 6307 3174 6298 3216 6292 3255 6288 3290 6287 3322 6289 3350 6294 3373 6301 3392 6310 3407 6323 3419 6337 3419 6337
% 
\special{pn 8}%
\special{pa 1953 7108}%
\special{pa 1945 7077}%
\special{pa 1949 7046}%
\special{pa 1960 7016}%
\special{pa 1975 6987}%
\special{pa 1993 6961}%
\special{pa 2033 6911}%
\special{pa 2055 6888}%
\special{pa 2078 6865}%
\special{pa 2124 6821}%
\special{pa 2148 6800}%
\special{pa 2173 6780}%
\special{pa 2199 6761}%
\special{pa 2224 6742}%
\special{pa 2250 6722}%
\special{pa 2276 6703}%
\special{pa 2302 6685}%
\special{pa 2329 6667}%
\special{pa 2355 6650}%
\special{pa 2382 6632}%
\special{pa 2409 6615}%
\special{pa 2437 6599}%
\special{pa 2464 6582}%
\special{pa 2492 6567}%
\special{pa 2520 6551}%
\special{pa 2548 6536}%
\special{pa 2576 6520}%
\special{pa 2605 6506}%
\special{pa 2633 6492}%
\special{pa 2662 6477}%
\special{pa 2690 6463}%
\special{pa 2720 6449}%
\special{pa 2749 6437}%
\special{pa 2778 6424}%
\special{pa 2808 6411}%
\special{pa 2837 6399}%
\special{pa 2897 6377}%
\special{pa 2927 6365}%
\special{pa 2957 6355}%
\special{pa 2988 6345}%
\special{pa 3018 6335}%
\special{pa 3049 6326}%
\special{pa 3080 6318}%
\special{pa 3142 6304}%
\special{pa 3174 6298}%
\special{pa 3205 6293}%
\special{pa 3269 6287}%
\special{pa 3301 6287}%
\special{pa 3333 6291}%
\special{pa 3364 6298}%
\special{pa 3393 6311}%
\special{pa 3416 6333}%
\special{pa 3419 6337}%
\special{fp}%
% SPLINE 2 0 3 0 Black White  
% 42 2064 6022 2091 6001 2121 5985 2154 5972 2191 5963 2229 5958 2270 5959 2314 5962 2359 5971 2406 5983 2454 5999 2504 6020 2554 6043 2605 6071 2655 6102 2705 6137 2756 6175 2804 6216 2852 6259 2899 6305 2944 6352 2987 6401 3028 6453 3066 6504 3102 6557 3134 6611 3163 6665 3190 6718 3212 6771 3231 6823 3246 6875 3258 6924 3265 6973 3269 7018 3268 7062 3264 7103 3256 7140 3244 7176 3227 7207 3208 7236 3184 7261 3184 7261
% 
\special{pn 8}%
\special{pa 2064 6022}%
\special{pa 2089 6002}%
\special{pa 2117 5987}%
\special{pa 2147 5974}%
\special{pa 2177 5966}%
\special{pa 2209 5960}%
\special{pa 2241 5958}%
\special{pa 2273 5959}%
\special{pa 2305 5961}%
\special{pa 2336 5966}%
\special{pa 2368 5973}%
\special{pa 2399 5981}%
\special{pa 2429 5990}%
\special{pa 2459 6001}%
\special{pa 2489 6013}%
\special{pa 2518 6026}%
\special{pa 2576 6054}%
\special{pa 2604 6070}%
\special{pa 2658 6104}%
\special{pa 2684 6122}%
\special{pa 2736 6160}%
\special{pa 2761 6179}%
\special{pa 2786 6200}%
\special{pa 2834 6242}%
\special{pa 2880 6286}%
\special{pa 2903 6309}%
\special{pa 2947 6355}%
\special{pa 2968 6379}%
\special{pa 2989 6404}%
\special{pa 3029 6454}%
\special{pa 3048 6479}%
\special{pa 3067 6505}%
\special{pa 3085 6531}%
\special{pa 3103 6558}%
\special{pa 3119 6586}%
\special{pa 3135 6613}%
\special{pa 3151 6641}%
\special{pa 3166 6670}%
\special{pa 3180 6698}%
\special{pa 3194 6727}%
\special{pa 3206 6757}%
\special{pa 3218 6786}%
\special{pa 3229 6816}%
\special{pa 3247 6878}%
\special{pa 3255 6909}%
\special{pa 3261 6940}%
\special{pa 3265 6972}%
\special{pa 3268 7004}%
\special{pa 3269 7036}%
\special{pa 3268 7068}%
\special{pa 3264 7100}%
\special{pa 3258 7131}%
\special{pa 3250 7162}%
\special{pa 3236 7191}%
\special{pa 3220 7219}%
\special{pa 3201 7244}%
\special{pa 3184 7261}%
\special{fp}%
% SPLINE 2 2 3 0 Black White  
% 42 3198 7254 3163 7281 3126 7306 3085 7325 3042 7341 2996 7352 2948 7359 2898 7361 2846 7360 2793 7353 2739 7343 2684 7327 2630 7308 2574 7285 2520 7257 2466 7225 2413 7190 2361 7152 2312 7111 2264 7067 2218 7019 2175 6970 2135 6918 2098 6865 2065 6810 2035 6755 2008 6699 1985 6641 1967 6584 1952 6528 1941 6472 1935 6418 1933 6364 1936 6312 1943 6263 1953 6215 1969 6170 1988 6129 2012 6090 2039 6055 2070 6023 2070 6023
% 
\special{pn 8}%
\special{pn 8}%
\special{pa 3198 7254}%
\special{pa 3192 7259}%
\special{fp}%
\special{pa 3162 7282}%
\special{pa 3156 7286}%
\special{fp}%
\special{pa 3125 7306}%
\special{pa 3118 7310}%
\special{fp}%
\special{pa 3084 7326}%
\special{pa 3077 7329}%
\special{fp}%
\special{pa 3042 7341}%
\special{pa 3034 7343}%
\special{fp}%
\special{pa 2998 7351}%
\special{pa 2990 7353}%
\special{fp}%
\special{pa 2953 7358}%
\special{pa 2945 7359}%
\special{fp}%
\special{pa 2908 7361}%
\special{pa 2900 7361}%
\special{fp}%
\special{pa 2863 7360}%
\special{pa 2855 7360}%
\special{fp}%
\special{pa 2818 7356}%
\special{pa 2810 7355}%
\special{fp}%
\special{pa 2774 7350}%
\special{pa 2766 7348}%
\special{fp}%
\special{pa 2730 7340}%
\special{pa 2722 7338}%
\special{fp}%
\special{pa 2686 7328}%
\special{pa 2679 7325}%
\special{fp}%
\special{pa 2644 7313}%
\special{pa 2636 7310}%
\special{fp}%
\special{pa 2602 7297}%
\special{pa 2594 7294}%
\special{fp}%
\special{pa 2561 7278}%
\special{pa 2554 7275}%
\special{fp}%
\special{pa 2521 7257}%
\special{pa 2514 7253}%
\special{fp}%
\special{pa 2482 7235}%
\special{pa 2475 7231}%
\special{fp}%
\special{pa 2444 7211}%
\special{pa 2437 7206}%
\special{fp}%
\special{pa 2406 7185}%
\special{pa 2400 7181}%
\special{fp}%
\special{pa 2370 7159}%
\special{pa 2364 7154}%
\special{fp}%
\special{pa 2335 7130}%
\special{pa 2329 7125}%
\special{fp}%
\special{pa 2301 7101}%
\special{pa 2295 7096}%
\special{fp}%
\special{pa 2268 7070}%
\special{pa 2262 7065}%
\special{fp}%
\special{pa 2236 7038}%
\special{pa 2231 7033}%
\special{fp}%
\special{pa 2206 7005}%
\special{pa 2200 6999}%
\special{fp}%
\special{pa 2176 6971}%
\special{pa 2171 6965}%
\special{fp}%
\special{pa 2149 6935}%
\special{pa 2144 6929}%
\special{fp}%
\special{pa 2122 6899}%
\special{pa 2117 6893}%
\special{fp}%
\special{pa 2097 6862}%
\special{pa 2092 6855}%
\special{fp}%
\special{pa 2073 6824}%
\special{pa 2069 6817}%
\special{fp}%
\special{pa 2051 6784}%
\special{pa 2047 6777}%
\special{fp}%
\special{pa 2030 6744}%
\special{pa 2026 6737}%
\special{fp}%
\special{pa 2010 6704}%
\special{pa 2007 6697}%
\special{fp}%
\special{pa 1993 6662}%
\special{pa 1990 6655}%
\special{fp}%
\special{pa 1978 6620}%
\special{pa 1976 6612}%
\special{fp}%
\special{pa 1965 6577}%
\special{pa 1963 6569}%
\special{fp}%
\special{pa 1953 6533}%
\special{pa 1951 6525}%
\special{fp}%
\special{pa 1944 6489}%
\special{pa 1942 6481}%
\special{fp}%
\special{pa 1937 6444}%
\special{pa 1936 6436}%
\special{fp}%
\special{pa 1934 6399}%
\special{pa 1933 6391}%
\special{fp}%
\special{pa 1933 6354}%
\special{pa 1933 6346}%
\special{fp}%
\special{pa 1937 6309}%
\special{pa 1938 6302}%
\special{fp}%
\special{pa 1943 6265}%
\special{pa 1944 6257}%
\special{fp}%
\special{pa 1952 6221}%
\special{pa 1954 6213}%
\special{fp}%
\special{pa 1965 6178}%
\special{pa 1968 6170}%
\special{fp}%
\special{pa 1984 6137}%
\special{pa 1988 6130}%
\special{fp}%
\special{pa 2007 6098}%
\special{pa 2011 6091}%
\special{fp}%
\special{pa 2034 6062}%
\special{pa 2039 6056}%
\special{fp}%
\special{pa 2064 6029}%
\special{pa 2070 6023}%
\special{fp}%
% STR 2 0 3 0 Black White  
% 4 2676 5994 2676 6064 2 0 0 0
% A
\put(26.7600,-60.6400){\makebox(0,0)[lb]{A}}%
% STR 2 0 3 0 Black White  
% 4 2948 6231 2948 6301 2 0 0 0
% B
\put(29.4800,-63.0100){\makebox(0,0)[lb]{B}}%
% STR 2 0 3 0 Black White  
% 4 2266 6579 2266 6648 2 0 0 0
% C
\put(22.6600,-66.4800){\makebox(0,0)[lb]{C}}%
% STR 2 0 3 0 Black White  
% 4 2711 6948 2711 7017 2 0 0 0
% C*
\put(27.1100,-70.1700){\makebox(0,0)[lb]{C*}}%
% STR 2 0 3 0 Black White  
% 4 2516 7233 2516 7302 2 0 0 0
% A*
\put(25.1600,-73.0200){\makebox(0,0)[lb]{A*}}%
% STR 2 0 3 0 Black White  
% 4 2308 6989 2308 7059 2 0 0 0
% B*
\put(23.0800,-70.5900){\makebox(0,0)[lb]{B*}}%
\end{picture}}%

\end{center}
\begin{proof}
まず,角度$\alpha$の月形$L$の面積が$2\alpha$であることを示す.$\varphi$を円柱投影とすると,$\varphi(L)$は,円柱の側面を展開したときに幅$\alpha$,高さ2の長方形になる.よってその面積は$2\alpha$となる.

次に,球面三角形の面積公式を示す.図のように点をとる.また,$\Delta=[ABC]$とおく.点$A$,$A^*$は互いに極の位置にあるから,図の斜線部は月形となる.点$B,C$についても同様.このとき
\[ 2\angle A+2\angle A+2\angle B+2\angle B+2\angle C+2\angle C=4\pi+4\Delta. \]
よって
\[ \Delta=\angle A+\angle B+\angle C-\pi \]
となる.
\end{proof}

\begin{corollary}球面上の 凸$n$角形の面積は,$n$個の角度を$\alpha_i(1\le i\le n)$として
\[ \sum_{i=1}^n \alpha_i -(n-2)\pi \]
で表される.\end{corollary}
\begin{proof}任意の2頂点を結ぶ線分が常に領域の内部にあることに注意すればよい.\end{proof}

\begin{corollary}球面上の凸$n$角形の内角の和は$(n-2)\pi$より大きい.\end{corollary}

\section{点の位置関係と面積}
前章では,球面三角形の面積が角度によって定まることを見た.この章では,三角形の等積変形の原理を示すレクセルの定理を紹介する.
\begin{definition}
\begin{enumerate}[label=(\arabic*)]
\item 球面三角形$ABC$に対して,点$A$,$B$,$C$を通る$\mathbb{R}^3$内の円で切ったときの,三角形$ABC$を含む側を三角形$ABC$の外接キャップといい,$cap(ABC)$で表す.
\item 点$A,B$を両端とし点$C$を通る円弧を$\widehat{ACB}$で表し,半円より大きいとき優弧,小さいとき劣弧という.
\end{enumerate}
\end{definition}

\begin{proposition}[球面上の円周角の定理]

三角形$ABC$に対して,$X\in\widehat{ACB}$ならば
\[ \angle AXB-(\angle XAB+\angle XBA)=一定 \]
となる.
\end{proposition}
\begin{proof}
図を参照.
\end{proof}
\begin{center}
%WinTpicVersion4.32a
{\unitlength 0.1in%
\begin{picture}(14.4000,14.4700)(10.6000,-62.8300)%
% CIRCLE 2 0 3 0 Black White  
% 4 1858 5641 2334 6072 2334 6072 2334 6072
% 
\special{pn 8}%
\special{ar 1858 5641 642 642 0.0000000 6.2831853}%
% LINE 2 0 3 0 Black White  
% 12 1851 5641 2096 6240 2096 6240 1977 5011 1977 5011 1221 5711 1221 5711 1851 5641 1851 5641 1973 5008 1221 5708 2092 6229
% 
\special{pn 8}%
\special{pa 1851 5641}%
\special{pa 2096 6240}%
\special{fp}%
\special{pa 2096 6240}%
\special{pa 1977 5011}%
\special{fp}%
\special{pa 1977 5011}%
\special{pa 1221 5711}%
\special{fp}%
\special{pa 1221 5711}%
\special{pa 1851 5641}%
\special{fp}%
\special{pa 1851 5641}%
\special{pa 1973 5008}%
\special{fp}%
\special{pa 1221 5708}%
\special{pa 2092 6229}%
\special{fp}%
% CIRCLE 2 0 0 0 Black Black  
% 4 1333 5659 1343 5683 1473 5634 1438 5648
% 
\special{pn 8}%
\special{ar 1333 5659 26 26 6.1788042 6.1064765}%
% CIRCLE 2 0 3 0 Black White  
% 4 1900 5137 1882 5116 1882 5116 1882 5116
% 
\special{pn 8}%
\special{ar 1900 5137 28 28 0.0000000 6.2831853}%
% LINE 2 0 3 0 Black Black  
% 8 1403 5729 1350 5771 1350 5729 1396 5767 1959 6100 2015 6149 2005 6100 1966 6145
% 
\special{pn 8}%
\special{pa 1403 5729}%
\special{pa 1350 5771}%
\special{fp}%
\special{pa 1350 5729}%
\special{pa 1396 5767}%
\special{fp}%
\special{pa 1959 6100}%
\special{pa 2015 6149}%
\special{fp}%
\special{pa 2005 6100}%
\special{pa 1966 6145}%
\special{fp}%
% STR 2 0 3 0 Black Black  
% 4 1952 4931 1952 4966 2 0 0 0
% B
\put(19.5200,-49.6600){\makebox(0,0)[lb]{B}}%
% STR 2 0 3 0 Black White  
% 4 1382 5095 1382 5130 2 0 0 0
% C
\put(13.8200,-51.3000){\makebox(0,0)[lb]{C}}%
% STR 2 0 3 0 Black White  
% 4 1805 5561 1805 5596 2 0 0 0
% P
\put(18.0500,-55.9600){\makebox(0,0)[lb]{P}}%
% DOT 2 0 3 0 Black White  
% 1 1438 5151
% 
\special{pn 4}%
\special{sh 1}%
\special{ar 1438 5151 8 8 0 6.2831853}%
% STR 2 0 3 0 Black Black  
% 4 2120 6280 2120 6380 2 0 0 0
% A
\put(21.2000,-63.8000){\makebox(0,0)[lb]{A}}%
% STR 2 0 3 0 Black Black  
% 4 1060 5690 1060 5790 2 0 0 0
% X
\put(10.6000,-57.9000){\makebox(0,0)[lb]{X}}%
\end{picture}}%

\end{center}
\begin{corollary}
\[ \text{$\widehat{ACB}$が半円}\Leftrightarrow \angle C=\angle A+\angle B, \]
\[ \text{$\widehat{ACB}$が優弧}\Leftrightarrow \angle C<\angle A+\angle B, \]
\[ \text{$\widehat{ACB}$が劣弧}\Leftrightarrow \angle C>\angle A+\angle B. \]
\end{corollary}


\begin{theorem}[レクセルの定理]球面三角形$ABC$に対して,$X$を大円$ABA^*$の$C$側にある点とする.このとき
\[ X\in \widehat{A^*CB^*}\Leftrightarrow [ABX]=[ABC] \]
\[ \text{$X$が$\capp(A^*CB^*)$の外側}\Leftrightarrow [ABX]<[ABC], \]
\[ \text{$X$が$\capp(A^*CB^*)$の内側}\Leftrightarrow [ABC]>[ABC] \]
が成立する.ただし,$\capp(A^*CB^*)$の内側とは,キャップによって分けられる二つの領域のうち三角形$A^*CB^*$を含む方のこととする.\end{theorem}
\begin{center}
%WinTpicVersion4.32a
{\unitlength 0.1in%
\begin{picture}(17.4300,18.2400)(18.9700,-36.4000)%
% CIRCLE 2 0 3 0 Black White  
% 4 2800 2800 3640 2800 3640 2800 3640 2800
% 
\special{pn 8}%
\special{ar 2800 2800 840 840 0.0000000 6.2831853}%
% ELLIPSE 2 0 3 0 Black White  
% 4 2674 2401 3332 1883 1589 2576 3297 1918
% 
\special{pn 8}%
\special{ar 2674 2401 658 518 5.5050627 2.9397700}%
% SPLINE 2 0 3 0 Black White  
% 42 3535 3213 3529 3219 3518 3223 3503 3225 3484 3223 3459 3220 3431 3214 3399 3206 3363 3195 3323 3183 3280 3168 3234 3151 3185 3132 3133 3112 3079 3090 3024 3066 2967 3041 2908 3014 2850 2986 2790 2958 2731 2928 2671 2899 2613 2868 2555 2837 2498 2807 2444 2776 2392 2746 2342 2717 2295 2687 2250 2659 2209 2633 2171 2607 2137 2583 2107 2560 2081 2540 2059 2520 2043 2503 2029 2488 2022 2475 2019 2464 2020 2456 2020 2456
% 
\special{pn 8}%
\special{pa 3535 3213}%
\special{pa 3506 3225}%
\special{pa 3474 3222}%
\special{pa 3443 3217}%
\special{pa 3381 3201}%
\special{pa 3350 3191}%
\special{pa 3320 3182}%
\special{pa 3290 3171}%
\special{pa 3259 3161}%
\special{pa 3230 3149}%
\special{pa 3200 3138}%
\special{pa 3170 3126}%
\special{pa 3140 3115}%
\special{pa 3110 3103}%
\special{pa 3081 3091}%
\special{pa 3051 3078}%
\special{pa 2993 3052}%
\special{pa 2963 3039}%
\special{pa 2905 3013}%
\special{pa 2876 2999}%
\special{pa 2848 2985}%
\special{pa 2819 2971}%
\special{pa 2790 2958}%
\special{pa 2761 2943}%
\special{pa 2733 2929}%
\special{pa 2675 2901}%
\special{pa 2647 2886}%
\special{pa 2618 2871}%
\special{pa 2534 2826}%
\special{pa 2505 2811}%
\special{pa 2477 2795}%
\special{pa 2450 2779}%
\special{pa 2366 2731}%
\special{pa 2339 2715}%
\special{pa 2204 2630}%
\special{pa 2177 2611}%
\special{pa 2151 2593}%
\special{pa 2125 2574}%
\special{pa 2100 2555}%
\special{pa 2075 2535}%
\special{pa 2052 2513}%
\special{pa 2030 2489}%
\special{pa 2019 2460}%
\special{pa 2020 2456}%
\special{fp}%
% SPLINE 2 0 3 0 Black White  
% 42 3136 2016 3142 2023 3144 2034 3145 2048 3142 2068 3137 2092 3129 2120 3118 2151 3105 2186 3090 2225 3072 2267 3051 2311 3028 2359 3004 2409 2978 2461 2950 2514 2920 2569 2888 2625 2856 2682 2823 2739 2789 2796 2755 2852 2720 2908 2685 2963 2650 3018 2615 3069 2581 3119 2547 3167 2515 3212 2484 3254 2454 3293 2425 3328 2398 3361 2373 3389 2351 3412 2330 3433 2311 3448 2295 3460 2281 3466 2271 3469 2262 3467 2262 3467
% 
\special{pn 8}%
\special{pa 3136 2016}%
\special{pa 3145 2046}%
\special{pa 3140 2077}%
\special{pa 3133 2108}%
\special{pa 3123 2139}%
\special{pa 3111 2169}%
\special{pa 3100 2199}%
\special{pa 3089 2228}%
\special{pa 3076 2258}%
\special{pa 3063 2287}%
\special{pa 3035 2345}%
\special{pa 3021 2373}%
\special{pa 2979 2460}%
\special{pa 2934 2544}%
\special{pa 2886 2628}%
\special{pa 2871 2656}%
\special{pa 2855 2684}%
\special{pa 2839 2711}%
\special{pa 2807 2767}%
\special{pa 2790 2794}%
\special{pa 2774 2821}%
\special{pa 2757 2849}%
\special{pa 2655 3011}%
\special{pa 2637 3038}%
\special{pa 2618 3064}%
\special{pa 2600 3090}%
\special{pa 2582 3117}%
\special{pa 2564 3143}%
\special{pa 2545 3169}%
\special{pa 2527 3195}%
\special{pa 2470 3273}%
\special{pa 2450 3298}%
\special{pa 2430 3322}%
\special{pa 2410 3347}%
\special{pa 2389 3372}%
\special{pa 2345 3418}%
\special{pa 2322 3440}%
\special{pa 2296 3459}%
\special{pa 2266 3468}%
\special{pa 2262 3467}%
\special{fp}%
% SPLINE 2 2 3 0 Black White  
% 42 2016 2485 2022 2479 2033 2475 2048 2473 2067 2475 2092 2477 2120 2483 2152 2491 2188 2501 2228 2514 2271 2528 2318 2545 2367 2563 2419 2583 2472 2605 2528 2629 2585 2653 2644 2680 2703 2707 2763 2735 2822 2764 2882 2794 2941 2824 2999 2854 3055 2884 3110 2914 3163 2943 3213 2973 3261 3002 3305 3030 3347 3056 3384 3081 3419 3105 3449 3128 3475 3149 3496 3167 3513 3185 3527 3200 3534 3213 3538 3223 3536 3232 3536 3232
% 
\special{pn 8}%
\special{pn 8}%
\special{pa 2016 2485}%
\special{pa 2023 2482}%
\special{fp}%
\special{pa 2059 2474}%
\special{pa 2067 2475}%
\special{fp}%
\special{pa 2104 2479}%
\special{pa 2112 2481}%
\special{fp}%
\special{pa 2148 2490}%
\special{pa 2156 2492}%
\special{fp}%
\special{pa 2192 2502}%
\special{pa 2199 2505}%
\special{fp}%
\special{pa 2235 2516}%
\special{pa 2243 2519}%
\special{fp}%
\special{pa 2278 2530}%
\special{pa 2286 2533}%
\special{fp}%
\special{pa 2321 2546}%
\special{pa 2328 2548}%
\special{fp}%
\special{pa 2363 2561}%
\special{pa 2371 2564}%
\special{fp}%
\special{pa 2406 2578}%
\special{pa 2413 2580}%
\special{fp}%
\special{pa 2448 2595}%
\special{pa 2455 2598}%
\special{fp}%
\special{pa 2489 2613}%
\special{pa 2497 2616}%
\special{fp}%
\special{pa 2531 2630}%
\special{pa 2538 2633}%
\special{fp}%
\special{pa 2573 2648}%
\special{pa 2580 2651}%
\special{fp}%
\special{pa 2614 2666}%
\special{pa 2622 2670}%
\special{fp}%
\special{pa 2656 2685}%
\special{pa 2663 2689}%
\special{fp}%
\special{pa 2697 2704}%
\special{pa 2704 2708}%
\special{fp}%
\special{pa 2738 2723}%
\special{pa 2745 2726}%
\special{fp}%
\special{pa 2779 2743}%
\special{pa 2786 2746}%
\special{fp}%
\special{pa 2820 2762}%
\special{pa 2827 2766}%
\special{fp}%
\special{pa 2860 2783}%
\special{pa 2867 2787}%
\special{fp}%
\special{pa 2901 2803}%
\special{pa 2908 2807}%
\special{fp}%
\special{pa 2941 2824}%
\special{pa 2948 2828}%
\special{fp}%
\special{pa 2981 2845}%
\special{pa 2989 2848}%
\special{fp}%
\special{pa 3021 2866}%
\special{pa 3029 2870}%
\special{fp}%
\special{pa 3061 2888}%
\special{pa 3068 2892}%
\special{fp}%
\special{pa 3101 2909}%
\special{pa 3108 2913}%
\special{fp}%
\special{pa 3141 2931}%
\special{pa 3148 2935}%
\special{fp}%
\special{pa 3180 2953}%
\special{pa 3187 2958}%
\special{fp}%
\special{pa 3219 2977}%
\special{pa 3226 2981}%
\special{fp}%
\special{pa 3258 3000}%
\special{pa 3265 3005}%
\special{fp}%
\special{pa 3296 3025}%
\special{pa 3303 3029}%
\special{fp}%
\special{pa 3335 3049}%
\special{pa 3341 3053}%
\special{fp}%
\special{pa 3373 3073}%
\special{pa 3379 3078}%
\special{fp}%
\special{pa 3410 3099}%
\special{pa 3417 3104}%
\special{fp}%
\special{pa 3447 3126}%
\special{pa 3453 3131}%
\special{fp}%
\special{pa 3482 3155}%
\special{pa 3488 3160}%
\special{fp}%
\special{pa 3514 3186}%
\special{pa 3520 3192}%
\special{fp}%
\special{pa 3537 3225}%
\special{pa 3536 3232}%
\special{fp}%
% SPLINE 2 2 3 0 Black White  
% 42 2254 3465 2248 3458 2246 3447 2245 3433 2248 3413 2253 3389 2261 3361 2272 3330 2285 3295 2300 3256 2318 3214 2339 3170 2362 3122 2386 3072 2412 3020 2440 2967 2470 2912 2502 2856 2534 2799 2567 2742 2601 2685 2636 2629 2671 2573 2706 2518 2741 2463 2775 2412 2809 2362 2843 2314 2875 2269 2906 2227 2936 2188 2965 2153 2992 2120 3017 2092 3039 2069 3060 2048 3079 2033 3095 2021 3109 2015 3119 2012 3128 2014 3128 2014
% 
\special{pn 8}%
\special{pn 8}%
\special{pa 2254 3465}%
\special{pa 2252 3457}%
\special{fp}%
\special{pa 2247 3421}%
\special{pa 2249 3413}%
\special{fp}%
\special{pa 2256 3377}%
\special{pa 2258 3369}%
\special{fp}%
\special{pa 2270 3334}%
\special{pa 2273 3326}%
\special{fp}%
\special{pa 2287 3291}%
\special{pa 2289 3284}%
\special{fp}%
\special{pa 2303 3249}%
\special{pa 2306 3242}%
\special{fp}%
\special{pa 2321 3208}%
\special{pa 2324 3200}%
\special{fp}%
\special{pa 2340 3167}%
\special{pa 2344 3159}%
\special{fp}%
\special{pa 2360 3126}%
\special{pa 2364 3119}%
\special{fp}%
\special{pa 2380 3085}%
\special{pa 2384 3078}%
\special{fp}%
\special{pa 2400 3044}%
\special{pa 2403 3037}%
\special{fp}%
\special{pa 2420 3004}%
\special{pa 2424 2997}%
\special{fp}%
\special{pa 2442 2964}%
\special{pa 2445 2957}%
\special{fp}%
\special{pa 2463 2924}%
\special{pa 2467 2917}%
\special{fp}%
\special{pa 2486 2885}%
\special{pa 2490 2878}%
\special{fp}%
\special{pa 2508 2845}%
\special{pa 2512 2838}%
\special{fp}%
\special{pa 2530 2806}%
\special{pa 2534 2799}%
\special{fp}%
\special{pa 2553 2767}%
\special{pa 2557 2760}%
\special{fp}%
\special{pa 2575 2727}%
\special{pa 2579 2720}%
\special{fp}%
\special{pa 2599 2689}%
\special{pa 2603 2682}%
\special{fp}%
\special{pa 2623 2650}%
\special{pa 2627 2643}%
\special{fp}%
\special{pa 2647 2612}%
\special{pa 2652 2605}%
\special{fp}%
\special{pa 2671 2573}%
\special{pa 2675 2566}%
\special{fp}%
\special{pa 2695 2535}%
\special{pa 2699 2528}%
\special{fp}%
\special{pa 2719 2497}%
\special{pa 2724 2490}%
\special{fp}%
\special{pa 2744 2458}%
\special{pa 2749 2452}%
\special{fp}%
\special{pa 2769 2421}%
\special{pa 2774 2414}%
\special{fp}%
\special{pa 2795 2383}%
\special{pa 2799 2377}%
\special{fp}%
\special{pa 2820 2346}%
\special{pa 2825 2339}%
\special{fp}%
\special{pa 2847 2309}%
\special{pa 2851 2303}%
\special{fp}%
\special{pa 2873 2272}%
\special{pa 2878 2266}%
\special{fp}%
\special{pa 2900 2236}%
\special{pa 2905 2229}%
\special{fp}%
\special{pa 2927 2199}%
\special{pa 2932 2193}%
\special{fp}%
\special{pa 2956 2164}%
\special{pa 2961 2158}%
\special{fp}%
\special{pa 2985 2129}%
\special{pa 2990 2123}%
\special{fp}%
\special{pa 3014 2095}%
\special{pa 3020 2089}%
\special{fp}%
\special{pa 3046 2062}%
\special{pa 3051 2057}%
\special{fp}%
\special{pa 3080 2033}%
\special{pa 3086 2028}%
\special{fp}%
\special{pa 3120 2014}%
\special{pa 3128 2014}%
\special{fp}%
% STR 2 0 3 0 Black White  
% 4 3150 1876 3150 1946 2 0 0 0
% A*
\put(31.5000,-19.4600){\makebox(0,0)[lb]{A*}}%
% STR 2 0 3 0 Black White  
% 4 1897 2366 1897 2436 2 0 0 0
% B*
\put(18.9700,-24.3600){\makebox(0,0)[lb]{B*}}%
% STR 2 0 3 0 Black White  
% 4 2184 3465 2184 3535 2 0 0 0
% A
\put(21.8400,-35.3500){\makebox(0,0)[lb]{A}}%
% STR 2 0 3 0 Black White  
% 4 3570 3220 3570 3290 2 0 0 0
% B
\put(35.7000,-32.9000){\makebox(0,0)[lb]{B}}%
% DOT 2 2 3 0 Black White  
% 1 3192 2709
% 
\special{pn 4}%
\special{sh 1}%
\special{ar 3192 2709 8 8 0 6.2831853}%
% STR 2 0 3 0 Black White  
% 4 3220 2709 3220 2779 2 0 0 0
% C
\put(32.2000,-27.7900){\makebox(0,0)[lb]{C}}%
% STR 2 0 3 0 Black White  
% 4 2702 2947 2702 3017 2 0 0 0
% X
\put(27.0200,-30.1700){\makebox(0,0)[lb]{X}}%
% CIRCLE 2 0 3 0 Black White  
% 4 2016 2478 2128 2569 2079 2625 2079 2373
% 
\special{pn 8}%
\special{ar 2016 2478 144 144 5.2528085 1.1659045}%
% STR 2 0 3 0 Black White  
% 4 2198 2380 2198 2450 2 0 0 0
% ��*
\put(21.9800,-24.5000){\makebox(0,0)[lb]{��*}}%
% CIRCLE 2 0 3 0 Black White  
% 4 3136 2023 3038 2163 2989 1981 3094 2191
% 
\special{pn 8}%
\special{ar 3136 2023 171 171 1.8157750 3.4198923}%
% STR 2 0 3 0 Black White  
% 4 2898 2037 2898 2107 2 0 0 0
% ��*
\put(28.9800,-21.0700){\makebox(0,0)[lb]{��*}}%
% CIRCLE 2 0 3 0 Black White  
% 4 2261 3458 2429 3514 2394 3542 2408 3367
% 
\special{pn 8}%
\special{ar 2261 3458 177 177 5.7288778 0.5633163}%
% STR 2 0 3 0 Black White  
% 4 2457 3416 2457 3486 2 0 0 0
% ��
\put(24.5700,-34.8600){\makebox(0,0)[lb]{��}}%
% CIRCLE 2 0 3 0 Black White  
% 4 3528 3213 3451 3367 3353 3199 3416 3374
% 
\special{pn 8}%
\special{ar 3528 3213 172 172 2.1785983 3.2214226}%
% STR 2 0 3 0 Black White  
% 4 3304 3241 3304 3311 2 0 0 0
% ��
\put(33.0400,-33.1100){\makebox(0,0)[lb]{��}}%
% STR 2 0 3 0 Black White  
% 4 2646 2695 2646 2765 2 0 0 0
% ��
\put(26.4600,-27.6500){\makebox(0,0)[lb]{��}}%
% CIRCLE 2 0 3 0 Black White  
% 4 2716 2905 2758 2772 2786 2779 2590 2842
% 
\special{pn 8}%
\special{ar 2716 2905 139 139 3.6052403 5.2194875}%
\end{picture}}%
\input{sphere4.tex}
\end{center}


\begin{proof}
左図を参照して,$X\in \widehat{A^*CB^*}$のとき円周角の定理より$\theta-\alpha^*-\beta^*$は一定.また,
\[ \alpha+\alpha^*=\pi, \]
\[ \beta+\beta^*=\pi \]
より
\[ \theta+\alpha+\beta-\pi=\text{一定}.\]
よって$[ABX]=[ABC]$.
$X$が$\capp(A^*CB^*)$の外側にあるとき,右図より
\[ [ABX]<[ABX^{\prime}]=[ABC] \]
となる.$X$が内側にあるときも同様.
\end{proof}

\section{球面上の等周定理}
これまで見てきたように,球面幾何学とユークリッド幾何学ではかなり様子が違う.しかし,一方で類似の定理も存在する.最後にそれを紹介しよう.
\begin{theorem}[球面上の四角形の等周定理]\label{toushuu}
  球面上の四辺形$ABCD$で$AB=a$,$BC=b$,$CD=c$,$DA=d$ ($a+b+c+d<2\pi$)と定まっているもののうち面積最大のものはキャップに内接する.\end{theorem}
証明のためにいくつかの補題を用意する.

\begin{lemma}$\widehat{ABC}$が半円のとき,$AC$の長さを変えると$[ABC]$は減少する.\end{lemma}
\begin{proof}
まず,$\widehat{ABC}が半円なら\widehat{B^*A^*C}が半円$となる.なぜなら,系9より
\begin{align*}
\widehat{ABC} が半円 &\Leftrightarrow \angle B=\angle A+\angle C \\
&\Leftrightarrow \pi-\angle CB^*A^*=\pi-\angle CA^*B^*+\angle C \\
&\Leftrightarrow \angle CB^*A^*=\angle CA^*B^*+\angle C \\
&\Leftrightarrow \widehat{B^*A^*C} が半円
\end{align*}
となるからである.よって$B^*C$は$\capp(B^*A^*C)$の直径となる.そこで,中心$B$で$C$を通るキャップは$\capp(B^*A^*C)$に接する.$AB,BC$の長さを変えずに$AC$の長さを変えると$C$は$\capp(B^*A^*C)$の外側に行き,レクセルの定理から面積は減少する.
\end{proof}

\begin{corollary}$AB$,$BC$の長さが一定でその和が$\pi$未満の三角形は$\widehat{ABC}$が半円のときに限り面積最大になる.\end{corollary}

\begin{lemma}$a+b+c<\pi$となる3つの正実数$a,b,c$に対して$AB=a,BC=b,CD=c$と定まっている四辺形$ABCD$の面積が最大となるのは四角形$ABCD$が直径$AD$のキャップに内接するときであり,またそのときに限る.\end{lemma}
\begin{proof}
まず,面積最大のものが存在することを示す.$x=AD,y=AC$とおくと,
\[ x_0\le x\le a+b+c   \    (x_0はxがとりうる最小の値) \]
\[ \max\{|a-b|,|c-x|\}\le y\le \min\{a+b,c+x\} \]
が表す領域はコンパクトだから,$[ABCD]$は最大値を持つ.

次に,面積最大となるのが命題の条件を満たすときに限ることを示す.$ABCD$は凸であるとしてよい.$ABCD$が直径$AD$のキャップ上にないと仮定する.このとき,$\widehat{ABD}$または$\widehat{ACD}$は半円ではない.たとえば$\widehat{ABD}$が半円でないとすると,系14より$AD$の長さのみを変えて$\widehat{ABD}$を半円にすることで$[ABD]$は増加する.
\end{proof}

\begin{lemma}\label{naisetsu}周の長さが$2\pi$より小さい$n$辺形に対して,この$n$辺形と同じ順に同じ長さの辺を持つ$n$辺形でキャップに内接するものが存在する.\end{lemma}
\begin{proof}
辺の長さを順に$a_i(1\le i\le n)$とする.大円上に点$A_i$を,$|A_iA_{i+1}|=a_i$となるように置く.このとき
\[ |A_nA_0|>a_n \]
だから,大円を徐々に小さいキャップに変形することで$|A_nA_0|=a_n$となる瞬間がある.
\end{proof}


\vspace{5 ex}
\begin{proof}\cref{toushuu}の証明] \cref{naisetsu}によりキャップに内接する四辺形の存在は示されているから,その状態から図形を変形すると必ず面積が減少することを示せばよい.$c$が長さ最大の辺で,$b\le d$であるとしてよい.$A$とキャップの中心を結ぶ直線とキャップが再び交わる点を$P$とする.ここで,$AP$は$CD$と交わっていることに注意する.$C,D,P$を固定し,変形で$A,B$が移った先を$A^{\prime},B^{\prime}$とすると$[A^{\prime}B^{\prime}CP]<[ABCD]$かつ$[A^{\prime}DP]<[ADP]$が成り立つから
\[ [A^{\prime}B^{\prime}CD]<[ABCD] \]
となる.\end{proof}
\begin{center}
%WinTpicVersion4.32a
{\unitlength 0.1in%
\begin{picture}(17.2300,18.6000)(19.5900,-36.4800)%
% CIRCLE 2 0 3 0 Black White  
% 4 2800 2807 3640 2758 3640 2758 3640 2758
% 
\special{pn 8}%
\special{ar 2800 2807 841 841 0.0000000 6.2831853}%
% LINE 2 0 3 0 Black White  
% 22 2380 3542 3633 2793 3633 2793 2912 1967 2912 1967 2086 2359 2086 2359 2373 3549 2387 3549 3248 3528 3248 3528 3619 2800 3241 3514 2905 1946 2394 3528 2499 2338 2499 2338 3619 1967 3619 1967 3241 3535 3626 1967 3640 2800
% 
\special{pn 8}%
\special{pa 2380 3542}%
\special{pa 3633 2793}%
\special{fp}%
\special{pa 3633 2793}%
\special{pa 2912 1967}%
\special{fp}%
\special{pa 2912 1967}%
\special{pa 2086 2359}%
\special{fp}%
\special{pa 2086 2359}%
\special{pa 2373 3549}%
\special{fp}%
\special{pa 2387 3549}%
\special{pa 3248 3528}%
\special{fp}%
\special{pa 3248 3528}%
\special{pa 3619 2800}%
\special{fp}%
\special{pa 3241 3514}%
\special{pa 2905 1946}%
\special{fp}%
\special{pa 2394 3528}%
\special{pa 2499 2338}%
\special{fp}%
\special{pa 2499 2338}%
\special{pa 3619 1967}%
\special{fp}%
\special{pa 3619 1967}%
\special{pa 3241 3535}%
\special{fp}%
\special{pa 3626 1967}%
\special{pa 3640 2800}%
\special{fp}%
% STR 2 0 3 0 Black White  
% 4 3255 3556 3255 3626 2 0 0 0
% P
\put(32.5500,-36.2600){\makebox(0,0)[lb]{P}}%
% STR 2 0 3 0 Black White  
% 4 3682 2751 3682 2821 2 0 0 0
% D
\put(36.8200,-28.2100){\makebox(0,0)[lb]{D}}%
% STR 2 0 3 0 Black White  
% 4 3675 1911 3675 1981 2 0 0 0
% A'
\put(36.7500,-19.8100){\makebox(0,0)[lb]{A'}}%
% STR 2 0 3 0 Black White  
% 4 2905 1848 2905 1918 2 0 0 0
% A
\put(29.0500,-19.1800){\makebox(0,0)[lb]{A}}%
% STR 2 0 3 0 Black White  
% 4 2009 2296 2009 2366 2 0 0 0
% B
\put(20.0900,-23.6600){\makebox(0,0)[lb]{B}}%
% STR 2 0 3 0 Black White  
% 4 2331 3556 2331 3626 2 0 0 0
% C
\put(23.3100,-36.2600){\makebox(0,0)[lb]{C}}%
% STR 2 0 3 0 Black White  
% 4 2520 2359 2520 2429 2 0 0 0
% B'
\put(25.2000,-24.2900){\makebox(0,0)[lb]{B'}}%
\end{picture}}%

\end{center}

\vspace{10 ex}
\noindent 参考文献

[1]前原濶,桑田孝泰,『数学のかんどころ3 知っておきたい幾何の定理』,共立出版,2011.

[2]前原濶,『円と球面の幾何学』,朝倉書店,1998.

[3]杉浦光夫,『解析入門II』,東京大学出版会,1985.
\end{document}