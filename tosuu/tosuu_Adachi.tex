\documentclass[uplatex]{jsarticle}
\usepackage{MAstandard, appendix}
\DeclareMathOperator{\Int}{Int}
\newcommand{\smooth}{C^{\infty}}
\newcommand{\setinverse}[2]{#1 ^{-1}(#2)}
\DeclareMathOperator{\sign}{sign}
\crefname{theorem}{定理}{定理}
\crefname{proposition}{命題}{命題}
\crefname{lemma}{補題}{補題}
\crefname{definition}{定義}{定義}
\crefname{example}{例}{例}
\crefname{remark}{注意}{注意}
\newtheorem{case}{事例}

\title{都数副会長引継ぎ資料}
\author{安達 充慶}

\begin{document}

\maketitle

\begin{abstract}
    都数の副会長の引継ぎ資料です.副会長の仕事および今年度の事例について述べます.
\end{abstract}

\tableofcontents

\section{副会長の仕事}
副会長の仕事は一言で言うと「会長およびその他の執行部役員の補佐」です.したがって,これと言って決まった仕事はありません.以下では,今年度私がした仕事について記述します.副会長という仕事の性質上,これらの仕事が例年あるとは限らないので注意してください.

\subsection{副会長の心構え}
まず,副会長になるにあたって意識してほしいことは「会長の対になること」です.会長が暴走しそうになったらそれにブレーキをかけ,逆に会長が尻込みする場面では積極的に周りを先導しましょう.

今までの事例だと,2つ上の代では会長(宮澤さん)が積極的に動き,副会長(一之瀬さん)がブレーキ役,1つ上の代では会長(名取さん)が慎重派,副会長(高橋さん)が周りを先導するという形でした.一人の人間がすべてを決めてしまう状況は当然避けるべきですが,何も進まないのもいいことではありません.どちらの形が良い悪いということではないので,自分の代の会長のタイプに合わせて自分のスタンスを決めてください.

なお,今年度は少し特殊で,編集長の樋川が執行部を積極的に引っ張ってくれました.規約の改正案も大部分を彼が考えてくれました.そういうときに副会長がする仕事は,その内容に偏りがないかをチェックし,規約改正の決議までの段取りに待ったをかけ,執行部全体でのコンセンサスを形成することです.執行部内での認識に相違があると会員・会友は不信感を覚えます.そうならないように執行部全員で情報をきちんと共有するよう促しましょう.特にその案件に興味のなさそうな役員にも情報を確認してもらうようにしてください.

\subsection{会長の補佐}
日常的な仕事は主に会長の補佐です.Slackのgeneralを見れば分かるように,会長は会員・会友に向けた広報としての役割が強いです.他にも総会の発表を会員に依頼するなど,結構仕事量が多いです.必ずしも会長がやらなくてもいい仕事は副会長が積極的に引き受けましょう.次に挙げる仕事はやった方がいいと思われます.

\begin{itemize}
    \item 会長が書いた,Slackのgeneralなどに挙げる連絡事項の内容のチェック.
    \item 各種イベントの日程調整や発表練習などの進捗管理.
    \item 問題発生時の執行部での議論の先導.
\end{itemize}

総会の会場の予約は副会長がやってもいいかもしれません.特に石塚くんは東工大生なのでおあつらえ向きだと思います.

\subsection{種々の雑務}
副会長以外の人は決まった仕事があって,ときに人手が足りなくなります.そうしたときに手助けに入れるのが特定の仕事のない副会長の強みです.新歓会議に新歓担当が行けなければ代わりに出席し,打ち上げ花火を一人で上げるのが大変そうだったら助けに行く,というように縦横無尽に行動してください.

\section{事例}
\subsection{今年度の事例集} ここでは今年度あった事例について書きます.

\begin{case}
8月総会の会場の予約に不備があり,もともと使おうとしていたところが使えなかった.
\end{case}

8月はなんとか別の会場を見つけられましたが,2月にまた同じ問題を起こしてしまいこのときは失敗してしまいました.会場の確保ミスは発表者に多大な迷惑がかかるので絶対に起こしてはいけないミスです.特に先輩に発表を依頼している場合,忙しい合間を縫って準備をしてくださっています(8月総会のときは赤澤さんが大変優しい方で救われました).我々のミスを反面教師にして,会場を押さえるのに失敗したときのために代わりになる施設をあらかじめリストアップしておくといった対策を講じてください.

\begin{case}
2019年12月,出版社から「ある大学の先生の全集のようなものを作るため,過去の数学のなかまに掲載されたその先生のインタビューを掲載したい」と申し出があった.先生の側は掲載を希望しているようだった.
\end{case}

最終的には現執行部の権限で掲載を許可する旨を伝えました.

\begin{case}
2020年の2月,あるテレビ番組から「数学に関する内容で監修してほしい」と頼まれた.
\end{case}

都数という団体としては回答することができないことを伝えました.

\subsection{過去の事例}
\begin{case}
合宿中,夜間に酔った状態で外に出た会員が崖から足を滑らせて落ちた.幸い大きなけがはなかった.
\end{case}

この事件以降,合宿時は基本的に(特に飲酒時は)外出禁止,昼間であっても必ず執行部の許可を得てからにする,というルールができました.

\begin{case}
ある年の会計係の仕事が杜撰だったらしく,帳簿と実際が10万円近くずれた.
\end{case}

その会計係の人はクビになったらしいです.教訓は,会計係はしっかりした人を選ぼうということと執行部みんなできちんと監査をしようということです.

\subsection{申し送り事項}
\begin{case}
高校生で都数の総会に参加を希望する方がおり,その方が会費を払うか払わないかで一悶着あった.
\end{case}

結果的に来年度から会友として都数に参加することになりました(ちなみに会員は学部生がなるもので,それ以外はどんな場合でも会友です).

\section{今年度の反省}
副会長のする仕事が少なかったかなと思います.周りからのアクションがなくても自分から何かやることがないか声をかけて仕事をもらってください.副会長は普段何をしているか一番分からない役職なので,そういう姿勢を見せることが他の役員からの信頼獲得につながるはずです.

\section{最後に}
副会長は仕事内容うんぬんというよりはその立ち位置が重要な役職です.周りの状況を俯瞰して,全体が極端が法に流れないようにバランスをとることを意識してください.

また,都数内での取り決めにはそれなりに意味があって制定されたものがほとんどです.例えば合宿中の夜間外出禁止なども上の過去の事例に書いた事件をもとに決められたことです.どうしてそうなっているのか分からないことがあれば先輩に訊いてください.もしかしたら重大な理由があるかもしれません.

それでは,健闘をお祈りします.

\end{document}

メールの文面チェック

高校生
出版社
ガキ使
総会の教室借り

仕事が少なかった