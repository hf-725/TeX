\documentclass[uplatex]{jsarticle}
\usepackage{tosuustd2019,appendix,caption}

\DeclareMathOperator{\Int}{int}
\DeclareMathOperator{\sign}{sign}
\newcommand{\smooth}{C^{\infty}}

\crefname{theorem}{定理}{定理}
\crefname{proposition}{命題}{命題}
\crefname{lemma}{補題}{補題}
\crefname{definition}{定義}{定義}
\crefname{example}{例}{例}
\crefname{remark}{注}{注}

\title{今年度の大学院入試制度}
\author{受験生有志}
\date{}

\begin{document}

\maketitle

\begin{abstract}
今年度の修士大学院入試は新型コロナウイルス感染症の影響でその実施形態に多大な影響があった.
本稿は,各大学で生じた変更の記録を目的としており,例年と
今年の形態を比較する形をとっている.今年度の大学院入試は特殊ではあるが,それでも今後各大学を受験する際の
対策を練るための参考にはなるであろう.この記事が来年度以降受験を控える都数会員の助けとなれば幸いである.
\end{abstract}

\tableofcontents

\newpage
\section*{京都大学大学院理学研究科数学・数理解析専攻}
\addcontentsline{toc}{section}{京都大学大学院理学研究科数学・数理解析専攻}
京都大学大学院において,数学の専門教育を担うのが「理学研究科数学・数理解析専攻」である.
\subsection*{コース分け}
\addcontentsline{toc}{subsection}{コース分け}
数学・数理解析専攻は,更に「数学系(数学基盤コース)」「数学系(数学先端コース)」「数理解析系」の3コースに
分かれており\footnote{「数学系」を「数学教室」と呼ぶことがあり,また,「数理解析系」を「数理研」「RIMS」と
呼ぶことがある.「数理研」「RIMS」は教員の所属研究機関である「数理解析研究所」の略称である.},
学生はこのうちどれか1つのみに属する.
「数学系」における\textbf{「基盤」と「先端」の区別は2022年度以降の修士課程入学者からは廃止予定}であるため,
この2つの違いは詳述しないが,後者の方が合格するのが難しい.
「数学系」と「数理解析系」の最大の違いは,所属教員が全く異なることである.

出願者は,志望順位を決めた上で複数コースを併願可能であり,「第$n$志望に合格すると第$n+1$志望以下には不合格と
なる」という原則を除けば,いかなる形においても,この志望順位は不利益には働かない(という旨を,筆者は数理解析系の
入試説明会において教員から聞いた).
\subsection*{例年の入試}
\addcontentsline{toc}{subsection}{例年の入試}
例年,レポート・筆答試験・口頭試問の3つが課される.

例年の入試では数学系の2コースは(口頭試問の内容まではわからないがそれ以外は)全く同じ試験を受けるので,
ここでは区別しない(つまり,数学系と数理解析系の区別のみ覚えていればよい).

筆答試験は両系に(ほぼ)共通であり,レポートと口頭試問は両系で別々である.詳細は後述する.

たとえば令和2年度入学者選抜(コロナ禍の前年)においては,日程は以下の通りであった.

\begin{table}[h]
\begin{center}
	\begin{tabular}{|c|c|} \hline
		日程・期間 & 事項(試験科目等) \\ \hline \hline
		7月3日~7月5日 & 出願書類及びレポート受理 \quad \\ \hline
		8月24日9:00~12:30 &\quad 筆答試験(基礎科目) \quad \\ \hline
		8月24日13:45~16:15 &\quad 筆答試験(専門科目) \quad \\ \hline
		8月25日9:00~10:00 &\quad 筆答試験(英語) \quad \\ \hline
		8月25日~8月27日 &  口頭試問 \quad \\ \hline
	\end{tabular}
\end{center}
\end{table}

\subsubsection*{レポート}
令和2年度入学者選抜(コロナ禍の前年)での内容を述べるが,毎年同じであったと推察される.

数学系では,「志望研究分野調査書」という予め決まった書式(A4両面)に書き込むものが要求される.表面では,
学部での講究のテキスト名,志望研究分野,希望指導教員などを書く.また裏面では,「重要だと思う定理,重要だと思う
理由」のような内容を書く欄と,「勉強した書物・論文」を書く欄がある.それとは全く別に,オリジナルと思われる結果が
ある人は,それをレポートにまとめ提出してもよいとされている.

数理解析系も要求はほぼ同じであるが,フォーマットが異なっており,「数理解析研究所で勉強したいと思う具体的テーマ
(複数可)があれば書いてください。」等,問いが少し増える.また,「重要だと思う定理」のような内容については,
別途レポート(A4用紙で2,3ページ以上,上限なし)にまとめることが要求される.数理解析系ではオリジナルの結果に
ついては特に言及はない.

\subsubsection*{筆答試験}
筆答試験は過去15年ほどのうちに(コロナ禍の対応を除いて)3回,その科目区分等の形式が変わっており,
最近の形式では計4回試験が行われている.最近の形式では次の3科目がある.
\begin{description}
\item[・基礎科目]\mbox{}\\ 3時間30分で6題全部を解く
\footnote{厳密には,数理解析系志願者のみ選択可能な選択問題がある
	(「5番の代わりに7番を解いてよい」など).
	数理解析系では応用数学や物理を専門とする受験者にも門戸を開いており,
	そのための配慮だと教員から聞いている.数学系併願者でもそちらの問題を解くことはできるが,
	数学系では採点対象にならない.
	また,「選択」と書いたが,両系併願者は両方を解くことも許されていることが,
	問題の表紙に明記されている.
	「両方解けたが,数理解析系限定の問題の方がより自信がある」場合などへの配慮であろうか.}.
微積分と線形代数が計算と理論で2問ずつ計4問,常微分方程式と複素解析が年度ごとに交代で1問出題されている.
もう1問は多様体論などである.
\item[・専門科目]\mbox{}\\ 2時間30分で10題中2題を解く
\footnote{厳密には,両系共通の10題に加え,数理解析系志願者しか選択できない2題がある(量子力学など).
また,やはり両系併願者は最大4問を解くことが許されている.}.
内容は専門的なもので,たとえば代数系では,有限群論,可換環論,Galois理論の計3題が出題されている.
\item[・英語]\mbox{}\\ 2時間で2題を解く.
「数学の文章の英文和訳」と「数学の問題の和文英訳および英語での解答」を
課される年度が多い.後者では論理記号等の使用を禁じられる.	
\end{description}
ある年度の数理解析系の入試説明会において,「数学と英語の得点配分はどれほどか」という趣旨の質問に対し,
教員は次のように回答していたと筆者は記憶している.

「数学の筆記は,非常に重視します.英語は,......見ています.」
\subsubsection*{口頭試問}
口頭試問の受験資格者は,各系ごとに,筆答試験の後に発表される.試問は各系で別々に行われるので,併願者は
(受験資格が両方とも得られれば)2度試問を受けることとなる.

どちらの系でも,筆答試験の解き直しや数学の専門的内容の試問のほか,事前提出のレポートについても聞かれることがある.

他大学(東大など)と日程が重なることもあるが,これは適切に配慮してもらえる.そうした事情や,滞在費がかかること
への気遣いからか,遠方の受験者の方が順番が早く回ってくるようである.
\subsection*{今年度の入試}
\addcontentsline{toc}{subsection}{今年度の入試}
コロナ禍対応による例年との主な違いは以下の3点である.
\begin{itemize}
	\item レポートの要求が増えた.
	\item \textbf{筆答試験が行われず},それに伴い,\textbf{1次選抜が提出書類とレポートのみによる}ものとなった.
	\item \textbf{口頭試問がZoomを用いてオンライン化}され,内容や形式が変わった.
\end{itemize}
日程は以下の通りである.
\begin{table}[h]
	\begin{tabular}{c}
\begin{minipage}{0.5\hsize}
		\begin{center}
			\captionsetup{labelformat=empty,labelsep=none}
			\caption{数学系}
			\begin{tabular}{|c|c|} \hline
				日程・期間 & 事項 \\ \hline \hline
				7月6日~7月10日 & 出願書類受理 \quad \\ \hline
				7月17日 & レポート締切 \quad \\ \hline
				8月2日9:00 & 1次合格者発表 \quad \\ \hline
				8月22日~8月23日 & 1次口頭試問 \quad \\ \hline
				8月23日21:00 & 2次口頭試問受験者発表 \quad \\ \hline
				8月24日~8月26日 & 2次口頭試問 \quad \\ \hline
			\end{tabular}
		\end{center}
\end{minipage}
\begin{minipage}{0.55\hsize}
		\begin{center}
			\captionsetup{labelformat=empty,labelsep=none}
			\caption{数理解析系}
			\begin{tabular}{|c|c|} \hline
				日程・期間 & 事項 \\ \hline \hline
				7月6日~7月10日 & 出願書類受理 \quad \\ \hline
				7月17日 & レポート締切 \quad \\ \hline
				8月3日12:00 & 1次合格者発表 \quad \\ \hline
				8月22日~8月25日 & 口頭試問 \quad \\ \hline
			\end{tabular}
		\end{center}
\end{minipage}
\end{tabular}
\end{table}
出願書類は両系で一緒だが,レポートは送付先からして異なっていた.

上記のほかに,Zoomの接続テストが各系ごとに事前に行われた.

以下に,例年との違いを詳述する.
\subsubsection*{レポート}
数学系で生じた比較的些細な変更は脚注にまとめる
\footnote{
「重要だと思う定理」のような内容に関し,「調査書」に書き込むのではなく,
別途レポートを要求されるようになった(数理解析系程度まで量的な要求が増えた).
なお,このレポートには,数理解析系のレポートの流用が許可されている.
また,「オリジナルの結果」のレポートについて言及がなくなった.}.
より大きな変更としては,
\begin{itemize}
	\item 数学系で,ある年度の過去問を解き,その解答を書いて送ることが要求された.
	\item 数理解析系で,それまでに勉強したことの概要をA4用紙3~5枚にまとめることを要求された.
	\item 数理解析系の例年通り要求されるレポートについて,更にその英語での要約を要求された.
\end{itemize}
以上に加え,数学系の数学先端コース志願者は,1次合格発表当日に,
\begin{center}
	発表と同時に指定された過去問2題を解いてオンライン提出する
\end{center}
ことが要求され,これは口頭試問の資料として用いられた.
問題は人それぞれ違ったが,全員が別というわけでもなかった.
発表が午前9時で提出期限が午後5時であった.
インターネット上の情報のみならず書籍等の参照すら禁じられていたが,それを不正とみなすなら,
実質的に「不正し放題」であった.筆者も「この問題の答えはAti-Macに載っているよな」と思ったりしたが,
びびりなので,見るのはやめておいた
\footnote{幸運にも易しい問題であったので,結局は筆者は損をしなかったが,
	正直者が損をするルールはやめてほしいものである.}.
\subsubsection*{口頭試問}
	口頭試問は数学系で2回,数理解析系で1回,いずれもZoomを用いてオンラインで行われた.以下に内容を記すが,
	口頭試問については受験者や分野によって内容が大きく変わりうる.
	以下は,京大の公式の情報,筆者自身の経験,知り合いの受験生からの伝聞を無難にまとめたものである.予め了解されたい.
\begin{description}
	\item[・数学系1次口頭試問]\mbox{}\\
	「類題を出題するのでこの期間の過去問を見ておいてください」
	というような指示を受験生全員が共通に受けており,その通りに過去問の類題をその場で解かされる.
	あとは合否に影響しない事務的な内容のみであった.
	筆者の場合,行列計算,広義積分,抽象線型代数の3問であった.
	積分で詰まっているとヒントをくれた.1人30分くらいでやろうとして押しているようだった.それはそう.
	\item[・数学系2次口頭試問]\mbox{}\\
	先端コース志願者であって,1次口頭試問で先端コースの不合格が確定しなかった者が2次に進む.過去問レポートや
	「重要な定理」レポートについて聞かれ,また,それとは別に志望分野に関連した専門的な内容を聞かれる.
	ひとり1時間半が与えられているが,実際の時間は個人差が大きいようである.
	\item[・数理解析系口頭試問]\mbox{}\\
	数学系同様,予め過去問を見ておくことが要求され,その上,それを事前に印刷するなどして手元に置いておくことに
	なっていた.
	筆者の場合,過去問をそのまま聞かれたあと,「この仮定をこのように変えるとどうか」ということを聞かれ,また,
	過去問とは独立に専門分野(数論)関連の別の問題を聞かれた.
	専門的な高度な知識などは本質的には全く聞かれなかった(このあたりは特に受験者や分野によって変わるだろう).
	時間は45分程度であった.
\end{description}

\subsection*{対策のポイント}
\addcontentsline{toc}{subsection}{対策のポイント}
また次は次で変わるかも知れないが,今回の「コロナ禍対応」の特殊事情から生じるアドバイスを列挙する.
\begin{description}
	\item[・レポートのコピー]\mbox{}\\ 試問前に復習するために,提出前にコピーを取っておくことを強く勧める.
	
	\item[・レポートの重要性]\mbox{}\\ 事前提出のレポートは想像以上に重視されているようで,たとえば数理解析系では
	レポートで判断される1次合格発表の時点で強烈な選抜が働いており,筆者はとてもびびった.受験生は十分に時間をかけ,
	内容をよく検討されることを勧める.筆者の場合,(他の受験者からの伝聞や自身の体験から判断するに,)運良く
	「重要な定理」レポートでは両系で高い評価を得たようである.
	理由を挙げるので参考にされたい.まず,筆者は数理解析系を第1志望としていたが,レポートでは「なぜ数理研に
	入学したいか」をよく表現できるよう工夫した.
  口頭試問での会話から判断するにこれが先生方に伝わったようである.
	数学系では,(誰が見ても「しょうもない」ものであり本質的には既出だと筆者は思うのだが)一応は非自明な独自の
	結果を得ているのが評価に繋がったようである.
	
	\item[・口頭試問の筆記媒体]\mbox{}\\ 口頭試問では「紙に書いたものを映す」やり方も認められていたが,
  筆者は電子機器(iPad等)の使用を強く推奨する.「紙を映す」だったら無駄に沢山時間がかかっていただろう,
	あるいは「紙を映す」だと今の会話は難しかっただろう,という場面が少なくなかった.
	
	\item[・過去問について]\mbox{}\\ 例年から(たとえば数理解析系の説明会での教員の発言によると)
  過去問の勉強は推奨されているが,
  今回はその比でなく,
  過去問を真面目に研究した人が直接的に有利になる内容であった.
  早くから解いてよく分析しておくことを勧める.ちなみに,
  「試験実施年(度)」と「入学年(度)」の違いが,
  京大が公開している過去問PDFに現れている(PDFの名前と中身で年度表記がずれている)ので,
	人と話すとき等に気を付けよう.
	\newpage %体裁を整える用.もしコンパイル時に変なことがあったらここを修正してください.
	\item[・英語について]\mbox{}\\ 数学系は英語力を問うような課題を全体を通して
  一度も出してこなかった.この扱いからも,入試対策という意味では,
  英語は(何か試験やレポートが課されていても)ほぼ気にしなくてよい
  であろうことが改めて推察される.
  もちろん入試以外のことも考えれば「英語を気にしない」のはとても推奨できないが,
	瀬戸際で優先順位をつける上では悪くないだろう.
\begin{flushright}
	(文責 立原礼也)
\end{flushright}
\end{description}

\newpage
\section*{東京大学大学院数理科学研究科}
\addcontentsline{toc}{section}{東京大学大学院数理科学研究科}
\subsection*{例年の入試}
\addcontentsline{toc}{subsection}{例年の入試}
例年東大数理の入試は8月の下旬から9月の上旬にかけて行われる.筆記試験を2日かけて行い,
筆記試験合格者を対象に口頭試問が行われる.

筆記試験は
\begin{itemize}
  \item 外国語
  \item 専門科目1
  \item 専門科目2
\end{itemize}
の3つに分かれている
\footnote{
  専門科目1,2については専門科目A,Bという呼び方をすることもある.
}.外国語は要するに英語力を問う試験である.数学の問題を解く試験は2つに分かれており,
専門科目1は学部1,2年生で習う内容について問われ,必答問題2題と選択問題2題を3時間で解く.
専門科目2では学部3年生以降で習う発展的な内容について問われ,選択問題3題を4時間で解く.

筆記試験は2日続けて行われ,初日に外国語および専門科目1が,2日目に専門科目2が行われる.

筆記試験の翌日の午後に筆記試験合格者の発表が行われる.日程は2日用意され,指定された日程で口頭試問を
受けることになる.

\subsection*{今年度の入試}
\addcontentsline{toc}{subsection}{今年度の入試}
まずは出願時の提出書類に変更が生じた.例年通りの出願書類に加えて,
\begin{itemize}
  \item 今まで読んだ本のリスト
  \item 自分が興味を持って勉強したことについてのレポート
  \item レポートの内容を英語で要約したもの
\end{itemize}
の提出を求められた.これに加えて,専門科目1,専門科目2の試験および口頭試問をオンラインで行うことになった.
試験時間および試験の内容に変更はない.
試験に関する例年との変更点は主に
\begin{itemize}
  \item 外国語の試験がなくなったこと
  \item 専門科目1と専門科目2が連続した2日間では行われなかったこと
  \item 専門科目1の必答問題と選択問題は,午前と午後に分けてそれぞれ1時間半ずつで行われたこと
  \item 第一次選抜が専門科目1の結果のみで行われたこと
  \item 口述試験の日程として4日間が確保されたこと
\end{itemize}
である.特に第一次選抜に合格した場合,専門科目2の出来不出来に関わらず口頭試問を受けることができる.
口頭試問に関しては結果的には3日間で全日程が終了した.

\newpage %体裁を整える用.
試験の日程は次の表の通りである.
\begin{table}[h]
  \begin{center}
    \begin{tabular}{|c|c|} \hline
      期日 & 試験科目等 \\ \hline \hline
      8月20日 &\quad 専門科目1のオンライン試験 \quad \\ \hline
      8月26日 &  \quad 第一次選抜者の合格発表 \quad \\ \hline
      8月31日 & \quad 専門科目2のオンライン試験 \quad \\ \hline
      9月1日~9月4日 & \quad オンラインによる口述試験 \quad \\ \hline
    \end{tabular}
  \end{center}
  \end{table}

\subsection*{対策のポイント}
\addcontentsline{toc}{subsection}{対策のポイント}
東大に関しては試験がオンラインに変わったもののその骨格自体に大きな変化はなかったため,
ほとんどの内容は例年の入試形態と今年度の入試形態どちらでも通用する.

\subsubsection*{外国語}
普段から洋書で数学を勉強する癖がついている人は特別に対策すべきことはない.
もしかしたら意味を知らない専門用語があるかもしれないので,過去問を数年分解いてみて穴がないかをチェック
する程度でもいいだろう.

逆に普段あまり洋書を読んでいない場合は力を入れて対策をする必要がある.特に専門用語で「日本語でなら
分かるが英語での表現が分からない」という概念を洗い出す必要がある.一番早いのは過去問を解いて
ボキャブラリーを増やすことだろう.また,当然のことであるが本来は早いうちから洋書を読むことに慣れて
おくべきである.特に専門書はそこまで難しい英語が使われているわけではないので,低学年の人はなるべく
早いうちに洋書に手を出すことをおすすめする.

\subsubsection*{専門科目1}
内容は線型代数と微分積分が中心で,基礎基本がしっかりしていれば2問半くらいはとれる構成になっている.
特殊関数の問題など誰しも勉強しているわけではない題材が扱われることもあるが,そのような問題は無視しても
支障がないくらいの問題数は用意されている.

対策の基本は,日頃の勉強を丁寧にすることに尽きる.発展的な内容を勉強し始めると線型代数や微分積分の
細かい部分(例えば積分と微分の順序交換など)のチェックをおろそかにしがちになるかもしれないが,それは
一番やってはいけないことである.専門科目1では,細かい部分の議論を丁寧にできるかどうかが問われている
と言っても過言ではない.ここを普段から意識している人にとってはそこまで難易度は高くないだろうし,
むしろそうでない人にとってはかなり難しい試験に感じる(あるいは解けたつもりになっても実際には
点数がとれていないという状況に陥る)だろう.

\subsubsection*{専門科目2}
数学科3年次以上で習うような発展的事項について問われる試験である.全部でだいたい15問ほど用意されており,
おおむね代数・幾何・解析から4問ずつとその他の分野から数問が出題される.応用数学志望の人は問題選択のしかた
が難しいが,それ以外の代数・幾何・解析のいずれかを志望している人は自分の志望の問題を選択すれば3題に到達
する.どの問題も難易度はそれなりに高いので,普段勉強していない分野の問題は多少対策してもまず
解けないと思っていた方がよい.院試で得点することだけを考えるなら,今まで勉強してきて十分な慣れのある分野
の問題を解けるようにするのが一番である.

筆者が幾何専攻なので代数や解析の問題がどうなっているのかは分からないが,幾何の問題に関しては例年1題は
やさしい問題がある.まずはそれをきっちりと解ききることが重要である.残り2問は完答できなくても途中まで
解くことができれば十分合格ラインに乗っているはずである.3問全部を完答しに行くと一問あたり1時間20分で
短く感じてしまうが,1完2半で十分だと考えると完答する問題には2時間ほどかけてもよいと思えるのではない
だろうか.

\subsubsection*{口頭試問}
主に専攻しようとしている分野の基礎的な知識について聞かれる.筆記試験の内容についても聞かれることがある
ため,筆記試験終了後の解き直しは非常に重要である.この解き直しのときには友人と
一緒に答え合わせをし,分からなかった問題は友人から答えを聞くべきである(筆記試験終了後の解答の共有は
全くもって禁止されていない).また,これまでに読んだ本の内容については(当たり前ではあるが)
その主要な部分については復習をしておくべきである.短い試問時間の中で細かい計算を聞いてくることはないので,
\begin{itemize}
  \item 主要な定理の正確な主張
  \item キーとなる概念の定義
  \item 議論のアウトライン
\end{itemize}
をしっかり頭に入れて試験に臨むべきである.

\begin{flushright}
  (文責 安達充慶)
\end{flushright}

\newpage
\section*{東京工業大学理学院数学系}
\addcontentsline{toc}{section}{東京工業大学理学院数学系}
\subsection*{例年の入試}
\addcontentsline{toc}{subsection}{例年の入試}
試験は2日に分けて行われ,初日に筆記試験,2日目に口頭試問が行われる.初日の夜に口頭試問を受けることができる者の
発表が行われる.まず希望者は事前にTOEIC,TOEFL PBT,TOEFL iBTのいずれかのスコアシートを提出する.TOEICなら
550点以上,TOEFL PBTなら490点以上,TOEFL iBTなら57点以上で英語試験は合格となる.提出しなかった場合は
初日の専門科目の試験の直後に英語の筆記試験(1時間)を受ける.

初日は基礎科目と専門科目からなる数学の筆記試験(と英語の試験)が行われる.まず午前の基礎科目は2時間半の試験で,
微分積分2題・線形代数2題・位相空間論1題からなる大問5題が出題される.午後の専門科目は2時間の試験で,
学部3年以降で習う専門の問題を2時間で8問中2題選択し解く.大まかな傾向だが,代数が2問,幾何が3問,解析が3問
出題される.その後,事前に外部英語テストのスコアシートを提出しなかった者は1時間の英語の試験(辞書持ち込み可)
を受験する.

\subsection*{今年度の入試}
\addcontentsline{toc}{subsection}{今年度の入試}
例年と同じく2日間に分けて行われた.初日に基礎科目と専門科目の試験が行われ,2日目に口頭試問が行われた.
しかしすべてオンラインでの受験となったほか,英語の試験が外部試験を用いず2日目の口頭試問の一部として
行われたり,基礎科目の出題が3問に減ったりするなど,内容が一部変更となった.

試験の初日は,最初に事前に通知されたZoomの部屋に入室した.その後試験官から試験の説明を受け,まず30分の線形代数の
試験を受けた.資料の持ち込みは可とされ,常にZoom上で自分の顔を映しておくよう指示された.30分後自分の答案をPDF
ファイルにし,指定されたメールアドレスに提出した.その後同様に位相空間論と微分積分の試験が行われ,基礎科目の
試験が終了した.

午後は専門の問題を90分で7問中2題を選択して解く形式だった.出題範囲は例年通りだった.その日の夜にメールで筆記試験の
結果が通知された.

2日目にZoom上での口頭試問が行われた.指定された時間にZoomの部屋に入り,面接官たちに自分の画面を共有しながら
30分間質問に回答した.主に聞かれたのは前日間違えた箇所だった.英語の試験もこの時間で行われた
\footnote{僕は「英語には自信がありますか?」と聞かれただけでした.}.
\begin{flushright}
	(文責 下境琢海)
\end{flushright}

\newpage
\section*{名古屋大学大学院多元数理科学研究科}
\addcontentsline{toc}{section}{名古屋大学大学院多元数理科学研究科}
	\subsection*{例年の入試}
	\addcontentsline{toc}{subsection}{例年の入試}
  名古屋大学は例年筆記試験のみで,午前の部(大問4つ3時間)と午後の部(大問4つ3時間)の2回に分かれている.
  出題範囲は微積分,線形代数,複素解析(留数定理まで),集合と位相の4分野.
	
  午前の部と午後の部の難易度差はそれほどなく\footnote{過去問は1年分しか解いてないため参考程度に.},
  どの大問も気づければ\footnote{自明にこれが一番難しいため,普段の大学の授業の勉強とは別に,演習本を多くこなして
  解法パターンを知る必要がある.}15分程度で解けるかと思われる.またどの問題も答えはかなり綺麗
  \footnote{名古屋大は試験の翌日に合格発表なので採点を早くするための工夫かと思われる.}になる.
  ボーダーは基本は6割・7割とれば安全圏,5割でも年によっては合格すると言われている.
  今年を除くここ6年の平均倍率は1.73である.
	\subsection*{今年度の入試} 
	\addcontentsline{toc}{subsection}{今年度の入試}
	\subsubsection*{試験} 
  筆記試験から口述試験に変更になった.出題範囲は例年通りということだったが実際出題されたのは微積分と
  線形代数で,複素解析と位相からは出題されなかった\footnote{おそらく受験者ごとに出題が異なるので私が
  たまたまそうだっただけの可能性もある.}.面接官は一人で,試験は4題を制限時間20分.解答を手元の紙に
  書きZoomのカメラに映すといったもので,口述試験感は全くなく,筆記試験をカメラで監視されながら受ける
  ようなものだった.ちなみに例年は1日で試験が終わるが,口述試験になった今年は2日間に渡って試験が行われた.
  試験官の反応から推測するに4問中2問正解で安全圏であると思われる.倍率は1.47だった.
	\subsubsection*{その他}
  名古屋大学では入学後,授業が始まる前に予備テスト\footnote{過去問の様子からすると,
  少なくとも例年の院試問題よりは難易度が高いようである.1問3点満点で計4問の12満点中9点が合格ラインで
  これに不合格であれば半期もある補修の授業を受け,前期の終わりにある修了テストをさらに受けて合格する
  必要がある.またこの修了テストも不合格であれば,その翌年にもう一度補修の授業と修了テストを
  受けなければならず,再度これに不合格であれば留年扱いとなる.}を受ける必要がある.
  例年では院試の成績上位者はこれを免除されるのだが今年は免除されることはなく,入学者全員が予備テストを
  受けなければならない.
  \begin{flushright}
    (文責 幡航太朗)
	\end{flushright}
	
	\newpage
	\section*{北海道大学大学院理学院数学専攻}
	\addcontentsline{toc}{section}{北海道大学大学院理学院数学専攻}
	\subsection*{例年の入試}
	\addcontentsline{toc}{subsection}{例年の入試}
  北海道大学は例年,筆記試験を行わず,好きな定理についてのレポート\footnote{詳しくは,
  「入学後に研究したい分野並びに特に興味を持った定理,理論について自分が十分に理解し説明できる事柄を
  A4版用紙7~10ページにまとめること」と募集要項に書いてある.また,A4版用紙7~10ページとは
  目安として人に話したら2時間くらいかかる量である.}とそのレポートについての質問\footnote{おそらく
  これは本人確認のためと思われる.}と基本的な数学に関する質問のみで合否を決める.この面接には札幌会場と
  東京会場が設けられている.去年の倍率は1.02だった.
	\subsection*{今年度の入試}
	\addcontentsline{toc}{subsection}{今年度の入試}
  面接は本来の札幌会場と東京会場では実施せず,全てオンラインによる口頭試問となった.しかしながら,
  北海道大学の場合は元々が口述試験なので例年とほぼ変わらなかった.
  例年と違うところは,口頭試問の最初に10分程度,出願時に提出したレポートについて事前に用意したスライドや
  ノートなどを共有したりあるいは直接カメラに映したりしながら説明しなければならないことぐらいであった.
  口頭試問ではレポートに関する質問はもちろんのこと,「かなり具体例について聞かれたな」という印象を
  もった.また,「大学院に入ったら何をしたいか」,「そのために今どんな勉強をしているのか」という勉強の
  姿勢を問うような質問もあった.これらに加えて基本的な数学に関する質問があった.面接官は3人,質問個数は
  約20個,面接時間は全部で30分程度であった.倍率は1.15であった.
  \begin{flushright}
    (文責 幡航太朗)
	\end{flushright}
	
\newpage	
\section*{早稲田大学大学院基幹理工研究科数学応用数理専攻}
\addcontentsline{toc}{section}{早稲田大学大学院基幹理工研究科数学応用数理専攻}
\subsection*{例年の入試}
\addcontentsline{toc}{subsection}{例年の入試}
例年,早稲田の入試は7月上旬に行われ,結果が7月中旬には発表されている.
入試は2日連続で行われ,初日に筆記選考,翌日に面接選考という流れである.

筆記選考では学部1,2年レベルの内容を問う必答問題を3題,及び学部3年レベルの内容を問う選択問題を1題解く必要がある.
時間は3時間で,必答,選択問題の解答は同時に行う.

必答問題は,微積分1題,線形代数1題を必ず解答し,基礎数理
\footnote{ 位相の問題であることが多いが,2020年度は一方が初等的な組合せ論の問題だった.}
2題のうちどちらか一方を選択して解答する.

選択問題は,代数2題,解析2題,幾何2題,そして,関数論,確率統計,応用数理から1題ずつの合計9題のうちから
1題選択して解答する.

また,出願時にTOEIC L\&R 550以上のスコアレポート
\footnote{ TOEFL iBT 57以上またはIELTS Academic5.5以上のスコアレポートでも代用できる.}
を提出する必要がある.これは出願開始日の2年前以降に受験したものである必要があるため,例えば大学1年及び
2年前期でのスコアは使えないことに注意してほしい.

\subsection*{今年度の入試}
\addcontentsline{toc}{subsection}{今年度の入試}
まず,コロナの影響で3月から8月までTOEIC試験が行われなかったため,スコアレポートの提出は1月19日まで受け付ける
ことになった.しかし,スコアレポートを提出しなくても良いとまではならなかった.

さらに,選考方法にも変更が生じた.筆記選考を実施せず,面接選考のみをZoomを用いたオンラインで行うという形になった.
面接時間は30分程度であり,非常に短い院試となった.


\subsection*{対策のポイント}
\addcontentsline{toc}{subsection}{対策のポイント}
今年度は筆記選考が行われなかったため,するべき対策について述べることができない.
今年度の面接選考に対してするべき対策を述べるが,例年の面接選考に対して役に立つかはわからない.

面接選考では,初めに本当に面接をした.志望した理由,そして大学院で研究したいことについて尋ねられた.
いずれも事前に提出する書類において記入したことなのだが,準備をしておかないと何を書いたのか忘れてしまうので,
即答できるように練習しておいた方がいいだろう.
次に,口頭試問に移り,学部1,2年レベルの内容について尋ねられた.線形代数や位相空間論,多様体論の基本的な定義や
性質について尋ねられる
\footnote{位相空間論や多様体について尋ねられたのは執筆者が幾何学専攻であったからだろう.代数学や解析専攻の
場合には別のことを尋ねられると思われる.}.
難しいことは聞かれずに,どの教科書を見ても書いてあるような本当に基本的なことが問われるので,線形写像というものや
コンパクト集合というものを空気の様に扱えるのであれば何も問題ないだろう.
\begin{flushright}
	(文責 富永直弥)
\end{flushright}
\end{document}

変更によるトラブルなども書く→やっぱ書かないことにした
各科目の出題内容や難易度などの詳細は「対策のポイント」で書く→書くかどうかは文責に任せた

句読点の位置の調整
ZOOM→Zoom
句読点をコンマピリオドに
()は半角