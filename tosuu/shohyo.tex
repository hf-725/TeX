\documentclass[uplatex]{jsarticle}
\usepackage{tosuustd2019}
\DeclareMathOperator{\Int}{Int}
\newcommand{\smooth}{C^{\infty}}
\newcommand{\setinverse}[2]{#1 ^{-1}(#2)}
\DeclareMathOperator{\sign}{sign}
\DeclareMathOperator{\Real}{Re}

\title{書評\\--- 幾何学への架け橋---}
\author{東京大学理学部数学科3年 安達充慶}
\date{}

\begin{document}
\maketitle

東京大学3年の安達充慶です.漠然と幾何学を勉強したいと思っている人がどんな本を読めばいいのか一つの道しるべを提示できればいいな,という思いで書評を書きました.一口に幾何学といっても色んな分野がありますが,ここでは主に多様体の幾何学にまつわる本を紹介します.

\section*{Lee, J.,\textit{Introduction to Smooth Manifolds.}}

多様体論の本といえば松本幸夫『多様体の基礎』や松島与三『多様体入門』が有名ですが,僕の一押しはこの本です.一応入門書なので多様体の基礎事項から書いてありますが,この本の魅力は何といっても扱う内容の豊富さです.
Sardの定理の証明も書いてありますし,その応用としてWhitneyの埋め込み定理・近似定理,そして横断性定理も勉強できます.de Rhamの定理も書いてあればLie群の基礎事項も書いてあり,勉強を進めるうちに不足していることに気が付いた足りない知識は大体この本で補えると言っていいのではないでしょうか.
そしてもう一つの素晴らしいところは,本の記述の簡明さです.証明の筋がいいことは言うまでもなく,
記号遣いもとてもうまいです.どうやったらこんなにきれいに本が書けるんでしょうか.他の本を読んで
「うわあこの定理の証明めっちゃ難しいじゃん」と思っていたものがいとも簡単に証明されていて驚いたことは数知れません(Frobeniusの定理ってめちゃめちゃすっきり証明できるんですよ).

とはいえ,逆に言えば話題が豊富すぎて,初めて多様体を勉強する人はどこまで読めばいいものか分からず途方に暮れてしまうと思います.まずは『多様体の基礎』を読んでみるのがいいのではないかと思います.


\section*{Milnor, J., \textit{Topology from the Differntiable Viewpoint.}}

幾何学の中での
バイブルと言ってもいい本かもしれません.多様体の基礎事項と,第二可算性にまつわるやや高度な位相空間の
知識だけで写像度の理論を展開します.写像度は2つの同次元の向きづけられたコンパクト多様体の間のなめらかな写像に対して定義される整数で,大雑把に言えば「その写像が値域の多様体を何回覆っているか」を表す量です.
写像度自体はde Rhamコホモロジーや特異ホモロジーなどを使って定義することもできますが,多様体以外にも
そこそこ色んな知識を使います.その点,この本は最小限の知識で多様体の幾何の面白いところを
感じ取ることができます.

一方でこの本の記述はかなりラフです.慣れている人が見れば「ああ,いつものあれね」となる議論が多いのですが,多様体を学びたての人が見るとかなり難しく感じると思います(かといってそういう人がこの本以外に読むのにぴったりな本が思いつくわけではありません).悪いことに,そういう行間をどのように埋めればいいかのヒントはあまり書かれていません.できれば多様体に慣れている先輩にいろいろ質問できる環境で読むことをお勧めします.

また,Milnorといえば(もちろん業績もさることながら)この本の他にも{\textit{Morse Theory}, \textit{Lectures on H-Cobordism Theorem}, \textit{Characteristic Classes}}といった名著を著しているレジェンドです.幾何学をやっていれば,彼の本はどこかで必ず読むことになります.

\section*{Bott, R. and Tu, L.,\textit{Differential Forms in Algebraic Topology.}}

この本も大変有名な本です.初めの方はde Rhamコホモロジーの基本的な性質から始まります.Poincar\'{e}双対やK\"{u}nneth公式といった定理が序盤で鮮やかに証明されます.ここまで読めば,Stokesの定理がde Rham理論の成否に深くかかわっているということに深く納得がいくでしょう.
2章では\v{C}echコホモロジーを定義して,これがde Rhamコホモロジーと自然に一致することが示されます.
\v{C}echコホモロジー自体は一般の位相空間に対して定義できるものであり,(微分同相より弱い)同相の下での不変量になります.
したがってこの同型の存在は,de Rhamコホモロジーが多様体の微分構造を反映していないことを意味しています.
この事情は4章で取り上げられる特性類の話につながっていきます.

前半だけ取り上げてもここまで綺麗にde Rhamコホモロジーの基礎がまとまった本はないような気がします.Milnorの本と並んで,幾何の人にとってはこの本もいつかは必ず読むことになる一冊だと思います.

\section*{西川青季,『幾何学的変分問題』}

この本はRiemann幾何学の入門書としておすすめです.まず初めに,標準的な計量が入っているとは限らないEuclid空間上の2点を結ぶ最短線が満たすべき方程式を直接導出します.実際には「2点を結ぶ最短線」(やや不正確ですが,これを測地線と言います)を見つける問題は多様体上で考えたいので,一つのチャートで話が収まるEuclid空間のときより事態は複雑になります.結局「接続」という概念を導入することになるのですが,この本ではさきほど書いたように具体的な話から始まっているので接続というものがいかに有効な概念かが分かりやすくなっています.

実は,一般のRiemann多様体については,2点を結ぶ最短線が必ず存在するとは限りません.この本の前半の一つの到達点は,2点を結ぶ最短点が必ず存在する(このことをRiemann多様体が完備であると言います)ための必要十分条件を与えるHopf--Rinowの定理です.この定理によれば,特にコンパクト多様体ならどんなRiemann計量を入れても完備になることが分かります.このように,ある意味この定理は多様体の位相的性質がRiemann多様体の性質に影響を与えるというタイプの定理だと言うことができます.Riemann幾何パートの締めくくりは,この種の定理であるSyngeの定理やMyersの定理で締めくくられています.

後半は主に調和写像についての話題です.大雑把に言って調和関数(Laplace方程式の解のこと)の拡張らしいですが,残念ながらぼくはここを読めていません.

\section*{Hatcher, A.,\textit{Algebraic Topology}.}

一冊くらい一般の位相空間を扱う幾何の本を紹介します.代数的トポロジーの本としては枡田幹也『代数的トポロジー』が有名で,初めて勉強する人にはそちらもおすすめです.しかし,ところどころごまかした議論が見られるのでそこが不満ではあります.

Hatcherのこの本は枡田先生の本に比べてレベルは高いですが,そういったごまかしはありません.商位相をはじめとする位相空間の操作に習熟している必要はありますが,逆にそういったテクニックを駆使することで色々な議論が見通しよくできるようになります.基本群や被覆空間,特異ホモロジー・コホモロジー,ホモトピー群という代数的トポロジーでよく聞く話題は網羅されています.

個人的に一番心が躍ったのは$\Delta$複体という概念です.これはいわば単体複体と胞体複体のちょうど間にある概念です.図形に胞体構造を入れるのは比較的簡単ですが,胞体ホモロジーは理論としてなかなか難しいと思います.一方単体複体を使ったホモロジー論は直感的には分かりやすいですが,実際に図形を単体分割しようとするととても大変です.トーラスの正方形モデルに対角線を引いただけでは単体分割にならないということにがっくりきた人もいるのではないでしょうか.

実は,$\Delta$複体とは,この「トーラスの正方形モデルに対角線を引いたもの」を許容するような構造なのです.その上,ホモロジー論の構成及び計算はほとんど単体複体と同じなので計算も非常にやりやすいです.これを初めて見たとき,「単体ホモロジーは初めからこれで教わりたかった」と思いました.去年ぼくが寄稿した記事にKleinの壺の特異ホモロジーを求めるものがありましたが,$\Delta$複体を使えばあんなの1分あれば十分です.なんでこれを使わなかったんだろうと思いましたが,これを勉強したのは寄稿を終えた直後でした.

ちなみにこの本は著者のHatcherが自身のウェブページでPDFを無料公開しています.感謝の気持ちを持ってダウンロードさせてもらいましょう.

\end{document}

topology from the differentiable viewpoint

Warner

Bott-Tu

層とホモロジー代数

Morse Theory

Hatcher

たぶん2019年の年末に書いた.日本語が色々とおかしいな.
