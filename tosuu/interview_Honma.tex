\documentclass[uplatex]{jsarticle}
\usepackage{tosuustd2019}
\DeclareMathOperator{\Int}{Int}
\newcommand{\smooth}{C^{\infty}}
\newcommand{\setinverse}[2]{#1 ^{-1}(#2)}
\DeclareMathOperator{\sign}{sign}
\DeclareMathOperator{\Real}{Re}

\begin{document}

まず,最初に数学に興味を持ったのはいつごろでしょうか.

本間:むずかしい質問だね(笑).子どもの頃からなんとなく興味を持っていたような気がするけど,幼稚園のときくらいから図形的なものは好きだったかな.パズルとか,折り紙とか,そういうものは好きだった.それで数学が本当に好きになったのは高校生のときですかね.数学だけは高校2年,3年からそれなりに勉強していました.

実はもともとは物理学科に行こうと思ってたんだけど,秋山仁の授業を聴いて数学科に行こうと方針転換したので,そこが一番の大きな転機だったような気がします.駿台予備校での夏期講習ですかね.あの先生はちゃんと授業をしないっていうか(笑),すごく衝撃的な授業だったんですよね.自由な感じの授業だった.例えばだけど,ある日の授業の前に,受講生の誰かが折り紙で何かの多面体を作って先生の机の上に置いといたわけね.それを見て「おっ,これはすごい!」とか言い始めて,20分くらい「じゃあみんなでこれ作ろう」とかいう話になって(笑).予備校でお金払ってるのになんだこの先生はと思ったけど,すごい人気講師でしたね.この人ちょっと変わってて面白いな,と思って.その頃,その秋山仁が書いてる本がいっぱいあったんです.当時駿台出版から出ていて,今は森北出版から再版されている『立体の捉え方』とか『技巧的な解き方』とかを一生懸命勉強して,それでもう数学にハマっていった感じですかね.

数学者になりたいと意識したことはありますか?

本間:意識は全くしたことがないです.数学者になろうと思ったのはいつだろうかなあ…….数学が好きだったからそのままやってただけで,大学に入っても別に数学者になろうとは思ってないから都数とか自主ゼミとかそういうのは学部時代は全くやってません.授業の復習をしたくらいで,もちろん授業が分からなくなったら自分で佐竹一郎の線型代数の本を読んだりはしたけど.あと『多様体入門』か.でもこれを読んだのも4年生とかだったからそんなに力を入れて勉強していたわけではないですね.

本腰入れて「数学やるぞ」という気持ちになったのはいつ頃でしたか?

本間:それはやっぱり修士になってからかな.4年生で卒業したときに,周りの文系の友達とかがみんな一生懸命就職していくわけじゃないですか.人生を目指してね(笑).俺はそういうことをあんまり考えずに生きてきた人間だったから,どうしようかなて真面目に考えたんです.それで勉強を一生懸命やろうかなと思って.修士にはそのまま行こうと思ってて,そこから一生懸命勉強し始めたんですけど,まあやっぱり修論はうまく書けなくて.でも就職活動もしてなかったから博士に入って,その頃はもう高校の数学教員になろうと思って教員免許をとったのかな.数学に携わって生きていきたいというのはあったんで.

博士の2年生の時に1年間で教職の授業を取って,だから博士の2年生なのに週5,6コマあったのかな.教育倫理学とか法学とかの授業のために文系の早稲田キャンパスに一人で通ってました.教育実習も同時にやってたし,教育心理学とかが面白くて数学は勉強せずにずっとそういうのを勉強してたかな.

まあ笑い話として言うんだけど,学部の授業に出てるわけだから博士の自分は教室に友達がいなくて.だからノートとか借りれないし前に座って一生懸命聴いてるわけですよ.それで真面目にノートとって聴いてたら,学部生の若いやつが「あ~なにキミ,なんかずげえマジメに出てんじゃん?」とか話しかけてきて.すっごい腹立つわけよ(笑).「ノート貸してくれん?」とか言ってくるから,(いや,なんだよこいつ)とか思って.「嫌です」って言ったけどね.
\end{document}