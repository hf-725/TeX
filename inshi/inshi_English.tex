\documentclass[dvipdfmx,uplatex]{article}
\usepackage{tosuustd2019, appendix, cleveref}
\DeclareMathOperator{\Int}{Int}
\newcommand{\smooth}{C^{\infty}}
\newcommand{\setinverse}[2]{#1 ^{-1}(#2)}
\DeclareMathOperator{\sign}{sign}
\DeclareMathOperator{\ind}{ind}
\crefname{theorem}{定理}{定理}
\crefname{proposition}{命題}{命題}
\crefname{lemma}{補題}{補題}
\crefname{definition}{定義}{定義}
\crefname{example}{例}{例}
\crefname{remark}{注意}{注意}

\begin{document}
I think the mapping degree theory is important because it gives us the way of thinking which is valid 
in many other theories. 

For a given map $f$ between compact connected manifolds, 
the mapping degree of $f$ is defined by counting the signed number of the preimage of a regular value of
$f$. We can show that this definition is independent of the choice of the regular value. 

An immediate corollary of the definition is that if the degree of $f$ is not zero, then $f$ is surjective. 
It means local information about $f$ sometimes tells us global information. This local-to-global
framework is important for many other disciplines of mathematics. 

In fact, the technique which is used when we show the independendence of the choice of the regular value
is applicable in the other situations. For example, the same idea is used when we show that the Morse
complex is actually a chain complex. 

The mapping degree is also characterized using the de Rham cohomology. From this point of view, the homotopy 
invariance of the mapping degree becomes trivial because of the homotopy invariance of the induced map 
from $C^\infty$ maps. From this we can learn the power of the algebraic technique in topology. 
\end{document}