\documentclass[dvipdfmx,uplatex]{jsarticle}
\usepackage{tosuustd2019, appendix, cleveref}
\DeclareMathOperator{\Int}{Int}
\newcommand{\smooth}{C^{\infty}}
\newcommand{\setinverse}[2]{#1 ^{-1}(#2)}
\DeclareMathOperator{\sign}{sign}
\DeclareMathOperator{\ind}{ind}
\crefname{theorem}{定理}{定理}
\crefname{proposition}{命題}{命題}
\crefname{lemma}{補題}{補題}
\crefname{definition}{定義}{定義}
\crefname{example}{例}{例}
\crefname{remark}{注意}{注意}

\title{院試体験記}
\author{hf\_725}
\date{2020年9月1日}
\begin{document}
\maketitle

\section{はじめに}

9月1日に東大数理の口頭試問が終わって予定されていた院試を無事すべて終えることができた.
まずはとにかく最後までやり切れてよかったというのが素直な感想である.
本稿は私の院試がどんな風であったかを記録として残すことを目的として書かれている.
人それぞれで体験した内容は違うだろうが,一例として参考になれば幸いである.

\section{受験大学および試験形態について}

今年は新型コロナウイルス感染症COVID-19の影響で試験の形態が大きく変わった.そこでまずは
私が受験した大学を記し,試験形態がどのように変わったのかについて述べる.

\subsection*{京都大学大学院理学研究科数学系先端コース\footnote{
事務的なことで必要があって大学に電話した際,間違って「理学系研究科」と言ってしまった.
そうしたら事務の方があからさまに慇懃な態度をとってきたので皆さんは気をつけよう.
}}

数学志望の人が京都大学を受験する場合,主に3つのコースがある.列挙すると
\begin{itemize}
    \item 数学系基盤コース
    \item 数学系先端コース
    \item 数理解析研究所
\end{itemize}
の3つである.これら3つは同時に志望できる(出願時に志望順位を書くことになっている).
数学系先端コースと数理解析研究所はかなりの狭き門なので数学系基盤コースにも出願しておいた
方がいいと思う(私はぼーっとしていて基盤コースには出願しなかったが後から馬鹿なことをした
と思った).

試験形態についてだが,例年は出願時に提出する志望調査書などの他に
\begin{itemize}
    \item 英語,基礎科目,専門科目という3種類の筆記試験
    \item 筆記試験の合格者に対する口頭試問
\end{itemize}
があり合否が決まる.今年に関しては次の変更点があった.
\begin{itemize}
    \item 出願時の書類に「あなたが重要だと思う定理についてのレポート」および「院試の
    基礎科目に相当するレベルの過去問の解答レポート」が加わる.
    \item 出願時の書類に基づいて1次合格発表が行われ,先端コース志望者には専門科目に相当
    するレベルの過去問についての解答レポートをその日のうちにオンラインで提出するよう
    求められる.
    \item 基礎的な知識を問う1次口頭試問がオンラインで行われる.
    先端コース志望者については,その中から
    十分な実力があると判断された人について専門的な知識を問う2次口頭試問が行われる.
\end{itemize}
気づいた方もいるかもしれないが,もともとあった英語の要素が今回の試験では完全に
欠落している.それでいいのかとも思うがきっとそれでいいんだろう.
例年と違うのは,院試が始まってから終わるまでが非常に長いことだった.例年であれば
1つの大学あたり長くて1週間で終わるが,今年は8月2日から始まって終わったのが8月25日であった.
そのかわりではないが,京都までわざわざ足を運ばずとも受験することができ,後述の理由から
私にとってはそれが大きくプラスに働いた.

\subsection*{東京大学大学院数理科学研究科}
この大学も例年であれば出願時の書類の他に
\begin{itemize}
    \item 英語,専門A, 専門Bの3つの筆記試験
    \item 筆記試験の合格者に対する口頭試問
\end{itemize}
が課せられ,合格者が決まる.専門Aは(知識としては)基礎的な内容が問われ,
専門Bは個々人の志望分野ごとの専門的な内容が問われる.対して今年の試験形態は
\begin{itemize}
    \item 出願時の書類に「読んだ本のリスト」「あなたが興味を持って勉強したことについての
    レポート」「興味を持って勉強したことについてのレポートの英語での要約」が追加される.
    \item 専門Aの筆記試験をオンライン形式で行う.
    \item 専門Aの試験が終了した時点で一度選抜を行い,合格者に対して専門Bの筆記試験と
    口頭試問を行う.
\end{itemize}
ということになった.専門Aの筆記試験は必答問題2問と選択問題2問に分かれていて,例年ならこれらを
合わせて3時間なのだが今年は必答と選択で時間が区切られていて1時間半ずつだった.ここは例年とは
戦略を変えなければならないところかもしれない.また,専門Bは3問を3時間で解く形式で,こちらは
例年と変更点はない.

見ていただいて分かったと思うが,信じられないほど面倒くさい.単純に
例年よりやるべきことが増えている.というか筆記試験やるならレポートの提出と化させる必要
なくないか?意味分からん.

また,京大と比べてみるとこちらはちゃんと英語の要素が残っている.しかしレポートの要約を
しろとか例年のぬるい和訳よりも難しいんじゃないか?語数指定もあるし大学入試かよ.

同期の友人が「教員たちの話し合いでもめた結果全部やろうということに
なったんじゃないか」と言っていて確かにそんな気がしてきた.許せない.

\section{私の持病「クローン病」について}
院試の話とは直接関係がなくなるが,このタイミングで私の持病である
クローン病についての話をしようと思う.私の院試体験記はどう考えてもこの病気
抜に語ることはできない.

「クローン病」という病名にに聞きなじみのない人がほとんどであると思うが,
この病気は小腸を中心として消化管全体に潰瘍ができる病気であり,
現在指定難病の1つである.8月28日に安倍総理が辞任を
表明する際,潰瘍性大腸炎
\footnote{ちなみに潰瘍性大腸炎を抱えながら活躍しているスポーツ選手にオリックス・バファローズの
安達了一選手がいる.もともと同じ名字で親近感があったが,自分もクローン病を抱えてから
彼の存在に励まされた.}
の再燃が原因という説明があった
\footnote{診断後すぐは病気のせいでこれからの人生で自分が輝ける可能性はなくなった
のではないかという思いを抱えていた(というか今もそうである).そんなとき
安倍総理が総理を長い間務めているという事実は(政策の是非はさておき)
私の希望になった.だから彼を病気のことで馬鹿にするような発言をする人間を私は絶対に
許さない.}
が,クローン病は潰瘍性大腸炎の
仲間であり,IBD(炎症性腸疾患)の名前でくくられることも多い.これらの病気はどちらも自己
免疫疾患であると考えられており,どちらも根治療法は確立されていない.
潰瘍性大腸炎は大腸に潰瘍ができることで腹痛や大量出血,腸管の狭窄が発生するが,クローン
病の場合は小腸の潰瘍がきっかけで同様の症状が出る.
どちらの病気も再燃と寛解を繰り返し,多くの人が狭窄がひどくなって手術することに
なるそうだ
\footnote{クローン病の場合は狭窄が起きている小腸の一部分を切り取ることが多く,あまり
切りすぎると短腸症候群といって栄養障害になるらしい.潰瘍性大腸炎の場合は大腸全摘らしいが,術後も潰瘍性大腸炎の症状から解放されるわけではないそうだ.}.

私は現在小腸に狭窄が少なくとも2か所ある.こういうとき食事内容は相当厳しく制限される.
クローン病患者の食事としてよく言われるのは「低脂肪・低残渣」であるが,これは常人が
思いつく制限よりずっと厳しい.「低脂肪」の食事については「ラーメンとかをやめろって
ことでしょ」と思うかもしれない.しかしそういうことではなく,
「脂が含まれている食品は口にするな」ということである.例えば牛乳でさえ飲むべきではないと
指導された.「低残渣」についても,のりやわかめすら食べられない(難溶性食物繊維が多く消化が悪い)
というありさまである.逆に何なら食べられるのかというと鶏肉のささみや脂の乗りが悪い
魚,米飯やうどんくらいである.私は今年の4月にクローン病であると診断されたが,実際
ここに書いたもの以外の食べ物はほとんど口にしていない.よく「IBDでも寛解期には普通の人と
同じ生活が送れて,普通の食事ができます」と言われるが実際は全くそんなことはない.昼ごはんに
のり弁を食べている友人を見ながら,あるいは夕飯にわかめの味噌汁を飲んでいる
家族を見ながら強烈な嫉妬を覚える生活である.

こういう苦境に立たされているときにやはり同じ境遇の人の存在は心強い.潰瘍性大腸炎を
抱えながら活躍している有名人はそこそこいるもので,「自分にもできるんじゃないか」という
気持ちを与えてくれる.先日水泳の池江璃花子選手が復帰戦に臨んで好タイムを出したという
報道があったが,そういう話も自分の糧になる.

また,こういうときに身近な人が助けてくれるのもありがたい.そういう人たちの助けがなかったら
今頃食べるものもなく餓死していてもおかしくないと思う.病気になってよかったとは1ミリも
思わないが,周囲の人の優しさに触れることができたのも事実である.

ちなみに潰瘍性大腸炎を抱える著名人は多いがそれに比べるとクローン病を抱える著名人は
少ないように感じる.私のぼんやりとした目標は私自身がクローン病を抱えながらも活躍をする
人間になることである.
\section{8月 院試本番}
\subsection{8月2日 京大書類合格発表}
院試の開始を告げたのが京大の書類合格発表だった.合格通知とともにこの日のうちに解かなければならない
レポート問題も出題された.

実はこの週が8月中でも最も体調の悪い時期だった.7月の下旬ごろから食後の腹痛が強くなるようになり,
それが原因で眠れない日も増えていたのだが,この日はあまりに体調が悪くてそもそも食事ができなかった.
流動食なら大丈夫だろうと思い,inゼリーを片手になんとか問題を解ききり正午ごろには解答PDFを大学に
提出した.

結局この後もなかなか体調が戻らず何回か病院に駆け込んだ
\footnote{そうしたら2回目くらいに医師にあしらわれた.}.
家族で下した決断は,体調維持のため院試が終了するまで固形の食事は摂らないということだった.
というよりそうしないと腹痛以上のひどい症状が出るのでそうせざるを得なかった.
8月8日あたりからエレンタール(ほとんど消化済みの栄養剤で,とてもまずい)とゼリーだけを摂取する
生活が始まった.

\subsection{8月20日 東大一次筆記試験}
この日まで前述の体調管理を徹底した結果,比較的体調は安定していた.それでも体がだるいことには
変わりないし,散歩ができるほどの体調ではなかったので体力はかなり落ちていたと思う.
午前中の必答問題パートを完答して「昼食」のエレンタールを飲んだ.午後の選択問題パートは,
問題をパッと眺めたところすぐに解けそうな問題がなくて少し焦った.それでも「これは時間をかければ
解けるはず」と思った解析の問題を解き,常微分方程式の問題を「一般解ふってこねえかな」と思っていたら
本当にふってきてくれたので2問解ききることができた.

この日の午後は非常にタイミングのいいことにGoogleが全国的に通信障害を起こした日だった.
選択問題の解答をGoogleフォーム経由で送ろうとしても全然送れないので「連絡あるまで何もしないで
だらだらしとこ~」と思っていたのだが,すると本部から「解答をメールで送信してください」と連絡があった.
しかしメールで送信しようにもGmailだってサーバー落ちしているんだからファイルが添付できない.
あれこれ頑張って携帯のメールアドレスから送信すると,1時間後に「実は東大数理のメールアドレスは
Googleのサーバーを使っているのでメールが開けないんです,Googleフォームが復旧したらまたそっちから
送ってね~」という連絡が来てかなり面白かった.

\subsection{8月22日 京大一次口頭試問}
一人45分の時間設定で行われると事前にアナウンスされていて,私はこの日の10時半から12時15分の間で
行うと言われていた.しかしながら11時40分を過ぎてもなかなか連絡がこない.あまりに来ないので
リビングにいる母親に「もしかしたらぼく足し算できなくなっちゃったのかもしれないんだけど,11時40分から
さらに45分経ったら12時15分過ぎちゃうよね……?」という話をしていた.結局試問開始の
連絡が来たのは11時50分だった.本人確認をした後,「これから3題解いてもらい,解きながら説明を
してもらいます.内容は線型代数10分,微積分10分,多様体
\footnote{これはおそらく私が幾何志望だったからで,代数志望だったら環論とかを聞かれるのかもしれない.}
15分です」と言われた.この瞬間に「ああ,この時間設定だったら確実に時間押していくわ……」と察した.
どんなに簡単な問題だったとしても説明をしながら解くということは容易ではないし,院試という特殊な環境では
誰しも緊張するのでなおさらだ.これをやりたいなら本来一人1時間くらいは確保すべきだったのではないかと思う.

結果的には線型代数と微積分はすらすら解けた.まあ確かに簡単で,もしも家で演習しているなら一問5分で十分
すぎるようなレベルだった.多様体に関しては陰関数定理を使ってある関数の1点の逆像が部分多様体であることを
示す問題だったが,その点が正則値であることを示すのが少し難しくて手間取っていたところ,「時間もないですし
とりあえずそこまででいいです,どんな定理を使おうとしたかだけ教えてください」と言われ陰関数定理について
言及したところそれでいいということになって試問が終了した.

結局試験時間は35分くらいだったらしい.これですら12時15分を過ぎているのでやはり初めの時間設定は
甘かったと思う.

\subsection{8月25日 京大二次口頭試問}
23日の夜に2次試問への進出者が発表され,見事合格していた.二次試問は8月2日のレポート,および今まで
勉強してきた発展事項について質問されるということで,試問時間も1時間30分と長丁場の設定だった.

自分の試問の順番を待ちながら,そういえば京大の先生方の名前が誰一人として分からないことに気づく.
慌ててホームページを開き,自分の志望と関係のありそうな先生の名前を確認して頭に入れた.
そうして試問が始まると,Zoomミーティングの中にさっき調べた先生の名前が写り「うわやべえまじでこの先生
来たわ」と一人で勝手に焦っていた.自分の志望と被っているのだからいるに決まっているのだが.

はじめに今まで勉強してきたことについて聞かれた.MilnorのMorse Theoryを読んだと事前に回答していた
のでそのことについて聞かれたのだが,「さすがに復習しなくてもこたえられるでしょ~」とのんきに考えていたため
いざ説明を始めるとだんだん自信がなくなってきて,少しずつ思い出しながら丁寧に説明した.すると
「あの,あまり時間もないので細かいところはいいです」と言われ,ラッキー♪と思いながら適当な
説明を始めた.割とそれでも大丈夫な感じだった.

次に「Audin-Damianを講究で読んでいるようですが,シンプレクティック幾何はどの程度勉強しましたか」と
聞かれ,「まだDarbouxの定理とか基本的なことを勉強し始めたばかりです」と話したらそれ以上追及されなかった.
「ではAudin-DamianのPart Iを読んだとのことですが,Morseホモロジーの定義について説明してください」と
言われ,通り一遍の説明をしたところやや突っ込んだことについての質問をされたのでやや突っ込んだ内容の
回答をした.

次にレポート問題についての話に移った.2題解かされていたのだが,2題とも「よくできています」という評価を
いただいた.それぞれについて関連事項の質問をされたが,1問目について追加の問題を出され,なかなか手こずって
いると色々とヒントを出していただいて,なんとか最後まで解ききった.その後
\begin{quote}
  先生「ですからこの問題は~~という意味があるわけですね」

  私「なるほど~」
\end{quote}
というやり取りが交わされ,「もしかしてこれセミナーなのかな」という思いが芽生え始めた.
2つ目の問題はホモロジーの問題だったが,そこで使った切除定理について説明してくださいという質問があり,
もう一つ
\begin{quote}
  先生「ここは局所ホモロジーとよばれるものを考えているわけですが,これは局所系をなすわけですね,
  何の局所系かというのは分かりますか……?」

  私「向き,でしょうか?」

  先生「ああはいはいそうです,すみません頓智みたいな質問で」

  私「いえいえ(笑)」
\end{quote}
という和やかなやり取りがあった.今考えると先生方も受験生が緊張しないような配慮を色々としていただいて
いたのではないかと思う.思い返すと言葉の端々から優しさが見て取れた.
「2問目の方も追加の問題を用意していたのですが私としては聞かなくてもいいかと思います,反対の先生は
いらっしゃいますか?」という問いかけに全員賛成という様子で,「最後に個別に質問のある方」という問いかけには
2つ反応があった.一つは「Chern-Weil理論を勉強なさったとのことですが,第一Chern類が消えない例を教えてください」
と言われ,「$\setC P^1$です」と答えると「それの……?」と言われ,「あ,tautological line bundle」
と答えると「それでいいです」ということになった.正直tautological line bundleの第一Chern類が消えない
ことの証明がちゃんと頭に入っていなかったのでラッキー♪と思った.二つ目は「Morse理論を勉強されたと
おっしゃっていましたが,これの複素版もありますよね,Picard-Lefschetz theoryというものですが,
これとシンプレクティック幾何のつながりはどうお考えですか」という質問が飛んできて,(あれ,さっき
シンプレクティック幾何は勉強し始めたばかりって言わなかったっけ)と思いながら,Picard-Lefschetz理論も
勉強したことがなかったので「不勉強なものでどちらも分かりません」と言ったところ「分かりました」で
終了した.

試問時間は1時間15分だった.

\subsection{8月31日 東大二次筆記試験}
この日が一番緊張して眠れなかった.相変わらず体がだるいのが続いていた中で睡眠不足になったので
かなり不安な立ち上がりだった.しかも普段だったら11時試験開始という人間に優しい設計だったのに今回は
9時開始という鬼の所業をしてきた.

私は幾何が専門なので幾何の問題を中心に解くつもりだった.たまに解析の問題が簡単なので一応解析も覗く
ことにしていたが,今年のラインナップを見た瞬間解析の方はそっとページを閉じた.
例年1問くらいは簡単な問題があるもので,今年は5番がそれだった.にもかかわらず最後の小問がなかなか
解けず,結構焦った.いったん切り替えようと思って6番のホモロジーの問題を見てみたが全く解ける気がしなかった.
7番の多様体の問題は少なくとも一部分は簡単に解けそうだったので解答を書き始めてみた.
8番は初めの方は簡単そうだったが最後がめっちゃ難しそうだったので開いた瞬間あきらめた
\footnote{いまだに解けていない.}.

7番が一通りできるところまで終わったので5番に戻ってもう一度冷静に考えてみると10分くらいで
解くことができた.この時点で試験時間4時間のうち1時間ほどが経っていたと思う.6番をもう一回見てみたが
やはり全然分からないのでパス.7番の(3)が例の構成で,(2)が分からなくても思いつくのではと思って
少し考えてみると運のいいことに割と早く見つかった.1完半くらい取れれば十分と考えていたので,ここからは
気楽に受けようとマインドを切り替えた.

(2)が分からないので6番をもう一回見てみることにした.すると三度目の正直,
初めの一歩を踏み出すことができた.その一歩のおかげでその空間に胞体構造を入れることが
でき,あとはルーティンのホモロジー計算をするだけになった.この大門は(1)が難しく(2)が簡単なタイプで,
この問題も解ききることができた.ここまでで2時間半.

最後にもう一回7番の(2)を眺めてみた.どうやら局所的な情報だけではこの問題が解けないみたいだな~と思いながら
与えられた式を眺めていると「あれ?これ写像度じゃん!」ということに気づいた.するともうするするとすべてが
分かり,自宅受験で他に人がいないので「ははーなるほど~」などとわざとらしくつぶやいてから
解答を丁寧に書いた.これが終了したのはだいたい試験開始3時間15分くらいだった.

もうそこからは余裕の表情をしながら,まずはリビングに行き,水を汲んで一杯飲んだ.
ソファにもたれかかり,「ふぅー」と息をついてから,見直しでもするかと思い自室に戻って見直しをした.
3つくらい誤植が見つかった.

見直しが終わったころの時間に試験も終了し,解答を提出した.その後解答を確認していると,
誤植がまだもう一か所あることに気がついた.

\subsection{9月1日 東大口頭試問}
正直筆記がことごとくうまくいったので余裕しゃくしゃくの気持ちで試験を受けた.私は全受験生の中で
トップバッターだった.受験者はあらかじめ指定されたZoomミーティングルームに15分前に入室し,試験が
始まるまで待機室機能で待つことになっていた.もちろん私もそうしたのだが,するとまだ向こうの準備が
整っていなかったらしく普通の部屋に通された.メンバーを確認するとホスト役の先生しかいなかったので,
「もしかして先生方遅刻ですか??」と失礼なことを考えていると待機室に戻され,試験時間まで
そこで待機していた.

定刻通りに口頭試問が始まった.そこにいた先生は
\begin{itemize}
  \item 入江慶准教授(講究の指導教官)
  \item 植田一石准教授
  \item 金井雅彦教授
  \item 古田幹夫教授
  \item 松尾厚准教授(共同ホスト)
  \item 吉野太郎准教授(ホスト)
\end{itemize}
だった(50音順,肩書は当時).

初めは多様体論についての基礎について質問された.具体的には多様体の間の$C^\infty$級写像の微分の定義,
正則値の定義,そして陰関数定理の主張である.初め2つは問題なく答え,陰関数定理については
「正則値の逆像が部分多様体になるという定理です」と説明した.これに対し古田先生が「陰関数定理について
ですが,これを大学1年生にも分かるように説明してくださいますか」と追加の質問をなさった.
まずは大学1年生が持っている知識を確認して,杉浦光夫先生の解析入門IIに載っているような主張を
書くことにした.が,書き始めてみるとどこか条件が足りないことに気がついた.どう修正すればいいか
分からない.すると古田先生があの落ち着いた美声で,しかしいつもよりトーンの低い様子で「あの,初めから
答えを書こうとしなくていいです.書きながら考えてください」とおっしゃり,別の先生(おそらく松尾先生?)が
「たとえば具体例で説明していただけますか」とおっしゃったので円周を定義する関数を例に考えた.

それをもとに一般の場合での主張の修正方法が思いついたので,「証明はぱっと思いつきませんがこのように修正
すればいいと思います」と言った.修正前の問題点と修正方法について整理して話すよう言われたのでそれを説明
すると,古田先生に「証明はぱっとできないということですが,今おっしゃった主張で正しいという確信は
ありますか」と訊かれたので「はい,あります.ただ証明しろと言われると30分くらい時間をいただきたいです」と
回答し,この話は終わった.少ししくじったかなという思いを持った.

次に京大のときと同じくMilnorのMorse Theoryについて聞かれた.
京大のときの反省を生かして事前に復習していたし,どういう塩梅で話せばいいのかもある程度分かっていたので
結構スムーズに答えられたと思う.「追加で質問のある先生は?」という問いかけに入江先生が反応した.
\begin{quote}
  入江先生「エネルギー汎関数に言及されていましたが,その定義を教えてください.」

  私「~~~が定義です.」

  入江先生「その汎関数の臨界点はどういう意味を持つものですか?」

  私「ちょうど測地線に一致します.」
\end{quote}
というやり取りだった.入江先生の質問は,「答えられなければこの本を読んだとはいえない」というレベルに
易しい質問だったので,もしかしたらフォローをするつもりで質問してくださったのかもしれない.

再び「質問のある先生は?」という言葉に松尾先生が反応した.
\begin{quote}
  松尾先生「レポートに写像度をde Rhamコホモロジーで特徴づけるということを書いていましたが,
  それについての詳細を説明してください.」

  私「はい.これはまさしく昨日の第6問の話題ですが(分かってるぞアピール),de Rhamを使うと~~という定義
  ができます.」

  松尾先生「今書いているのはレポートに書いてあるのとは違うと思うのですが.」

  私「?すみません,一度レポートを見てもいいでしょうか.」

  松尾先生「どうぞ」

  私「ありがとうございます.(レポートを見る)あ,ここの条件の抜けですか?」

  松尾先生「はい」

  私「(抜けてるとまでは言い切れないと思うけどな)確かに条件不足でした,すみません.訂正します.」

  松尾先生「そうですよねえ.」

  私「(ん?)」

  松尾先生「あとこれはsingularでも定式化できますよね.」

  私「はい,~~~と定義することもできます.」

  松尾先生「それとde Rhamとの関係性を説明してください.」

  私「それは定義の等価性を説明するということでいいですか?」

  松尾先生「はい.」

  私「分かりました.それを示すには,もう一つ別の方法で定義されたものと一致することを見ることで示せます.
  詳細を説明した方がいいでしょうか.」

  松尾先生「そうしてください.」

  私「(ある程度説明して)~~というストーリーです.これ以上詳しく説明した方がいいでしょうか.」

  松尾先生「私が聞いているのはsingularとde Rhamの定義の関係なのですが.」

  私「それは別のもう一つの理論と結びつけることで説明しました.」

  松尾先生「(こちらの反応には応えず)手短に説明できないのならこれでやめますが.」

  私「(内心とても苛立ちながら)singularの方もコホモロジーで考えるべきなのですが,そうするとsingular
  とde Rhamは関手として一致するので写像度の定義も一致します.」

  吉野先生「松尾先生,以上でよろしいですか?」

  松尾先生「はい,結構です.」
\end{quote}
みたびの「他に質問は?」に誰も応えなかったため,「少し早いですが終わりにしましょう」ということで
終了した.

全部で45分の試問だった.なんだかもやもやを抱えたまま試験が終了した.

\section{9月 結果発表}
結果発表は9月中旬だった.その間の期間に小腸内視鏡検査をするため数日入院した.
\begin{quote}
  \textbf{京都大学大学院理学研究科数学系先端コース} 合格.

  \textbf{東京大学大学院数理科学研究科} 合格.

  \textbf{小腸内視鏡検査} 全快ではないものの,発症時に比べると改善が見られる.
\end{quote}

\section{結びに}
結果的に最高の結果で試験を終えることができた.8月は体調面でかなりつらい日々を過ごしたが,
病勢も弱まっているということで苦労が少し報われた思いがする.体調管理について非常に気を使ってくれた
両親(特に母親)には大変感謝している.

また,試験対策は富永直弥と協力して行っていた.彼のおかげで気づいた新しい解法もあり,実りの多い
試験対策ができたと思う.この場を借りて感謝したい.

最後に後輩にメッセージを送る.試験対策は遅くとも4月からは始めるべきである.特に最新年度の
過去問をきちんと解き,難易度を把握しておくのが賢い.「院試は大学受験までとは違うから対策は軽めでいいかな」
と思う人もいるだろうが,きちんと対策をしておいた方がいいに越したことはない.問題をいくつか眺めれば,
日頃の勉強を積み重ねることである程度は問題が解けることに気づくと思う.あとはゆっくり演習を重ねて
解ける問題を増やしていけばよい.決して6月の終わりに問題を初めて見てその難しさに慌てるようなことはしては
いけない.

また,いつ何時自分が難病患者になるとも知れない.私は傲慢な人間であるのでこのような身分になるまで
人の苦しみを理解することに目が向いていなかった.皆さんにはできるだけこのような人の存在を身近に感じ,
共感はしなくとも理解を示す人物になっていただきたいと切に願っている
\footnote{非常に上から目線な物言いになってしまいすみません.}.

\end{document}

試験形態について
クローン病について
出願について
8月に入ってからの動向について