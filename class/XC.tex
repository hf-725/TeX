\documentclass{jsarticle}
\usepackage{latexsym,amsmath,amsthm,amssymb,enumitem,color}
\usepackage{type1cm}
\theoremstyle{definition}
\newtheorem{theorem}{定理}[section]
\newtheorem{definition}[theorem]{定義}
\newtheorem{corollary}[theorem]{系}
\newtheorem{proposition}[theorem]{命題}
\newtheorem{lemma}[theorem]{補題}
\newtheorem{example}[theorem]{例}
\newtheorem{remark}[theorem]{注}
\renewcommand{\proofname}{\textgt{証明}}
\newcommand\triv{\varepsilon_B}

\title{計算数学Ⅰ課題 幾何学XCノート}
\author{安達 充慶}

\begin{document}
\maketitle
\section{ベクトル束の定義}
\begin{definition}\label{vbdl}
$B$を位相空間,$n$を非負整数とする.$B$上の次元$n$のベクトル束とは,
\begin{enumerate}
\item 位相空間$E$
\item 連続写像$\pi \colon E \to B$
\item 各$b \in B$に対する$E_b := \pi^{-1}(b)$上のベクトル空間の構造
\end{enumerate}
の組($E$,$\pi$)であって次の条件(局所自明性)を満たすもののことをいう.すなわち,すべての$b \in B$に対して$b$の開近傍$U$と同相写像$h \colon U \times \mathbb{R}^n \to \pi^{-1}(U)$の組$(U,h)$が存在し,次を満たす.
\begin{enumerate}[label=(\arabic*)]
\item $p \colon U \times \mathbb{R}^n \to U$を$U$への射影とするとき$\pi \circ h = p$である.
\item 各$u \in U$に対して$h_u \colon \mathbb{R}^n \to E_b$を$h_u(x) = h(u,x)$で定めると,これは線型同型写像である.
\end{enumerate}
\end{definition}

\begin{remark}
\begin{enumerate}[label=(\arabic*)]
\item 上の定義において,$E$を全空間,$B$を底空間,$\pi$を射影,$E_b$を$b$上のファイバーという.また,$(U,h)$を局所自明化という.
\item $\mathbb{R}$を$\mathbb{C}$にとりかえたものを複素ベクトル束という.
\item $1 \leq r \leq \infty$に対して$E$,$B$を$C^r$級多様体,$\pi$を$C^r$級写像,$h$を$C^r$級微分同相写像にしたものを$C^r$級ベクトル束という.
\item しばしばベクトル束を一つのギリシャ文字$\xi$などで表し,その全空間を$E(\xi)$,射影を$\pi(\xi)$と書く.$\xi = (E(\xi),\pi(\xi))$である.
\end{enumerate}
\end{remark}

以降は単にベクトル束といえば定義\ref{vbdl}で定められたものを指すことにする.

\begin{definition}
$\xi_1$,$\xi_2$を$B$上のベクトル束とする.$\xi_1$から$\xi_2$への同型写像とは,同相写像$f \colon E(\xi_1) \to E(\xi_2)$であって次の二つの条件を満たすもののことである.
\begin{enumerate}[label=(\arabic*)]
\item $\pi(\xi_2) \circ f = \pi(\xi_1)$.
\item すべての$b \in B$で$\left.f\right|_{E(\xi_1)_b} \colon E(\xi_1) \to E(\xi_2)$は線型同型写像である.
\end{enumerate}
\end{definition}

\begin{example}
$B$上のベクトル束$\triv^n$を次で定める.
\begin{enumerate}[label=(\arabic*)]
\item $E(\triv^n) = B \times \mathbb{R}^n$.
\item $\pi(\triv^n) \colon B \times \mathbb{R}^n \to B$,\ $\pi(\triv^n)(b,x) = b$.
\item $E(\triv^n)_b = \{b\} \times \mathbb{R}^n$上のベクトル空間の構造を次で定める.
\[ (b,x) + (b,y) = (b,x+y),\ r(b,x) = (b,rx). \]
ただし,$r \in \mathbb{R}$である.
\end{enumerate}
$\triv^n$を$B$上の$n$次元積束という.$\triv^n$と同型なベクトル束を自明束という.
\end{example}

以下,自明なベクトル束の特徴づけを与える.

\begin{definition}
$\xi$を$B$上のベクトル束とする.
\[ \Gamma(\xi) = \{s \colon B \to E(\xi) \mid s \mbox{は連続写像かつ} \pi \circ s = \operatorname{id}_B \} \]
の元を$\xi$の切断という.また,$s_0 \colon B \to E(\xi)$を
\[ s_0(B) = 0_{E(\xi)_b} \]
で定義するとこれは連続である.$s_0$を$\xi$の零切断という.
\end{definition}

\begin{proposition}\label{chartriv}
$\xi$を$B$上のベクトル束とする.$\xi$が自明であることの必要十分条件は,切断の族$s_1,s_2,\dots s_n \in \Gamma(\xi)$が存在し,各$b \in B$に対して$s_1(b),s_2(b),\dots s_n(b)$が$E(\xi)_b$の基底になることである.
\end{proposition}

証明の前に補題を用意する.

\begin{lemma}
$\xi_1=(E_1,\pi_1)$,$\xi_2=(E_2,\pi_2)$を$B$上のベクトル束とする.$f \colon E_1 \to E_2$が
\begin{enumerate}[label=(\arabic*)]
\item $\pi_2 \circ f = \pi_1$,
\item すべての$b \in B$に対して$\left.f\right|_{(E_1)_b} \colon (E_1)_b \to (E_2)_b$は線型同型写像
\end{enumerate}
を満たせば$f$は同相写像であり,したがって同型写像である.
\end{lemma}

\begin{proof}
すべての$b \in B$に対して$b$の開近傍$U$が存在し,$\left.f\right|_{\pi_1^{-1}(U)} \colon \pi_1^{-1}(U) \to \pi_2^{-1}(U)$が同相写像であることを示せばよい.$b$を含む$\xi_1$,$\xi_2$の局所自明化$(U,h_1)$,$(U,h_2)$をとり,$U=U_1 \cap U_2$とおく.$g=h_2^{-1} \circ f \circ h_1 \colon U \times \mathbb{R}^n \to U \times \mathbb{R}^n$とおくとこれは連続写像で,さらにこれは連続写像$A \colon U \to GL_n(\mathbb{R})$を用いて
\[ g(u,x) = (u,A(u)x) \]
と書ける.$g$の逆写像が連続であればよい.ところで,$F \colon GL_n(\mathbb{R}) \to GL_n(\mathbb{R})$を$F(X)=X^{-1}$で定めるとこれは連続写像である.したがって
\[ g^{-1}(u,x) = (u,F(A(u))x) \]
は連続写像である.
\end{proof}

\begin{proof}[命題\ref{chartriv}の証明]
補題から,連続写像$f \colon B \times \mathbb{R}^n \to E(\xi)$で$\pi(\xi) \circ f = \pi(\triv^n)$かつすべての$b \in B$について$\left.f\right|_{\{b\} \times \mathbb{R}^n} \colon \{b\} \times \mathbb{R}^n \to E(\xi)_b$が同型であるものの存在との同値性をいえばよい.

$\xi$が自明なら,$f \colon B \times \mathbb{R}^n \to E(\xi)$を同型写像として,$1 \leq i \leq n$に対して$s_i(b)=f(b,e_i)$とおけばよい.ただし$e_1,e_2,\dots e_n$は$\mathbb{R}^n$の標準基底である.

逆に題意の$n$個の切断があるとき,
\[ f\biggl( b,\sum_{i=1}^n t_ie_i\biggl) = \sum_{i=1}^n t_is_i(b) \]
とおけばよい.
\end{proof}
\end{document}











































