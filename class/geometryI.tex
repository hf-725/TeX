\documentclass{jsarticle}
\usepackage{latexsym,amsmath,amsthm,enumitem,amssymb,color}
\usepackage{type1cm}
\theoremstyle{definition}
\newtheorem{theorem}{定理}[section]
\newtheorem{definition}[theorem]{定義}
\newtheorem{corollary}[theorem]{系}
\newtheorem{proposition}[theorem]{命題}
\newtheorem{lemma}[theorem]{補題}
\newtheorem{example}[theorem]{例}
\newtheorem{remark}[theorem]{注}
\renewcommand{\proofname}{\textgt{証明}}
\newcommand{\euclid}[1]{\mathbb{R}^{#1}}
\DeclareMathOperator{\Ker}{Ker}
\DeclareMathOperator{\Image}{Im}


\begin{document}
\title{計算数学I課題 幾何学I講義ノート}
\author{安達 充慶}
\date{}
\maketitle

\section{逆写像定理とEuclid空間の部分多様体}
まず,微分同相について復習する.$n,m,l \in \mathbb{Z}_{\geq 0}$とする.この講義では$\mathbb{R}^n$は原則として縦ベクトルの空間とし,Euclidノルム
\[ \| x\|=\sqrt{\sum_{i=1}^n x_i^2} \]
によって位相が定まっているものとする.$M(n,m;\mathbb{R})$で$(m,n)$型の行列全体の集合を表すことにする.
これを$\mathbb{R}^{mn}$と同一視することにより$M(n,m;\mathbb{R})$にノルム空間の構造を与える.このとき,$A\in M(n,m;\mathbb{R})$と$x\in \mathbb{R}^n$について
\[ \| Ax\| \leq \| A\|\| x\| \]
が成立する.

\begin{definition}
$r=0,1,2,\dots \infty$とする.$O^in \mathbb{R}^n$を開集合とする.$F\colon O \to \mathbb{R}^m$が$C^r$級写像であるとは,$F$を成分表示して
\[ F(p)=(f_1(p),\dots,f_m(p) \]
と書いたとき,各$f_i\colon O\to \mathbb{R}$が$C^r$級関数であることと定める.
\end{definition}
以下,$r\geq 1$であるとする.$C^r$級写像$F$に対してそのJacobi行列$JF\colon O\to M(n,m;\mathbb{R})$を
\[ (JF)_p=(\frac{\partial f_i}{\partial x_j}(p))_{1\leq i\leq m,1\leq j\leq n} \]
で定める.$u\in \mathbb{R}^n$に対して
\[ (JF)_p u=\left.{\frac{d}{dt}}\right|_{t=0} F(p+tu) \]
であることに注意する.

\begin{lemma}\label{chain}
\begin{enumerate}[label=(\arabic*)]
\item $A\in M(n,m;\mathbb{R})$を自然に$\mathbb{R}^n$から$\mathbb{R}^m$への線型写像とみなすと,すべての$p\in \euclid{n}$で$(JF)_p=A$が成立する.特に,$(J1_{\euclid{n}})_p=I$である.
\item $O$を$\euclid{n}$の,$O'$を$\euclid{m}$の開集合とする.さらに$F\colon O\to \euclid{m}$,$F'\colon O'\to \euclid{s}$を$C^r$級写像とすると$F'\circ F\colon F^{-1}(O')\to \euclid{s}$は$C^r$級写像であって,すべての$p\in F^{-1}(O)$に対して
\[ (JF'\circ JF)_p=(JF')_{F(p)}(JF)_p \]
が成立する.
\end{enumerate}
\end{lemma}

\begin{definition}\label{diffeo}
$O$を$\euclid{n}$の,$O'$を$\euclid{m}$の開集合とする.$F\colon O\to O'$が$C^r$級微分同相写像であるとは,$F$が$C^r$級の全単射であり,さらに逆写像$G\colon O'\to O$が$C^r$級写像であることである.また,このとき$O$と$O'$は微分同相であるという.
\end{definition}

\begin{example}
恒等写像,Euclid空間における平行移動や線型変換は微分同相写像である.
\end{example}

\begin{lemma}
定義\ref{diffeo}の状況で,$(JF)_p$は$(JG)_{F(p)}$を逆行列に持つ.特に$O$が空集合でないとき,$n=m$である.
\end{lemma}

\begin{proof}
補題\ref{chain}より
\[ (JG)_{F(p)}(JF)_p=I_n,\  (JF)_p(JG)_{F(p)}=I_m \]
が成立する.
\end{proof}

この補題は大域的には成立しない(複素数に拡張された指数関数を考えよ).しかし,局所的には逆が成立する.

\begin{theorem}[逆写像定理]\label{inverse}
$O$を$\euclid{n}$の開集合とする.$C^r$級写像$F\colon O\to \euclid{n}$の$p_0\in O$でのJacobi 行列が正則ならば,$O$に含まれる$p$の近傍$U$と$F(p)$の近傍$V$が存在して,$F(U)=V$かつ$F\colon U\to V$は微分同相写像となる.
\end{theorem}

この定理を認めていくつかの結果を証明する.以下では$r=\infty$であるとする.$O$を$\euclid{n}$の開集合,$F\colon O\to \euclid{m}$を$C^\infty$写像とする.

\begin{definition}
$n=m$とする.任意の$p\in O$に対してその開近傍$U_p$と$F(p)$の開近傍$V_p$が存在して$F(U_p)=V_p$かつ$F\colon U_p\to V_p$が微分同相写像であるとき,$F$は局所微分同相であるという.
\end{definition}

\begin{theorem}[陰関数定理]\label{implicit}
$0\leq m\leq n$,$0\in O$とする.さらに$F(0)=0$で$(JF)_0$の階数は$m$であるとする.このとき,$0$の開近傍$U\in O$と$V$および微分同相写像$\varphi\colon U\to V$が存在して,
$\varphi(0)=0$,$F\circ \varphi^{-1}(y_1,y_2,\dots, y_n)=(y_1,y_2,\dots y_m)$をみたす.
\end{theorem}

\begin{proof}
$F$を成分表示して
\[ F=\begin{pmatrix} f_1 \\ \vdots \\ f_m \end{pmatrix} \]
と表示する.仮定より,$1\leq k_1 < k_2 < \dots < k_m \leq n$が存在して,
\[ \mathrm{det}\left(\frac{\partial f_i}{\partial x_{k_j}}\right) \neq 0 \]
をみたす.適当な線型変換により,$k_1=1$,$k_2=2$,$k_m=m$であるとしてよい.$\hat{F}\colon O\to \euclid{n}$を
\[ \hat{F}\begin{pmatrix} x_1 \\ x_2 \\ \vdots \\ x_n \end{pmatrix} = \begin{pmatrix} F(x) \\ x_{m+1} \\ \vdots \\ x_n \end{pmatrix} \]
とおくと,$(J\hat{F})_0$は正則行列である.したがって逆写像定理により0の開近傍$U\in O$と$V$が存在して$\hat{F}\colon U\to V$は微分同相写像になる.そこで$\varphi=$を$\hat{F}$の$U$への制限とする.あとは$F\circ \varphi^{-1}$を計算すればよいが,
\[ \begin{pmatrix} y_1 \\ \vdots \\ y_m \\ y_{m+1} \\ \vdots \\ y_n \end{pmatrix} = \hat{F}\circ \varphi^{-1} \begin{pmatrix} y_1  \\ \vdots \\ y_n \end{pmatrix} 
= \begin{pmatrix} F\circ \varphi^{-1}\begin{pmatrix} y_1 \\ \vdots \\ y_n \end{pmatrix} \\ \\ \ast \\ \  \end{pmatrix} \]
 なので$F\circ \varphi^{-1}(y_1, \dots, y_n)=(y_1, \dots y_m)$である.
\end{proof}

\begin{definition}
$0\leq l \leq m$,$M\subset \euclid{n}$とする.各$p\in M$について開近傍$U$と$0\in \euclid{n}$の開近傍$V$,微分同相写像$\varphi\colon U\to V$が存在して
\[ M\cap U=\varphi^{-1}(\{0\}\times \euclid{l}) \]
が成立するとき,$M$は$\euclid{n}$の$l$次元$C^\infty$級部分多様体であるという.
\end{definition}

\begin{corollary}\label{submfd}
$0\leq m \leq n$,$O\subset \euclid{n}$を開集合,$q_0\in\euclid{m}$とする.$f\colon O\to \euclid{m}$について,すべての$p\in F^{-1}(q_0)$で$(JF)_p$の階数が$m$ならば$F^{-1}(q_0)$は$\euclid{n}$の$(n-m)$次元$C^\infty$級部分多様体である.
\end{corollary}

\begin{example}
$n$次元球面
\[ S^n=\{x\in \euclid{n+1} \mid \|x\|=1 \} \]
は$\euclid{n+1}$の$n$次元$C^\infty$級部分多様体である.実際,$f\colon \euclid{n+1}\to \mathbb{R}$を
\[ f(x) = \|x\| \]
で定めると$S^n=f^{-1}(1)$であり,$(JF)_x$の階数が1にならないのは$x=0$のときだけなので系\ref{submfd}が使える.
\end{example}
\end{document}



































