\documentclass{jsarticle}
\begin{document}

\title{人間行動基礎論レポート}
\author{J4-171051 安達充慶}
\date{}
\maketitle

\section{意思決定の概要}
中学から高校に上がるとき、パソコンを自作した。このパソコンはかなりハイスペックに作ってあり、15万円ほどかけて作った。CPUはそのころ2番目に高いモデルにして、HDDも3TBのものを買った(その上に256GBのSSDも積んでいる)。しかしながら当時自分がパソコンを使う目的はネットサーフィンとテレビ視聴であり、これほどのスペックを用意する必要はなかった。しかも15万円と言えば(今でもそうだが)当時の自分としては相当の痛手だったはずである。費用対効果を考えれば、自分の選択は非合理的なものであったと考えるしかない。
\section{分析}
この選択に至った理由はかなり明白で、私自身のパソコンに関する経験にある。私の家では10年近く低スペックなパソコンを共用で使っていて、購入してから数年経ったころには起動もままならない状態だった。それにいらだちを募らせていたため、その反動で必要以上のハイスペックを追い求めるようになっていた。特に共用パソコンが重くなった原因が記憶容量の小ささにあると感じていたことから記憶容量の設備にはかなり力を入れていた。


これは感情的な衝動が理性的な判断を上回った典型的な例で、実際このパソコンを作ってから「重いパソコンを使っていた時のフラストレーション」からは完全に解放された上に、今に至るまでこのハイスペックパソコンを作ったことを後悔したことは一度もない(それほどに昔のパソコンへのいらいらは大きかった)。
\section{補足}
ちなみにパソコンを買って後悔していないのには「実際にこのハイスペックが役に立ったから」でもある。私は高校二年生のときアニメオタクになり、パソコンで大量のアニメを録画するようになった。30分で5GBくらいの容量を食うため、今日まで3TBの大容量は大活躍している。また、同じく高校二年生のころに部活の用事で詰将棋を作らなければいけないことになり、余詰め(作意手順以外の詰め手順)がないかを将棋ソフトに確認させるときにハイスペックCPUが役に立った。今の自分から振り返ってみると、案外非合理的な判断でもなかったのかもしれない。 

\end{document}