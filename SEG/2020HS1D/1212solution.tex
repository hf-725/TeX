\documentclass[dvipdfmx,uplatex,b5paper]{jsarticle}
\usepackage{MAstandard}
\usepackage[top=10mm,bottom=10mm]{geometry}

\DeclareMathOperator{\Int}{int}
\DeclareMathOperator{\sign}{sign}
\newcommand{\smooth}{C^{\infty}}
\newcommand{\Vector}[1]{\overrightarrow{\text{#1}}}

\crefname{theorem}{定理}{定理}
\crefname{proposition}{命題}{命題}
\crefname{lemma}{補題}{補題}
\crefname{definition}{定義}{定義}
\crefname{example}{例}{例}
\crefname{remark}{注}{注}

\begin{document}

\section*{高1D冬期第1週 宿題解答}
\subsection*{(104)}
\begin{enumarabicp}
  \item \begin{align*}
    I_1&= \int_1^2 (x^3+x^2)dx 
        = \bracketlr{\frac{1}{4}x^4+\frac{1}{3}x^3}_1^2 
        = \parenlr{4+\frac{8}{3}}-\parenlr{\frac{1}{4}+\frac{1}{3}} 
        = \boxed{\frac{73}{12}}
  \end{align*}

  \item \begin{align*}
    I_2&= \int_0^1 (2x+1)^2dx
       = \int _0^1(4x^2+4x+1)dx \\
       &= \bracketlr{\frac{4}{3}x^3+2x^2+x}_0^1
       = \frac{4}{3}+2+1=\boxed{\frac{13}{3}}
  \end{align*}

  \item \begin{align*}
    I_3 = \int_0^1 (x^3-x^2)dx
        = \bracketlr{\frac{1}{4}x^4-\frac{1}{3}x^3}_0^1
        = \frac{1}{4}-\frac{1}{3} = \boxed{-\frac{1}{12}}
  \end{align*}
\end{enumarabicp}

\subsection*{(105)}
$\dfrac{x^3}{2}+1$が$1\leq x\leq 2$で正であることに注意して立式すると
\begin{align*}
  \int_1^2 \parenlr{\frac{x^3}{2}+1}dx = \bracketlr{\frac{x^4}{8}+x}_1^2
  = (2+2)-\parenlr{\frac{1}{8}+1}= \boxed{\frac{23}{8}}
\end{align*}

\subsection*{\underline{復習}}
\[\int_a^b f(x)dx \]
という式の意味は?\ (面積という言葉を使わずに説明してみてください。)
\end{document}