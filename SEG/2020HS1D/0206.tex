\documentclass[dvipdfmx,uplatex,b5paper]{jsarticle}
\usepackage{MAstandard}
\usepackage[top=10mm,bottom=10mm]{geometry}

\DeclareMathOperator{\Int}{int}
\DeclareMathOperator{\sign}{sign}
\newcommand{\smooth}{C^{\infty}}
\newcommand{\Vector}[1]{\overrightarrow{\text{#1}}}
\newcommand{\innerprod}[2]{\overrightarrow{#1}\cdot\overrightarrow{#2}}
\newcommand{\Innerprod}[2]{\Vector{#1}\cdot\Vector{#2}}

\crefname{theorem}{定理}{定理}
\crefname{proposition}{命題}{命題}
\crefname{lemma}{補題}{補題}
\crefname{definition}{定義}{定義}
\crefname{example}{例}{例}
\crefname{remark}{注}{注}


\begin{document}

\section*{高1D 第3週宿題解答}

\begin{enumarabicp}
  \item \[ \Vector{OA}=\begin{pmatrix}
    4 \\ 4 \\ 4
  \end{pmatrix},\ 
  \Vector{OB}= \begin{pmatrix}
    7 \\ -2 \\ -2
  \end{pmatrix} \]
  なので
  \begin{align*}
    \abs{\Vector{OA}}^2&=48,\ \abs{\Vector{OB}}^2=57,\ \Innerprod{OA}{OB} = 12
  \end{align*}
  となる。したがって
  \begin{align*}
    \triangle\text{OAB} = \frac{1}{2}\sqrt{\abs{\Vector{OA}}^2\abs{\Vector{OB}}^2-(\Innerprod{OA}{OB})^2} = \boxed{18\sqrt{2}}
  \end{align*}
  である。
  \item \[ \overrightarrow{n} = \begin{pmatrix}
    p \\ q \\ r
  \end{pmatrix} \]
  とおくと、$\overrightarrow{n}$は$\Vector{OA}$および$\Vector{OB}$と垂直だから
  \begin{align*}
    4p+4q+4r &= 0, \\
    7p-2q-2r &= 0
  \end{align*}
  が成立する。これを$p,\ q$について解くと
  \[ p=0,\ q=-r\]
  となる。したがって
  \[\overrightarrow{n}= r\begin{pmatrix}
    0 \\ -1 \\ 1
  \end{pmatrix} \]
  だが、長さが1になるように$r$を決めると
  \[ r=\pm \frac{1}{\sqrt{2}} \]
  となる。よって$\overrightarrow{n}$としては
  \[ \pm \frac{1}{\sqrt{2}}\begin{pmatrix}
    0 \\ 1 \\ -1
  \end{pmatrix} \]
  がとれる(どちらでも正解)。
  以下では
  \[ \overrightarrow{n}= \frac{1}{\sqrt{2}}\begin{pmatrix}
    0 \\ 1 \\ -1
  \end{pmatrix}\]
  であるとする。
  \item \[\Vector{CO} = \begin{pmatrix}
    -6 \\ -3 \\ 3
  \end{pmatrix} \]
  の$\overrightarrow{n}$方向への正射影が$\Vector{CH}$なので、
  \[ (\Vector{CH}\text{の符号付長さ}) = \Vector{CO}\cdot \overrightarrow{n}=-3\sqrt{2} \]
  である。したがってCHの長さは$\boxed{3\sqrt{2}}$である。
  \item \[ \frac{1}{3}\times 18\sqrt{2}\times 3\sqrt{2} = \boxed{36} \]
\end{enumarabicp}

\end{document}