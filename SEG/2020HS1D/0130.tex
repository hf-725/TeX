\documentclass[dvipdfmx,uplatex,b5paper]{jsarticle}
\usepackage{MAstandard}
\usepackage[top=10mm,bottom=10mm]{geometry}

\DeclareMathOperator{\Int}{int}
\DeclareMathOperator{\sign}{sign}
\newcommand{\smooth}{C^{\infty}}
\newcommand{\Vector}[1]{\overrightarrow{\text{#1}}}

\crefname{theorem}{定理}{定理}
\crefname{proposition}{命題}{命題}
\crefname{lemma}{補題}{補題}
\crefname{definition}{定義}{定義}
\crefname{example}{例}{例}
\crefname{remark}{注}{注}

\begin{document}

\section*{高1D 第3週宿題}
$\text{O}(0,0,0),\ \text{A}(4,4,4),\ \text{B}(7,-2,-2),\ \text{C}(6,3,-3)$とする。四面体OABCの
体積を求めたい。以下では$\triangle\text{OAB}$を底面とみて考えることにする。
\begin{enumarabicp}
  \item $\triangle\text{OAB}$の面積を求めよ。
  \item テキスト(303)を参考にして、平面OABに垂直な単位ベクトル$\overrightarrow{n}$を一つ求めよ
  ($\overrightarrow{n}$はどんなベクトルと垂直になるか?)。
  \item 点Cから平面OABに下ろした垂線の足をHとする。CHの長さを求めよ。
  \item 四面体OABCの体積を求めよ。
\end{enumarabicp}
\end{document}