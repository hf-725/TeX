\documentclass[dvipdfmx,uplatex,b5paper]{jsarticle}
\usepackage{MAstandard}
\usepackage[top=10mm,bottom=10mm]{geometry}

\DeclareMathOperator{\Int}{int}
\DeclareMathOperator{\sign}{sign}
\newcommand{\smooth}{C^{\infty}}
\newcommand{\Vector}[1]{\overrightarrow{\text{#1}}}

\crefname{theorem}{定理}{定理}
\crefname{proposition}{命題}{命題}
\crefname{lemma}{補題}{補題}
\crefname{definition}{定義}{定義}
\crefname{example}{例}{例}
\crefname{remark}{注}{注}

\begin{document}

\subsection*{(302)}
三角形ABCにおいて,ABを$2:1$に内分する点をM,ACを$3:2$に内分する点をN,BNとCMの交点をPとする.
$\text{BP}:\text{PN}$と$\text{CP}:\text{PM}$を求めたい.$\overrightarrow{b}=\Vector{AB},\ \overrightarrow{c}=\Vector{AC}$
とおく.
\begin{enumarabicp}
  \item $\text{BP}:\text{PN}=t:1-t$とおくとき,$\Vector{AP}$を$t$と$\overrightarrow{b},
  \overrightarrow{c}$を用いて表せ.
  \item $\text{CP}:\text{PM}=s:1-s$とおくとき,$\Vector{AP}$を$s$と$\overrightarrow{b},
  \overrightarrow{c}$を用いて表せ.
  \item $s,t$を求め,$\text{BP}:\text{PN}$と$\text{CP}:\text{PM}$を求めよ.
\end{enumarabicp}


\end{document}