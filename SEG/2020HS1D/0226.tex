\documentclass[dvipdfmx,uplatex,b5paper]{jsarticle}
\usepackage{MAstandard}
\usepackage[top=10mm,bottom=10mm]{geometry}

\DeclareMathOperator{\Int}{int}
\DeclareMathOperator{\sign}{sign}
\newcommand{\smooth}{C^{\infty}}
\newcommand{\Vector}[1]{\overrightarrow{\text{#1}}}
\newcommand{\innerprod}[2]{\overrightarrow{#1}\cdot\overrightarrow{#2}}
\newcommand{\Innerprod}[2]{\Vector{#1}\cdot\Vector{#2}}

\crefname{theorem}{定理}{定理}
\crefname{proposition}{命題}{命題}
\crefname{lemma}{補題}{補題}
\crefname{definition}{定義}{定義}
\crefname{example}{例}{例}
\crefname{remark}{注}{注}

\begin{document}

\section*{高1D第7週 復習問題}
\begin{enumarabicp}
  \item 平面$\alpha\colon 3x+2y+5z-2=0$とA$(2,3,7)$との間の距離を
  \underline{点と平面の距離の公式}
  
  \underline{を使わずに}求めよ。

  \item 3点A$(1,3,5)$,\ B$(2,-1,2)$,\ C$(3,0,0)$を通る平面$\beta$とP$(5,3,2)$との
  間の距離を求めよ(こちらでは点と平面の距離の公式を用いてよい)。
\end{enumarabicp}

\end{document}