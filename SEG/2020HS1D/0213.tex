\documentclass[dvipdfmx,uplatex,b5paper]{jsarticle}
\usepackage{MAstandard}
\usepackage[top=10mm,bottom=10mm]{geometry}

\DeclareMathOperator{\Int}{int}
\DeclareMathOperator{\sign}{sign}
\newcommand{\smooth}{C^{\infty}}
\newcommand{\Vector}[1]{\overrightarrow{\text{#1}}}

\crefname{theorem}{定理}{定理}
\crefname{proposition}{命題}{命題}
\crefname{lemma}{補題}{補題}
\crefname{definition}{定義}{定義}
\crefname{example}{例}{例}
\crefname{remark}{注}{注}
\pagestyle{empty}

\begin{document}
\section*{高1D 第5週復習問題}
$(6,2,5)$を通り方向ベクトルが$\begin{pmatrix} 2 \\ 1 \\-2 \end{pmatrix}$の直線$l$と、
$(8,2,8)$と$(-4,-1,5)$を通る直線$m$がある。
\begin{enumarabicp}
  \item $l$と$m$が交わるかを調べ、交わる場合は交点の座標を求めよ。
  \item $(8,1,5)$から$l$に下した垂線の足の座標を求めよ。
\end{enumarabicp}

\newpage
\section*{平面の式まとめ}
\vspace{\baselineskip}
\begin{enumarabicp}
  \item 一般に
    \[ \overrightarrow{n} = \begin{pmatrix}
    a \\ b \\ c
  \end{pmatrix} \]
  に垂直で$(x_0,y_0,z_0)$を通る平面は
  \begin{center}
    \framebox[6cm]{\rule[-1zw]{0pt}{3zw}}
  \end{center}
  で表される。
  \vspace{\baselineskip}
  \item 逆に$a,\ b,\ c$の少なくとも1つが0でないとき、
  のとき、
  \[ ax+by+cz+d=0 \]
  の表す図形は
  \begin{center}
    \framebox[3cm][l]{\rule[-3zw]{0pt}{7zw}$\overrightarrow{n}=$}
  \end{center}
  を法線ベクトルとする平面である。
\end{enumarabicp}

\end{document}