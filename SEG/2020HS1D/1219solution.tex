\documentclass[dvipdfmx,uplatex,b5paper]{jsarticle}
\usepackage{MAstandard}
\usepackage[top=10mm,bottom=10mm]{geometry}

\DeclareMathOperator{\Int}{int}
\DeclareMathOperator{\sign}{sign}
\newcommand{\smooth}{C^{\infty}}
\newcommand{\Vector}[1]{\overrightarrow{\text{#1}}}

\crefname{theorem}{定理}{定理}
\crefname{proposition}{命題}{命題}
\crefname{lemma}{補題}{補題}
\crefname{definition}{定義}{定義}
\crefname{example}{例}{例}
\crefname{remark}{注}{注}

\begin{document}

\section*{高1D冬期第3週 宿題解答}
\subsection*{(205)}
\begin{enumarabicp}
  \item[(2)] \begin{align*}
    I_2 &= \int_\alpha^\beta -(x-\alpha)(x-\beta)dx
        = \int_0^{\beta-\alpha} -x\{x-(\beta-\alpha)\}dx \\
        &= \bracketlr{-\frac{1}{3}x^3+\frac{1}{2}(\beta-\alpha)x^2}_0^{\beta-\alpha}
        = -\frac{1}{3}(\beta-\alpha)^3+\frac{1}{2}(\beta-\alpha)^3 \\
        &= \boxed{\frac{1}{6}(\beta-\alpha)^3}
  \end{align*}
  \item[(3)] \begin{align*}
    I_4 &= 2\int_{-1}^3 (x+1)(x-3)dx 
        = -2\times \frac{1}{6}\times (3+1)^3 
        = \boxed{-\frac{64}{3}}
  \end{align*}
  \item[(4)] \begin{align*}
    I_5 &= 6\int_{-\frac{1}{3}}^{\frac{1}{2}} \parenlr{x+\frac{1}{3}}\parenlr{x-\frac{1}{2}} dx 
        =  -6\times \frac{1}{6}\times \parenlr{\frac{1}{2}-\frac{1}{3}}
        = \boxed{-\frac{125}{216}}
  \end{align*}
\end{enumarabicp}

\subsection*{(206)}
\begin{enumarabicp}
  \item \begin{align*}
    I_1 &= \int_{2-\sqrt{3}}^{2+\sqrt{3}} \bracelr{x-\parenlr{2-\sqrt{3}}} \bracelr{x-\parenlr{2+\sqrt{3}}} dx \\
        &= -\frac{1}{6}\times \bracelr{(2+\sqrt{3})-(2-\sqrt{3})}^3 = \boxed{-4\sqrt{3}}
  \end{align*}

  \item \begin{align*}
    I_2 &= \int_1^2 (x-1)^2(x-2)dx
        = \int_0^1 x^2(x-1)dx \\
        &= \bracketlr{\frac{1}{4}x^4-\frac{1}{3}x^3}_0^1
        = \frac{1}{4}-\frac{1}{3}=\boxed{-\frac{1}{12}}
  \end{align*}

  $\spadesuit$ 一般に
  \[\int_\alpha^\beta (x-\alpha)^2(x-\beta)dx \]
  の値はどうなるか?
\end{enumarabicp}

\end{document}