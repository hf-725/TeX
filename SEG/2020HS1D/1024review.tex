\documentclass[b5paper, uplatex]{jsarticle}
\usepackage{MAstandard, appendix}
\usepackage[top=10mm,bottom=10mm]{geometry}

\DeclareMathOperator{\Int}{int}
\DeclareMathOperator{\sign}{sign}
\newcommand{\Vector}[1]{\overrightarrow{\text{#1}}}
\newcommand{\smooth}{C^{\infty}}

\crefname{theorem}{定理}{定理}
\crefname{proposition}{命題}{命題}
\crefname{lemma}{補題}{補題}
\crefname{definition}{定義}{定義}
\crefname{example}{例}{例}
\crefname{remark}{注}{注}

\begin{document}

\section*{高1D 第8週復習問題}
\begin{enumarabicp}
  \item 線分ABを$t:1-t$に内分する点をPとし,Oを定点とする.このとき,$\overrightarrow{\text{OP}}$を
  $\overrightarrow{\text{OA}}$と$\overrightarrow{\text{OB}}$を用いて表せ(内分点公式に当てはめずに答えを出すこと).
  \item 三角形ABCと定点Oがある.三角形ABCの重心をGとおく.
  \begin{enumromanp}
    \item 線分ABの中点をMとおく.$\overrightarrow{\text{OM}}$を$\overrightarrow{\text{OA}}$と
    $\Vector{OB}$を用いて表せ.
    \item $\Vector{OG}$を$\Vector{OA},\ \Vector{OB},\ \Vector{OC}$を
    用いて表せ.
    \item $\text{A}(2,7),\ \text{B}(4,2),\ \text{C}(9,-3)$のとき,Gの座標を求めよ.
  \end{enumromanp}
\end{enumarabicp}
\end{document}
